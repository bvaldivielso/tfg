De camino a nuestro objetivo de definir el concepto de mónada
pasaremos por la definición de adjunción. Como veremos en las próximas
secciones las mónadas y las adjunciones están estrechamente relacionadas.
Esto no debe hacernos pensar que las mónadas son el único motivo
por el que las adjunciones son importante en matemáticas. Las adjunciones
han sido identificadas por algunos autores como uno de los conceptos
clave de la teoría de categorías. McLane resume la importancia de
esta noción diciendo que ``hay funtores adjuntos por todas partes''
\cite{ariseeverywhere}.

Empezaremos por precisar qué es una adjunción.

\subsection{Definición}
\begin{definition}
  Una adjunción entre categorías $\C$ y $\D$ es un par de funtores
  $F : \arr{\C}{\D}$ y $G: \arr{\D}{\C}$ tales que los funtores
  $\Hom_\D(F-, -) : \arr{\C^{op}\times\D}{\Set}$ y
  $\Hom_\C(-, G-) : \arr{\C^{op}\times\D}{\Set}$ son naturalmente isomorfos.

  Diremos que $F$ es el
  adjunto por la izquierda de $G$ y que $G$ es el adjunto
  por la derecha de $F$. Notaremos esta relación entre ambos funtores
  como $F \adjoint G$.
\end{definition}

Nótese que para cualquier
objeto $C$ de $\C$ y cualquier objeto
$D$ de $\D$ un par de funtores adjuntos $F \adjoint G$ nos permite
establecer biyecciones entre los conjuntos:

$$\Hom_\D(FC, D) \cong \Hom_\C(C, GD)$$

Sea $\phi_{C, D} : \arr{\Hom_\D(FC, D)}{\Hom_\C(C, GD)}$ una tal
familia de biyecciones. Comprobar que $\phi$ es natural se reduce
a, por un lado, comprobar que el siguiente diagrama en $\D$:

\begin{center}
  \begin{tikzcd}
    FC \arrow{r}{f} & D \arrow{r}{g} & D'
  \end{tikzcd}
\end{center}

Se traslada al siguiente diagrama conmutativo en $\C$:

\begin{center}
  \begin{tikzcd}
    C \arrow[bend right=50, swap]{rr}{\phi_{C, D'}(g \circ f)}
      \arrow{r}{\phi_{C,D}(f)} & GD \arrow{r}{G g} & GD'
  \end{tikzcd}
\end{center}

Y por otro lado, que el siguiente diagrama en $\C$:

\begin{center}
  \begin{tikzcd}
    C' \arrow{r}{h} & C \arrow{r}{f} & GD
  \end{tikzcd}
\end{center}

se traslada al siguiente diagrama conmutativo en $\D$

\begin{center}
  \begin{tikzcd}
    FC' \arrow[bend right=50, swap]{rr}{\phi^{-1}_{C', D}(f \circ h)}
        \arrow{r}{Fh} & FC \arrow{r}{\phi^{-1}_{C, D}(f)} & D
  \end{tikzcd}
\end{center}

\subsection{Unidad y counidad}
Dadas categorías $\C$ y $\D$ y funtores adjuntos $F \adjoint G$ es
posible construir dos transformaciones naturales
$\eta : \nat{1_\C}{GF}$ y $\epsilon: \nat{FG}{1_\D}$ a las que llamaremos
respectivamente la \textbf{unidad} y la \textbf{counidad} de la
adjunción.

Sea $\phi : \nat{\Hom(F-, -)}{\Hom(-, G-)}$ el isomorfismo natural que
nos viene dado por la adjunción. Damos la transformación $\eta$
por sus componentes
$\eta_C = \phi_{C, FC}(1_{FC}) : \arr{C}{GFC}$.

Veamos que en efecto es natural. Sea $f : \arr{C}{C'}$
una flecha de $\C$. El diagrama cuya conmutatividad
tenemos que verificar es el siguiente:

\begin{center}
  \begin{tikzcd}
    C \arrow{d}{f}
      \arrow{r}{\eta_C} & GFC \arrow{d}{GFf} \\

    C' \arrow{r}{\eta_{C'}} & GFC'
  \end{tikzcd}
\end{center}

Es decir tenemos que comprobar que
$$\phi_{C', FC'}(1_{FC'}) \circ f = \eta_{C'} \circ f = GFf \circ \eta_C
  = GFf \circ \phi_{C, FC}(1_{FC})$$

Pero teniendo cuenta que $\phi$ es natural tenemos que el diagrama

\begin{center}
  \begin{tikzcd}
    FC \arrow{r}{1_{FC}} & FC \arrow{r}{Ff} & FC'
  \end{tikzcd}
\end{center}

Lo podemos transponer en el siguiente diagrama conmutativo:

\begin{center}
  \begin{tikzcd}
    C
    \arrow[bend right=50, swap]{rr}{\phi_{C, FC'}(f \circ 1_C) = \phi_{C, FC'}(Ff) }
      \arrow{r}{\eta_C}
            & GFC \arrow{r}{GFf} & GFC'
  \end{tikzcd}
\end{center}

Luego $GFf \circ \eta_C = \phi_{C, FC'}(Ff)$ pero a su vez:


\begin{center}
  \begin{tikzcd}
    C \arrow{r}{f} & C'
       \arrow{r}{\eta_{C'}} & GFC'
  \end{tikzcd}
\end{center}

Lo podemos transponer en:

\begin{center}
  \begin{tikzcd}
    FC \arrow[bend right=50, swap]{rr}{\phi_{C, FC'}^{-1}(\eta_{C'} \circ f)}
       \arrow{r}{F f} & FC' \arrow{r}{1_{FC'}} & FC'
  \end{tikzcd}
\end{center}

Este último diagrama muestra que
$Ff = \phi^{-1}_{C, FC'}(\eta_{C'} \circ f)$.


Y por tanto
$$GFf \circ \eta_C = \phi_{C, FC'}(Ff) = \phi_{C, FC'}
  (\phi^{-1}_{C,FC'}(\eta_{C'} \circ f)) = \eta_{C'} \circ f$$

Y la naturalidad de $\eta : \nat{1_\C}{GF}$ queda probada.

\subsection{Ejemplos}
\paragraph{Producto}


Consideramos un conjunto $A$ y el funtor $-\times A : \arr{\Set}{\Set}$.

$-\times A$ es el adjunto por la izquierda del funtor $\Hom(A, -)$.
Dicho de otra forma para cualquier conjunto $B$ tenemos:

$$\Hom(B\times A, T) \cong \Hom(B, \Hom(A, T))$$

Esto nos dice que es lo mismo dar una aplicación entre
$B\times A$ y $T$ que una aplicación de $B$ a $\Hom(A, T)$. La
biyección es la siguiente:

$$\phi: \arr{\Hom(B, \Hom(A, T))}{\Hom(B\times A, T)}$$
$$\phi(h)(b, a) = h(b)(a)$$

\paragraph{Grupos libres}

\paragraph{Producto y diagonal}
