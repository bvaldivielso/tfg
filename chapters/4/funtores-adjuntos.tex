De camino a nuestro objetivo de definir el concepto de mónada
pasaremos por la definición de adjunción. Como veremos en las próximas
secciones las mónadas y las adjunciones están estrechamente relacionadas.
Esto no debe hacernos pensar que las mónadas son el único motivo
por el que las adjunciones son importante en matemáticas. Las adjunciones
han sido identificadas por algunos autores como uno de los conceptos
clave de la teoría de categorías. McLane resume la importancia de
esta noción diciendo que ``hay funtores adjuntos por todas partes''
\cite{ariseeverywhere}.

Empezaremos por precisar qué es una adjunción.

\subsection{Definición}
\begin{definition}
  Una adjunción entre categorías $\C$ y $\D$ es un par de funtores
  $F : \arr{\C}{\D}$ y $G: \arr{\D}{\C}$ tales que los funtores
  $\Hom_\D(F-, -) : \arr{\C^{op}\times\D}{\Set}$ y
  $\Hom_\C(-, G-) : \arr{\C^{op}\times\D}{\Set}$ son naturalmente isomorfos.

  Diremos que $F$ es el
  adjunto por la izquierda de $G$ y que $G$ es el adjunto
  por la derecha de $F$. Notaremos esta relación entre ambos funtores
  como $F \adjoint G$.
\end{definition}

Nótese que para cualquier
objeto $C$ de $\C$ y cualquier objeto
$D$ de $\D$ un par de funtores adjuntos $F \adjoint G$ nos permite
establecer biyecciones entre los conjuntos:

$$\Hom_\D(FC, D) \cong \Hom_\C(C, GD)$$

Sea $\phi_{C, D} : \arr{\Hom_\D(FC, D)}{\Hom_\C(C, GD)}$ una tal
familia de biyecciones. Comprobar que $\phi$ es natural se reduce
a, por un lado, comprobar que el siguiente diagrama en $\D$:

\begin{center}
  \begin{tikzcd}
    FC \arrow{r}{f} & D \arrow{r}{g} & D'
  \end{tikzcd}
\end{center}

Se traslada al siguiente diagrama conmutativo en $\C$:

\begin{center}
  \begin{tikzcd}
    C \arrow[bend right=50, swap]{rr}{\phi_{C, D'}(g \circ f)}
      \arrow{r}{\phi_{C,D}(f)} & GD \arrow{r}{G g} & GD'
  \end{tikzcd}
\end{center}

Y por otro lado, que el siguiente diagrama en $\C$:

\begin{center}
  \begin{tikzcd}
    C' \arrow{r}{h} & C \arrow{r}{f} & GD
  \end{tikzcd}
\end{center}

se traslada al siguiente diagrama conmutativo en $\D$

\begin{center}
  \begin{tikzcd}
    FC' \arrow[bend right=50, swap]{rr}{\phi^{-1}_{C', D}(f \circ h)}
        \arrow{r}{Fh} & FC \arrow{r}{\phi^{-1}_{C, D}(f)} & D
  \end{tikzcd}
\end{center}

\subsection{Unidad y counidad}
Dadas categorías $\C$ y $\D$ y funtores adjuntos $F \adjoint G$ es
posible construir dos transformaciones naturales
$\eta : \nat{1_\C}{GF}$ y $\epsilon: \nat{FG}{1_\D}$ a las que llamaremos
respectivamente la \textbf{unidad} y la \textbf{counidad} de la
adjunción.

Sea $\phi : \nat{\Hom(F-, -)}{\Hom(-, G-)}$ el isomorfismo natural que
nos viene dado por la adjunción. Damos la transformación $\eta$
por sus componentes
$\eta_C = \phi_{C, FC}(1_{FC}) : \arr{C}{GFC}$.

Veamos que en efecto es natural. Sea $f : \arr{C}{C'}$
una flecha de $\C$. El diagrama cuya conmutatividad
tenemos que verificar es el siguiente:

\begin{center}
  \begin{tikzcd}
    C \arrow{d}{f}
      \arrow{r}{\eta_C} & GFC \arrow{d}{GFf} \\

    C' \arrow{r}{\eta_{C'}} & GFC'
  \end{tikzcd}
\end{center}

Es decir tenemos que comprobar que
$$\phi_{C', FC'}(1_{FC'}) \circ f = \eta_{C'} \circ f = GFf \circ \eta_C
  = GFf \circ \phi_{C, FC}(1_{FC})$$

Pero teniendo cuenta que $\phi$ es natural tenemos que el diagrama

\begin{center}
  \begin{tikzcd}
    FC \arrow{r}{1_{FC}} & FC \arrow{r}{Ff} & FC'
  \end{tikzcd}
\end{center}

Lo podemos transponer en el siguiente diagrama conmutativo:

\begin{center}
  \begin{tikzcd}
    C
    \arrow[bend right=50, swap]{rr}{\phi_{C, FC'}(f \circ 1_C) = \phi_{C, FC'}(Ff) }
      \arrow{r}{\eta_C}
            & GFC \arrow{r}{GFf} & GFC'
  \end{tikzcd}
\end{center}

Luego $GFf \circ \eta_C = \phi_{C, FC'}(Ff)$ pero a su vez:


\begin{center}
  \begin{tikzcd}
    C \arrow{r}{f} & C'
       \arrow{r}{\eta_{C'}} & GFC'
  \end{tikzcd}
\end{center}

Lo podemos transponer en:

\begin{center}
  \begin{tikzcd}
    FC \arrow[bend right=50, swap]{rr}{\phi_{C, FC'}^{-1}(\eta_{C'} \circ f)}
       \arrow{r}{F f} & FC' \arrow{r}{1_{FC'}} & FC'
  \end{tikzcd}
\end{center}

Este último diagrama muestra que
$Ff = \phi^{-1}_{C, FC'}(\eta_{C'} \circ f)$.


Y por tanto
$$GFf \circ \eta_C = \phi_{C, FC'}(Ff) = \phi_{C, FC'}
  (\phi^{-1}_{C,FC'}(\eta_{C'} \circ f)) = \eta_{C'} \circ f$$

Y la naturalidad de $\eta : \nat{1_\C}{GF}$ queda probada.

\subsection{Ejemplos}
\paragraph{Producto}
Sea $A$ un conjunto. Veamos que los funtores
$-\times A : \arr{\Set}{\Set}$ y
$\Hom(A, -) : \arr{\Set}{\Set}$ conforman una adjunción.
Tenemos que encontrar un isomorfismo natural
$$\phi : \nat{\Hom(-\times A, -)}{\Hom(-, \Hom(A, -))}$$.
Para ello definimos
$$\phi_{B, Z} : \arr{\Hom(B\times A, Z)}{\Hom(B, \Hom(A, Z))}$$
$$\phi_{B, Z}(f)(b)(a) = f(b, a)$$

Veamos que $\phi_{B, Z}$ es una biyección.
En primer lugar vemos que es sobreyectiva.
Sea $f' \in \Hom(B, \Hom(A, Z))$. Definimos la aplicación
$f : \arr{B\times A, Z}$ dada por $f(b, a) = f'(b)(a)$. Aplicando
la definición de $\phi_{B, Z}$
tenemos $\phi_{B,Z}(f)(b)(a) = f(b, a) = f'(b)(a)$ y por tanto
$\phi_{B,Z}(f) = f'$. Veamos que $\phi$ es inyectiva. Supongamos
que tenemos $f_1, f_2 : \arr{B\times A, Z}$ tales que
$\phi_{B, Z}(f_1) = \phi_{B, Z}(f_2)$ pero entonces tenemos que
$f_1(b, a) = \phi_{B, Z}(f_1)(b)(a) = \phi_{B,Z}(f_2)(b)(a) = f_2(b, a)$
y por tanto $f_1 = f_2$.

Comprobamos ahora las condiciones de naturalidad. En primer lugar dado
el siguiente diagrama en $\C$
\begin{center}
  \begin{tikzcd}
    B \arrow{r}{f} & B' \arrow{r}{g} & \Hom(A, Z)
  \end{tikzcd}
\end{center}
Hemos de ver que el siguiente diagrama es conmutativo:
\begin{center}
  \begin{tikzcd}
    {B\times A}
      \arrow[bend right=50, swap]{rr}{\phi^{-1}_{B, Z}(g \circ f)}
      \arrow{r}{f\times A} & B'\times A \arrow{r}{\phi^{-1}_{B', Z}(g)} & Z
  \end{tikzcd}
\end{center}
Y esto equivale a probar
$$\phi_{B,  Z}(g \circ f) = \phi_{B', Z}(g) \circ (f\times A)$$
$$f\times A: \arr{B\times A}{B'\times A}$$
donde $f \times A$ está definida por
$(f\times A)(b, a) = (f(b), a)$. Aplicando sobre un valor cualquiera
$(b, a) \in B\times A$ tenemos que:
$$\phi_{B, Z}^{-1}(g \circ f)(b, a) = (g\circ f)(b)(a) = g(f(b))(a)$$
y
$$(\phi_{B', Z}^{-1}(g) \circ (f\times A))(b, a) =
   \phi_{B', Z}^{-1}(g)(f(b), a) = g(f(b))(a)$$
En segundo lugar dado el siguiente diagrama:
\begin{center}
  \begin{tikzcd}
    B\times A \arrow{r}{f} & Z \arrow{r}{g} & Z'
  \end{tikzcd}
\end{center}
Hemos de ver que el siguiene diagrama es conmutativo:
\begin{center}
  \begin{tikzcd}
    B \arrow[bend right=50, swap]{rr}{\phi_{B, Z'}(g \circ f)}
      \arrow{r}{\phi_{B, Z}(f)} & \Hom(A, Z) \arrow{r}{\Hom(A, g)} & \Hom(A, Z')
  \end{tikzcd}
\end{center}
Es decir, $\phi_{B, Z'}(g \circ f) = \Hom(A, g) \circ \phi_{B, Z}(f)$. Dados
$b \in B, a \in A$ tenemos que:
$$\phi_{B, Z'}(g \circ f)(b)(a) = (g\circ f)(b, a) = g(f(b, a))$$
y
\begin{multline*}
(\Hom(A, g)\circ\phi_{B, Z}(f))(b)(a) = (\Hom(A, g)(\phi_{B, Z}(f)(b))(a)\\
  = (g \circ \phi_{B,Z}(f)(b))(a) = g(f(b, a))
\end{multline*}
Y por tanto $\phi$ es transformación natural.

Esta adjunción nos dice que dar una aplicación que va de $B\times A$ a $Z$
es esencialmente lo mismo que dar una aplicación que vaya de $B$ al conjunto
de aplicaciones entre $A$ y $Z$.
\paragraph{Grupos libres}
El funtor de conjunto subyacente $U : \arr{\Grp}{\Set}$ y el funtor
$F : \arr{\Set}{\Grp}$ que asigna a cada conjunto su grupo libre asociado
son un par de funtores adjuntos $F \adjoint U$.

La unidad $\eta : \nat{1_\Set}{UF}$ es la inclusión
$\eta_X : X \subseteq UFX$ de un conjunto en el conjunto subyacente
al grupo libre generado por él. La counidad
$\mu : \nat{FU}{1_\Grp}$ nos da para cada grupo $G$
el homomorfismo de grupos $\mu_G : \arr{FUG}{G}$
tal que a cada palabra
$g_1g_2\ldots g_n \in FUG$ le asigna el valor $\prod_{i=1}^n g_i \in G$.
