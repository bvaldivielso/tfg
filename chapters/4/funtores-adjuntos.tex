De camino a nuestro objetivo de definir las mónadas tendremos
que pasar previamente por la noción de adjunción. Aunque
es cierto, como veremos en las próximas secciones, que cada par
de funtores adjuntos permite construir una mónada, este no es
ni de lejos el
único motivo por el que las adjunciones son importantes en matemáticas.

\begin{quotation}
  adjunction arises everywhere
\end{quotation}

Empezaremos por dar la definición.

\subsection{Definición}
\begin{definition}
  Una adjunción entre categorías $\C$ y $\D$ es un par de funtores
  $F : \arr{\C}{\D}$ y $G: \arr{\D}{\C}$ tales que los funtores
  $\Hom_\D(F-, -) : \arr{\C^{op}\times\D}{\Set}$ y
  $\Hom_\C(-, G-) : \arr{\C^{op}\times\D}{\Set}$ son naturalmente isomorfos.

  Diremos que $F$ es el
  adjunto por la izquierda de $G$ y que $G$ es el adjunto
  por la derecha de $F$. Notaremos esta relación entre ambos funtores
  como $F \adjoint G$.
\end{definition}

\subsection{Ejemplos}
\paragraph{Producto}
Consideramos un conjunto $A$ y el funtor $-\times A : \arr{\Set}{\Set}$.

$-\times A$ es el adjunto por la izquierda del funtor $\Hom(A, -)$.
Dicho de otra forma para cualquier conjunto $B$ tenemos:

$$\Hom(B\times A, T) \cong \Hom(B, \Hom(A, T))$$

Esto nos dice que es lo mismo dar una aplicación entre
$B\times A$ y $T$ que una aplicación de $B$ a $\Hom(A, T)$. La
biyección es la siguiente:

$$\phi: \arr{\Hom(B, \Hom(A, T))}{\Hom(B\times A, T)}$$
$$\phi(h)(b, a) = h(b)(a)$$

\paragraph{Grupos libres}

\paragraph{Producto y diagonal}
