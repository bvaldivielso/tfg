Comenzamos con el estudio de una de las nociones nacidas
en teoría de categorías que son
más importantes para nuestro trabajo: las mónadas.


\subsection{Definición}
\begin{definition}
  Una mónada sobre una categoría $\C$
  es una terna $(T, \eta, \mu)$ donde $T : \arr{\C}{\C}$ es un
  endofuntor y $\eta : \nat{1_\C}{T}, \, \mu: \nat{T^2}{T}$ \textcolor{red}{son} dos
  transformaciones naturales tales que los siguientes diagramas
  conmutan:


  \begin{center}
    \begin{tikzcd}
      T^3 \arrow{d}{\mu T} \arrow{r}{T\mu} & T^2 \arrow{d}{\mu} \\
      T^2 \arrow{r}{\mu} & T

    \end{tikzcd}

    \begin{tikzcd}
      T \ar[equal]{dr} \arrow{r}{T\eta} & T^2 \arrow{d}{\mu} & T \arrow{l}[swap]{\eta T} \ar[equal]{ld} \\
      & T &
    \end{tikzcd}
  \end{center}
\end{definition}

\textcolor{red}{los diagramas quedan mejor si los pones uno al lado del otro con un quad}
Estudiar algunos ejemplos será útil para aclarar qué parte hace qué
en la definición.

\subsubsection{Ejemplos}
\paragraph{Monoides}
\textit{Aquí estoy tratando con mucha ligereza la asociatividad del producto cartesiano pero es por simplificar la notación}\textcolor{red}{me parece bien, pero dilo}
Sea $M$ un monoide. Definimos el funtor $T : \arr{\Set}{\Set}$
como $T(X) = M\times X$, $\eta_X : \arr{X}{T(X)}$ definida
por $\eta_X(x) = (e, x)$ (con $e \in M$ el elemento neutro)
y $\mu_X : \arr{T^2(X)}{T(X)}$ definida por
$\mu_X(m, n, x) = (m\cdot n, x)$ \textcolor{red}{acuérdate de los puntos al acabar párrafo}


Veamos que $(T, \eta, \mu)$ es un monoide. \textcolor{red}{mónada}


\textcolor{red}{queda un poco farragoso, si después de una coma empiezas con un símbolo matemático no se sabe si sigues con las matemáticas o...yo haría algo así: Sea $M$ un monoide, veamos que la siguiente terna $(T, \eta, \mu)$ es una mónada, donde (casi con un itemize) $T$.........}

En primer lugar
tenemos que ver que para cualquier conjunto $X$ el siguiente
diagrama es conmutativo:

\begin{center}
  \begin{tikzcd}
    M\times M \times M \times X \arrow{r}{T\mu_X}
           \arrow{d}{\mu_{TX}} & M\times M\times X \arrow{d}{\mu_X} \\
    M\times M\times X \arrow{r}{\mu_X} & M \times X
  \end{tikzcd}
\end{center}
Es decir $\mu_X \circ \mu_{TX} = \mu_X \circ T\mu_X$. Para comprobar
esta igualdad de aplicaciones \textcolor{red}{yo quitaría lo de aplicaciones} tomamos $(m, n, k, x) \in T^3(X)$.
Por un lado:
$$(\mu_X \circ \mu_{TX})(m, n, k, x) = \mu_X(\mu_{TX}(m, n, k, x)
  = \mu_X(m\cdot n, k, x) = ((m\cdot n)\cdot k, x)$$
Por otro lado:
$$(\mu_X \circ T\mu_X)(m, n, k, x) = \mu_X(T\mu_X(m, n, k, x))
  = \mu_X(m, n\cdot k, x) = (m \cdot (n\cdot k), x)$$
y por la asociatividad del producto en $M$ tenemos que
el diagrama es conmutativo.
Queda ver que para cualquier conjunto $X$ el diagrama
\begin{center}
  \begin{tikzcd}
    T(X) \ar[equal]{dr}
         \arrow{r}{\eta_{TX}} & T^2(X) \arrow{d}{\mu_X} & \arrow{l}[swap]{T\eta_X} \ar[equal]{ld} T(X) \\
    & T(X) &
  \end{tikzcd}
\end{center}
es conmutativo, es decir, $\mu_X \circ T\eta_X = \mu_X \circ \eta_{TX}$. Podemos
comprobar esta igualdad tomando $(m, x) \in T^2(X)$:
$$\mu_X(T\eta_X(m, x)) = \mu_X(m, e, x) = (m \cdot e, x) = (m, x)$$
$$\mu_X(\eta_{TX}(m, x)) = \mu_X(e, m, x) = (e\cdot m, x) = (m, x)$$

\paragraph{Sumas}
Sea $\C$ una categoría con coproductos finitos. Dado un objeto
$C$ de $\C$ de definimos:

\begin{itemize}
\item El endofuntor $T: \arr{\C}{\C}$ dado por $T(X) = C + X$
\item La transformación natural $\eta : \nat{1_\C}{T}$ dada por
  $\eta_X : \arr{X}{C+X}$ la inyección canónica en el coproducto.
\item La transformación natural $\mu: \nat{T^2}{T}$ dada por
  $\mu_X : \arr{C + (C + X)}{C + X}$ \textit{Definir}
\end{itemize}

La terna $(T, \eta, \mu)$ es una mónada.


\paragraph{$\Hom(A, -)$}
Sea $A$ un conjunto. Definimos:

\begin{itemize}
\item El endofuntor $T = \Hom(A, -) : \arr{\Set}{\Set}$
\item La transformación natural $\eta : \nat{1_\Set}{T}$ dada
  por $\eta_X : \arr{X}{\Hom(A, X)}$ con $\eta_X(x)(a) = x$.
\item La transformación natural $\mu: \nat{T^2}{T}$ dada por
  $\mu_X : \arr{\Hom(A, \Hom(A, X))}{\Hom(A, X)}$ con
  $\mu_X(f)(a) = f(a)(a)$
\end{itemize}

La terna $(T, \eta, \mu)$ es una mónada.

\subsection{Las adjunciones dan lugar a mónadas}
\begin{theorem}
Sean $\C$ y $\D$ dos categorías y $F : \arr{\C}{\D}$ y $G: \arr{\D}{\C}$ un
par de funtores adjuntos $F \adjoint G$ con unidad
$\eta : \nat{1_{\C}}{GF}$ y counidad $\epsilon: \nat{FG}{1_\D}$. La terna
$(GF, \eta, G\epsilon F)$ es una mónada.
\end{theorem}
\begin{proof}
  El primer diagrama cuya conmutatividad probaremos es:

  \begin{center}
    \begin{tikzcd}
      GF \ar[equal]{rd} \arrow{r}{\eta GF} & GFGF \arrow{d}{G\epsilon F} & GF \arrow{l}[swap]{GF\eta} \ar[equal]{ld} \\
         & GF   &
    \end{tikzcd}
  \end{center}
\end{proof}

Pero este resultado lo sabemos. El siguiente resultado.

\subsection{Categoría de Kleisli de una mónada}
Hemos visto que todo par de funtores adjuntos definen una mónada.
Nos planteamos ahora el recíproco: dada una mónada, ¿existe un par
de funtores adjuntos que la definen?

Veamos que las respuesta es afirmativa.

\begin{theorem}[Categoría de Kleisli]
Sea $\C$ una categoría y $(T: \arr{\C}{\C}, \eta: \nat{1_\C}{T}, \mu: \nat{T^2}{T})$
una mónada sobre $\C$. Entonces:

\begin{enumerate}
\item Podemos construir una categoría $\D$ que tiene los mismos objetos
  de $\C$ y en la que $\Hom_\D(A, B) = \Hom_\C(A, TB)$ para cualquier
  par de objetos de $\C$. $\D$ es la categoría de Kleisli de la mónada.


\item Podemos definir $F: \arr{\C}{\D}$ que lleva cualquier objeto
  de $A$ en si mismo y cualquier flecha $f : \arr{A}{B}$ en la flecha
  $Ff = Tf \circ \eta_A$. $F$ es un funtor.

\item Podemos definir $G: \arr{\D}{\C}$ con $GA = TA$ para todo objeto
  $A$ de $\D$ y dada $f: \Hom_\D(A, B) = \Hom_\C(A, TB)$
  $$Gf =  \mu_B \circ Tf$$

\item $F$ y $G$ son un par de funtores adjuntos $F \adjoint G$
  con $T = GF$, su
  unidad es $\eta$ y además $\mu = F\epsilon G$ con $\epsilon$
  la counidad de la adjunción.
\end{enumerate}
\end{theorem}
\begin{proof}
  Probamos el resultado paso por paso.
  \begin{enumerate}
  \item $\D$ es una categoría. Dadas dos flechas $f : \Hom_\D(A, B)$ y
    $g: \Hom_\D(B, C)$ definimos la composición $g \circ_\D f : \Hom_\D(A, C)$
    como la flecha $g \circ_\D f = \mu_{C} \circ Tg \circ f$. Esta operación
    de composición cumple los axiomas:

    \begin{itemize}
    \item Es asociativa. Sean $f : \Hom_\D(A, B)$, $g : \Hom_\D(B, C)$
      y $h : \Hom_\D(C, D)$ entonces
      $$(h \circ_{\D} g) \circ_\D f = \mu_D \circ T(h \circ_\D g) \circ f
      = \mu_D \circ T(\mu_D \circ Th \circ g) \circ f
      = \mu_D \circ T\mu_D \circ T^2h \circ Tg \circ f$$
      $$= \mu_D \circ \mu_{TD} \circ T^2h \circ Tg \circ f
      =\mu_D \circ Th \circ \mu_C \circ Tg \circ f
      =\mu_D \circ Th \circ (g \circ_\D f)
      = h \circ_\D (g \circ_\D f)$$

    \item Existen identidades. La identidad es $\eta_A \in \Hom_\D(A, A)$. Lo
      comprobamos:
      $$f \circ_\D \eta_A = \mu_B \circ Tf \circ \eta_A$$
      $$= \mu_B \circ \eta_{TB} \circ f$$
      $$= f $$
    \end{itemize}
  \item $F$ es un funtor. Vamos por partes:
    \begin{itemize}
    \item $F$ se lleva bien con la composición.
      $A \xrightarrow{f} B \xrightarrow{C}$ tenemos que
      $$F(g \circ f) = T(g \circ f) \circ \eta_A = Tg \circ Tf \circ \eta_A
      = Tg \circ \eta_B \circ f$$

      $$Fg \circ_\D Ff = (Tg \circ \eta_B) \circ_\D (Tf \circ \eta_A)
      = \mu_C \circ T(Tg \circ \eta_B) \circ (Tf \circ \eta_A)
      = \mu_C \circ T^2g \circ T\eta_B \circ Tf \circ \eta_A$$
      $$= \mu_C \circ T^2g \circ T\eta_B \circ \eta_B \circ f
      = Tg \circ \mu_B \circ T\eta_B \circ \eta_B \circ f
      = Tg \circ \eta_B \circ f$$

      %$$=\mu_C \circ TFg \circ Tf \circ \eta_A$$

    \item $F$ lleva identidades a identidades: $F1_A = T1_A \circ \eta_A = 1_TA \circ \eta_A = \eta_A$ que
    sabemos que es la identidad.

    \end{itemize}

  \item $G$ es un funtor. Por partes
    \begin{itemize}
    \item $G$ se lleva bien con la composición. Dadas $f \in \Hom_\D(A, B)$
      y $g \in \Hom_\D(B, C)$ tenemos que:
      $$G(g \circ_\D f) = \mu_C \circ T(g \circ_\D f)
      = \mu_C \circ T(\mu_C \circ Tg \circ f)
      = \mu_C \circ T\mu_C \circ T^2g \circ Tf$$
      $$= \mu_C \circ \mu_{TC} \circ T^2g \circ Tf
      = \mu_C \circ Tg \mu_B \circ Tf
      = Gg \circ Gf$$
    \item $G$ se lleva bien con las identidades.
      $G\eta_A = \mu_A \circ T\eta_A = 1_{TA}$
    \end{itemize}
  \item El cuarto punto se deja como ejercicio.

  \end{enumerate}


  Diagramas para la demostración:
  \begin{center}
    \begin{tikzcd}
      T^2 B \arrow{d}{T^2g} \arrow{r}{\mu_B} & T B \arrow{d}{T g} \\
      T^2 (TC) \arrow{r}{\mu_{TC}} & TC
    \end{tikzcd}
  \end{center}
  \begin{center}
    \begin{tikzcd}
      T^2B \arrow{d}{T^2g} \arrow{r}{\mu_B} & T B \arrow{d}{T g} \\
      T^2 C \arrow{r}{\mu_C} & T C
    \end{tikzcd}
  \end{center}
  \begin{center}
    \begin{tikzcd}
      T^2 C \arrow{d}{T^2 h} \arrow{r}{\mu_C} & T C \arrow{d}{T h}\\
      T^2 (T D) \arrow{r}{\mu_{TD}} & T (T D)
    \end{tikzcd}
  \end{center}

  \begin{center}
    \begin{tikzcd}
      A \arrow{d}{f} \arrow{r}{\eta_A} & TA \arrow{d}{T f} \\
      T B \arrow{r}{\eta_{TB}} & T (T B)
    \end{tikzcd}
  \end{center}
\end{proof}
