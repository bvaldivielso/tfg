Comenzamos con el estudio de una de las nociones nacidas
en teoría de categorías que son
más importantes para nuestro trabajo: las mónadas.


\subsection{Definición}
\begin{definition}
  Una mónada sobre una categoría $\C$
  es una terna $(T, \eta, \mu)$ donde $T : \arr{C}{C}$ es un
  endofuntor y $\eta : \nat{1_\C}{T}$ $\mu: \nat{T^2}{T}$ dos
  transformaciones naturales tales que los siguientes diagramas
  conmutan:


  \begin{center}
    \begin{tikzcd}
      T^3 \arrow{d}{\mu T} \arrow{r}{T\mu} & T^2 \arrow{d}{\mu} \\
      T^2 \arrow{r}{\mu} & T

    \end{tikzcd}

    \begin{tikzcd}
      T \ar[equal]{dr} \arrow{r}{T\eta} & T^2 \arrow{d}{\mu} & T \arrow{l}[swap]{\eta T} \ar[equal]{ld} \\
      & T &
    \end{tikzcd}
  \end{center}
\end{definition}

Estudiar algunos ejemplos será útil para aclarar qué parte hace qué
en la definición.

\subsubsection{Ejemplos}
\paragraph{Monoides}
Sea $M$ un monoide. Definimos el funtor $T : \arr{\Set}{\Set}$
como $T(X) = M\times X$, $\eta_X : \arr{X}{T(X)}$ definida
por $\eta_X(x) = (e, x)$ (con $e \in M$ el elemento neutro)
y $\mu_X : \arr{T^2(X)}{T(X)}$ definida por
$\mu_X(m, n, x) = (m\cdot n, x)$

Veamos que $(T, \eta, \mu)$ es un monoide. En primer lugar
tenemos que ver que para cualquier conjunto $X$ el siguiente
diagrama es conmutativo:

\begin{center}
  \begin{tikzcd}
    M\times M \times M \times X \arrow{r}{T\mu_X}
           \arrow{d}{\mu_{TX}} & M\times M\times X \arrow{d}{\mu_X} \\
    M\times M\times X \arrow{r}{\mu_X} & M \times X
  \end{tikzcd}
\end{center}
Es decir $\mu_X \circ \mu_{TX} = \mu_X \circ T\mu_X$. Para comprobar
esta igualdad de aplicaciones tomamos $(m, n, k, x) \in T^3(X)$.
Por un lado:
$$(\mu_X \circ \mu_{TX})(m, n, k, x) = \mu_X(\mu_{TX}(m, n, k, x)
  = \mu_X(m\cdot n, k, x) = ((m\cdot n)\cdot k, x)$$
Por otro lado:
$$(\mu_X \circ T\mu_X)(m, n, k, x) = \mu_X(T\mu_X(m, n, k, x))
  = \mu_X(m, n\cdot k, x) = (m \cdot (n\cdot k), x)$$
y por la asociatividad del producto en $M$ tenemos que
el diagrama es conmutativo.
Queda ver que para cualquier conjunto $X$ el diagrama
\begin{center}
  \begin{tikzcd}
    T(X) \ar[equal]{dr}
         \arrow{r}{\eta_{TX}} & T^2(X) \arrow{d}{\mu_X} & \arrow{l}[swap]{T\eta_X} \ar[equal]{ld} T(X) \\
    & T(X) &
  \end{tikzcd}
\end{center}
es conmutativo, es decir, $\mu_X \circ T\eta_X = \mu_X \circ \eta_{TX}$. Podemos
comprobar esta igualdad tomando $(m, x) \in T^2(X)$:
$$\mu_X(T\eta_X(m, x)) = \mu_X(m, e, x) = (m \cdot e, x) = (m, x)$$
$$\mu_X(\eta_{TX}(m, x)) = \mu_X(e, m, x) = (e\cdot m, x) = (m, x)$$

\paragraph{Sumas}
Sea $\C$ una categoría con coproductos finitos. Dado un objeto
$C$ de $\C$ definimos
un endofuntor $T : \arr{\C}{\C}$ por $T(X) = C + X$. La
transformación natural $\eta_X : \arr{X}{C + X}$ será
la inyección canónica y $\mu_X : \arr{C + C + X}{C+X}$ será
la aplicación $(1_C + 1_C) + 1_X$. Veamos que
$(T, \eta, \mu)$ es una mónada. En primer lugar
dado un objeto $X$ de $\C$ comprobamos la conmutatividad
del siguiente diagrama:
\begin{center}
  \begin{tikzcd}
    C + C + C + X \arrow{r}{T\mu_X}
           \arrow{d}{\mu_{TX}} & T^2(X) \arrow{d}{\mu_X} \\
    T^2(X) \arrow{r}{\mu_X} & TX
  \end{tikzcd}
\end{center}
Lo que equivale a comprobar que
$\mu_X \circ \mu_{TX} = \mu_X \circ T\mu_X$.

\textit{Demostrar esto va a ser muy laborioso. Quizá solo debería afirmar que esto es una mónada y luego simplemente explicar la versión en Haskell}.

\paragraph{$\Hom(A, -)$}
Sea $A$ un conjunto. Consideramos el endofuntor $\Hom(A, -) : \arr{\Set}{\Set}$.
$\eta_X : \arr{X}{\Hom(A, X)}$ definida por $\eta_X(x)(a) = x$ y
$\mu_X : \arr{\Hom(A, \Hom(A, X))}{\Hom(A, X)}$ dada por
$\mu_X(f)(a) = f(a)(a)$. Veamos que $(T, \eta, \mu)$ es una
mónada. El primer diagrama cuya conmutatividad debemos
verificar es
\begin{center}
  \begin{tikzcd}
    \Hom(A, \Hom(A, \Hom(A, X))) \arrow{r}{T\mu_X}
    \arrow{d}{\mu_{TX}} & \Hom(A, \Hom(A, X)) \arrow{d}{\mu_X} \\

    \Hom(A, \Hom(A, X)) \arrow{r}{\mu_X} & \Hom(A, X)
  \end{tikzcd}
\end{center}

Sea $f : \Hom(A, \Hom(A, \Hom(A, X)))$ entonces:
$$(\mu_X \circ T\mu_X)(f)(a) = (\mu_X( (T\mu_X)(f) )(a)
   = ((\mu_X \circ f))(a))(a)
   = \mu_X(f(a))(a)
   = f(a)(a)(a)$$
Por el otro lado:
\begin{center}
  \begin{tikzcd}
    \Hom(A, X)
    \arrow{r}{T\eta_X} &
    \Hom(A, \Hom(A, X))
    \arrow{d}{\mu} &
    \Hom(A, X) \arrow{l}{\eta_{TX}}
  \end{tikzcd}
\end{center}
