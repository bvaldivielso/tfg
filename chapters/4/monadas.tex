Comenzamos con el estudio de una de las nociones nacidas
en teoría de categorías que son
más importantes para nuestro trabajo: las mónadas.


\subsection{Definición}
\begin{definition}
  Una mónada sobre una categoría $\C$
  es una terna $(T, \eta, \mu)$ donde $T : \arr{\C}{\C}$ es un
  endofuntor y $\eta : \nat{1_\C}{T}, \, \mu: \nat{T^2}{T}$ son dos
  transformaciones naturales tales que los siguientes diagramas
  conmutan:
  \begin{center}
    \begin{equation}\label{diagrama:mu}
    \begin{tikzcd}
      T^3 \arrow{d}[swap]{\mu T} \arrow{r}{T\mu} & T^2 \arrow{d}{\mu} \\
      T^2 \arrow{r}{\mu} & T
    \end{tikzcd}
    \end{equation}

    \begin{equation}\label{diagrama:eta}
      \begin{tikzcd}
        T \ar[equal]{dr} \arrow{r}{T\eta} & T^2 \arrow{d}{\mu} & T \arrow{l}[swap]{\eta T} \ar[equal]{ld} \\
        & T &
      \end{tikzcd}
    \end{equation}
  \end{center}
\end{definition}
Estudiar algunos ejemplos será útil para comprender mejor
los distintos elementos que forman parte
de la definición.

\subsubsection{Ejemplos}
\paragraph{Monoides}
Sea $M$ un monoide. Definimos:
\begin{itemize}
\item El funtor $T : \arr{\Set}{\Set}$
  como $T(X) = M\times X$
\item $\eta_X : \arr{X}{T(X)}$ definida
por $\eta_X(x) = (e, x)$ (con $e \in M$ el elemento neutro)
\item $\mu_X : \arr{T^2(X)}{T(X)}$ definida por
$\mu_X(m, n, x) = (m\cdot n, x)$
\end{itemize}

Veamos que $(T, \eta, \mu)$ es una mónada.
\textit{A lo largo de este ejemplo nos tomaremos la libertad de identificar $T^2X$ con $M\times M\times X$ en lugar con $M \times (M\times X)$ con la intención de simplificar la notación}.
En primer lugar
tenemos que ver que para cualquier conjunto $X$ el siguiente
diagrama es conmutativo (diagrama \eqref{diagrama:mu} de la definición
de mónada):
\begin{center}
  \begin{tikzcd}
    M\times M \times M \times X \arrow{r}{T\mu_X}
           \arrow{d}{\mu_{TX}} & M\times M\times X \arrow{d}{\mu_X} \\
    M\times M\times X \arrow{r}{\mu_X} & M \times X
  \end{tikzcd}
\end{center}
Es decir $\mu_X \circ \mu_{TX} = \mu_X \circ T\mu_X$. Para comprobar
esta igualdad tomamos $(m, n, k, x) \in T^3(X)$.
Por un lado:
$$(\mu_X \circ \mu_{TX})(m, n, k, x) = \mu_X(\mu_{TX}(m, n, k, x)
  = \mu_X(m\cdot n, k, x) = ((m\cdot n)\cdot k, x)$$
Por otro lado:
$$(\mu_X \circ T\mu_X)(m, n, k, x) = \mu_X(T\mu_X(m, n, k, x))
  = \mu_X(m, n\cdot k, x) = (m \cdot (n\cdot k), x)$$
y por la asociatividad del producto en $M$ tenemos que
el diagrama es conmutativo.
Queda ver que para cualquier conjunto $X$ el diagrama
(\ref{diagrama:eta} de la definición de mónada)
\begin{center}
  \begin{tikzcd}
    T(X) \ar[equal]{dr}
         \arrow{r}{\eta_{TX}} & T^2(X) \arrow{d}{\mu_X} & \arrow{l}[swap]{T\eta_X} \ar[equal]{ld} T(X) \\
    & T(X) &
  \end{tikzcd}
\end{center}
es conmutativo, es decir, $\mu_X \circ T\eta_X = 1_{TX} = \mu_X \circ \eta_{TX}$. Podemos
comprobar esta igualdad basta ver que para cualquier $(m, x) \in TX$:
$$\mu_X(T\eta_X(m, x)) = \mu_X(m, e, x) = (m \cdot e, x) = (m, x)$$
$$\mu_X(\eta_{TX}(m, x)) = \mu_X(e, m, x) = (e\cdot m, x) = (m, x)$$

\paragraph{Sumas}
Sea $\C$ una categoría con coproductos finitos. Dado un objeto
$C$ de $\C$ de definimos:

\begin{itemize}
\item El endofuntor $T: \arr{\C}{\C}$ dado por $T(X) = C + X$
\item La transformación natural $\eta : \nat{1_\C}{T}$ dada por
  $\eta_X : \arr{X}{C+X}$ la inyección canónica en el coproducto.
\item La transformación natural $\mu: \nat{T^2}{T}$ dada por
  $\mu_X : \arr{C + (C + X)}{C + X}$ \textit{Definir}
\end{itemize}

La terna $(T, \eta, \mu)$ es una mónada.


\paragraph{Mónada $\Hom(A, -)$}
Sea $A$ un conjunto. Definimos:

\begin{itemize}
\item El endofuntor $T = \Hom(A, -) : \arr{\Set}{\Set}$
\item La transformación natural $\eta : \nat{1_\Set}{T}$ dada
  por $\eta_X : \arr{X}{\Hom(A, X)}$ con $\eta_X(x)(a) = x$.
\item La transformación natural $\mu: \nat{T^2}{T}$ dada por
  $\mu_X : \arr{\Hom(A, \Hom(A, X))}{\Hom(A, X)}$ con
  $\mu_X(f)(a) = f(a)(a)$
\end{itemize}

La terna $(T, \eta, \mu)$ es una mónada.

\subsection{Las adjunciones dan lugar a mónadas}
\begin{theorem}
Sean $\C$ y $\D$ dos categorías y $F : \arr{\C}{\D}$ y $G: \arr{\D}{\C}$ un
par de funtores adjuntos $F \adjoint G$ con unidad
$\eta : \nat{1_{\C}}{GF}$ y counidad $\epsilon: \nat{FG}{1_\D}$. La terna
$(GF, \eta, G\epsilon F)$ es una mónada.
\end{theorem}
\begin{proof}
  \textcolor{red}{Hacer demostración}
\end{proof}

Pero este resultado lo sabemos. El siguiente resultado.

\subsection{Categoría de Kleisli de una mónada}
Hemos visto que todo par de funtores adjuntos inducen una mónada.
Es natural plantearse si el recíproco es cierto: ¿es toda mónada
la composición de un par de funtores adjuntos?
La respuesta es afirmativa.

\begin{theorem}
  Sea $\C$ una categoría y $(T, \eta, \mu)$ una mónada
  sobre $\C$. Entonces existe una categoría $\D$ y un par
  de funtores $F : \arr{\C}{\D}$, $G : \arr{\D}{\C}$ tales
  que $F \adjoint G$ y además:
  \begin{enumerate}
  \item $T = GF$
  \item $\eta$ es la unidad de la adjunción
  \item $\mu = F\epsilon G$ donde $\epsilon: \nat{FG}{1_\D}$ es
    la counidad de la adjunción.
  \end{enumerate}
\end{theorem}
\begin{proof}[\textbf{Demostración. } Categoría de Kleisli]
  Probamos el resultado paso por paso.
  \begin{enumerate}
  \item Construcción de $\D$. Definimos $\Obj{\D} = \Obj{\C}$ y
    $\Hom_\D(A, B) = \Hom_\C(A, TB)$.
    Dadas dos flechas $f \in \Hom_\D(A, B)$ y
    $g \in \Hom_\D(B, C)$ definimos la composición $g \circ_\D f \in \Hom_\D(A, C)$
    como la flecha
    $$g \circ_\D f :
    A \xrightarrow{f} TB \xrightarrow{Tg} T^2 C \xrightarrow{\mu_C} T C
    $$
    Veamos que esta operación de composición
    cumple los axiomas:
    \begin{itemize}
    \item Es asociativa. Sean $f : \Hom_\D(A, B)$, $g : \Hom_\D(B, C)$
      y $h : \Hom_\D(C, D)$ entonces
      $$(h \circ_{\D} g) \circ_\D f = \mu_D \circ T(h \circ_\D g) \circ f
      = \mu_D \circ T(\mu_D \circ Th \circ g) \circ f
      = \mu_D \circ T\mu_D \circ T^2h \circ Tg \circ f$$
      $$=^* \mu_D \circ \mu_{TD} \circ T^2h \circ Tg \circ f
      =^{**}\mu_D \circ Th \circ \mu_C \circ Tg \circ f
      =\mu_D \circ Th \circ (g \circ_\D f)
      = h \circ_\D (g \circ_\D f)$$

      Donde ($*$) es consecuencia del diagrama \eqref{diagrama:mu}
      de la definición de mónada aplicado sobre $D$ y ($**$) es
      consecuencia de la naturalidad de $\mu : \nat{T^2}{T}$.

    \item Existen identidades. Sean
      $f \in \Hom_\D(A, X)$ y $g \in \Hom_\D(X, A)$. Observamos
      que
      $$f \circ_\D \eta_A = \mu_X \circ Tf \circ \eta_A
      =^* \mu_X \circ \eta_{TB} \circ f
      =^{**} f$$
    Donde $(*)$ es consecuencia de la naturalidad de
    $\eta : \nat{1_\C}{T}$ y $(**)$ es consecuencia
    del diagrama \eqref{diagrama:eta} de la definición de mónada.
    Por otro lado
    $$\eta_A \circ_\D g = \mu_A \circ T\eta_A \circ g
    =^* g$$
    con $(*)$ consecuencia otra vez del diagrama \eqref{diagrama:eta}.

    Esto muestra que en la categoría $\D$ la flecha $\eta_A$ es la
    identidad del objeto $A$ para cualquier objeto $A$ de la categoría.
    \end{itemize}
  \item Ahora que conocemos la estructura de categoría
    de $\D$ definimos $F : \arr{\C}{\D}$ como $FA = A$
    para cada objeto $A$ de $\C$ y $Ff = Tf \circ \eta_A$ para
    toda flecha $f: \arr{A}{B}$ de $\C$.
    Veamos que $F$ es un funtor. Por partes:
    \begin{itemize}
    \item $F$ se lleva bien con la composición. Dado el diagrama
      $A \xrightarrow{f} B \xrightarrow{C}$ en la categoría $\C$
      tenemos que:
      $$F(g \circ f) = T(g \circ f) \circ \eta_A
      = Tg \circ Tf \circ \eta_A
      =^* Tg \circ \eta_B \circ f$$
      Donde $(*)$ se deduce de la naturalidad de
      $\eta : \nat{1_\C}{T}$. Igualmente:
      $$Fg \circ_\D Ff = (Tg \circ \eta_B) \circ_\D (Tf \circ \eta_A)
      = \mu_C \circ T(Tg \circ \eta_B) \circ (Tf \circ \eta_A)$$
      $$= \mu_C \circ T^2g \circ T\eta_B \circ Tf \circ \eta_A
      =^* \mu_C \circ T^2g \circ T\eta_B \circ \eta_B \circ f$$
      $$=^{**} Tg \circ \mu_B \circ T\eta_B \circ \eta_B \circ f
      =^{***} Tg \circ \eta_B \circ f$$
      Con $(*)$ consecuencia de la naturalidad de
      $\eta: \nat{1_\C}{T}$, $(**)$ consecuencia de la naturalidad
      de $\mu : \nat{T^2}{T}$, y $(***)$ por el diagrama
      \eqref{diagrama:eta} de la definición de mónadas.


    \item $F$ lleva identidades a identidades:
      $$F1_A = T1_A \circ \eta_A = 1_{TA} \circ \eta_A = \eta_A$$
      que es la flecha identidad de $A$ en $\D$.

    \end{itemize}

  \item Definimos $GA = TA$ para cada objeto $A$ de $\D$
    y $Gf = \mu_B \circ Tf$ para cada flecha $f \in \Hom_\D(A, B)$.
    Mostramos que $G: \arr{\D}{\C}$ es un funtor:
    \begin{itemize}
    \item $G$ se lleva bien con la composición. Dadas $f \in \Hom_\D(A, B)$
      y $g \in \Hom_\D(B, C)$ tenemos que:
      $$G(g \circ_\D f) = \mu_C \circ T(g \circ_\D f)
      = \mu_C \circ T(\mu_C \circ Tg \circ f)
      = \mu_C \circ T\mu_C \circ T^2g \circ Tf$$
      $$=^* \mu_C \circ \mu_{TC} \circ T^2g \circ Tf
      =^{**} \mu_C \circ Tg \circ \mu_B \circ Tf
      = Gg \circ Gf$$
      Con $(*)$ consecuencia del diagrama \eqref{diagrama:mu}
      de la definición de mónada y $(**)$ consecuencia de la
      naturalidad de $\mu : \nat{T^2}{T}$.
    \item $G$ se lleva bien con las identidades.
      $G\eta_A = \mu_A \circ T\eta_A = 1_{TA}$
      por el diagrama \eqref{diagrama:eta} de la definición de mónada.
    \end{itemize}
  \item La prueba de que $F \adjoint G$ es sencilla.
    El isomorfismo natural que necesitamos para establecer la
    adjunción es precisamente:
    $$\phi : \Hom_\D(F-, -) \cong \Hom_\C(-, G-)$$
    $$\phi_{A, B}(f) = f$$
    La comprobación de la naturalidad de $\phi$ es rutinaria.

  \end{enumerate}
\end{proof}
