Dedicamos este capítulo al tratamiento de construcciones elementales
sobre categorías. En este capítulo podremos ver cómo un marco tan general
como el de la teoría de categorías permite realizar definiciones
\textit{Continuar introducción de construcciones elementales luego}.

\section{Elementos}
Cuando hablamos de conjuntos es común hablar de sus elementos. Muchas
definiciones que se hacen sobre conjuntos y aplicaciones entre
ellos se hacen en base a los elementos de los conjuntos involucrados.
Dos ejemplos de este tipo de definiciones serían las definiciones de
aplicación inyectiva y de producto cartesiano.

\begin{definition*}
Una aplicación $f : \arr{A}{B}$ es inyectiva si dados
$a, a' \in A$ tenemos que $f(a) = f(a') \implies a = a'$.
\end{definition*}

\begin{definition*}
Dados dos conjuntos $A$ y $B$ definimos su producto cartesiano
como:
$$A\times B = \{ (a, b) : a \in A \quad b \in B \}$$
\end{definition*}

Si intentamos trasladar las construcciones que hacemos sobre los
conjuntos y sus aplicaciones a categorías arbitrarias con sus objetos
y sus flechas tenemos que saber cómo trasladar la noción de elemento.

Si consideramos el conjunto con un solo elemento, al que llamaremos
$*$, y las aplicaciones que salen de él nos damos cuentas de que podemos
identificar las aplicaciones entre el conjunto $* = \{ 1 \}$ y un conjunto
$A$ (estas aplicaciones son de la forma $1 \mapsto a \in A$)
arbitrario con los elementos (en el sentido de teoría de conjuntos)
de $A$. Dicho de otra forma $\Hom(*, A)$ son lo mismo $A$ como conjuntos.
pero $\Hom(*, A)$ está formado por flechas y eso es algo de lo que
sí podemos hablar dentro de una categoría. Sin embargo, para realizar
este procedimiento de identificar los elementos de un conjunto con
un conjunto de flechas de la categoría hemos tenido que acudir a un
objeto especial de la categoría de conjuntos: $*$. Este procedimiento
es algo que quizá no podamos realizar en otras categorías pero nos
motiva a hacer una definición más general de
\textit{elemento categórico}:

\begin{definition}
En una categoría $\C$ llamaremos elemento de un objeto $A$
a cualquier flecha $x : \arr{T}{A}$ (sea cual sea el objeto
$T$). De forma paralela a como se hace con conjuntos, utilizaremos
la notación $x \in^T A$.
\end{definition}

Con esta noción de elemento notamos que $f : \arr{A}{B}$ lleva
elementos de $A$ a elementos de $B$ mediante la composición: si
$x \in^T A$ entonces $f \circ x \in^T B$. Esto motiva que en lo que
sigue omitamos a veces el signo de composición ($f x$ en lugar
de $f \circ x$) si buscamos que se interpreten algunas flechas
como elementos categóricos y la acción de otras flechas sobre estos.

Veamos como esta noción nos permite llevar a teoría de categorías
algunos conceptos originados sobre conjuntos.

\section{Monomorfismos}
\begin{definition}
Consideremos una categoría $\C$ y una flecha $f : \arr{A}{B}$ de esta.
Diremos que $f$ es un monomorfismo si dados dos elementos de $A$
$x, y \in^T A$ tenemos que $fx = fy \implies x = y$.
\end{definition}

Nótese cómo esta definición es casi idéntica a la definición de
inyectividad sobre aplicaciones entre conjuntos. De hecho es sencillo
demostrar que la noción de monomorfismo sobre $\Set$ se corresponde
efectivamente con las aplicaciones inyectivas.


\paragraph{Ejemplos}
Esta noción se puede aplicar sobre otras categorías. En general
cuando consideramos categorías en la que los objetos son estructuras
matemáticas y las flechas son sus morfismos, los monomorfismos son
aquellas flechas tales que las aplicaciones entre conjuntos subyacentes
son inyectivas. Ejemplos de esto son las categorías
$\Grp$, $\Ring$...

\section{Isomorfismos}
\begin{definition}
Consideremos una categoría $\C$ y una flecha $f : \arr{A}{B}$ de
esta. Diremos que $f$ es un isomorfismo si $f$ es una biyección
entre los elementos de $A$ y los elementos de $B$ definido sobre
$T$ para cualquier objeto $T$ de $\C$.
\end{definition}

Esta definición es similar a la definición habitual
de biyección sobre conjuntos pero generalizando los elementos de
los conjuntos sobre elementos de los objeto sobre el resto los elementos
de la categoría.

Nótese que de este concepto podemos dar una definición equivalente
que no requiere del uso de elementos categóricos.

\begin{proposition}
Dada la categoría $\C$ y y una flecha $f: \arr{A}{B}$ tenemos que
$f$ es un isomorfismo sí y solo sí existe una flecha $g: \arr{B}{A}$
tal que $f \circ g = 1_B$ y $g \circ f = 1_A$.
\end{proposition}


Esta caracterización de isomorfismo nos permite ver rápidamente
que los isomorfismos de la categoría $\Set$ se corresponden con las
biyecciones y que los isomorfismos sobre $\Grp$, $\Ring$, $\Top$...
se corresponden con los isomorfismos de grupos,
isomorfismos de anillos y homeomorfismos de espacios topológicos.


\section{Productos}
Otra de los ejemplos que dimos anteriormente de construcción que
se realiza sobre conjuntos era el producto cartesiano. Esta definición
podemos llevarla a categorías arbitrarias de la siguiente forma:

\begin{definition}
Sea $\C$ una categoría y $A$ y $B$ dos objetos de esta. Diremos que
existe el producto de $A$ y $B$ si
existe una terna $(A\times B, \pi_1, \pi_2)$
donde $A\times B$ es un objeto de $\C$ y
$\pi_1 : \arr{A\times B}{A}, \pi_2 : \arr{A\times B}{B}$ son dos flechas tales
que para cualquiera elementos $x : \arr{T}{A}$ de $A$ e
$y: \arr{T}{B}$ de $B$, existe un único elemento
$(x,y): \arr{T}{A\times B}$ de $A\times B$ tal que $\pi_1 (x,y)=x$ y $\pi_2(x,y)=y$.

Expresaremos esto diciendo que  el siguiente diagrama es conmutativo:
\begin{center}
\begin{tikzcd}
&\arrow{ld}[swap]{x} T \arrow[d, dotted, "{\exists! (x, y)}"] \arrow[rd, "y"] & \\
A & \arrow[l, "\pi_1"] A\times B  \arrow{r}[swap]{\pi_2} & B
\end{tikzcd}
\end{center}

\end{definition}

Debemos resaltar dos aspectos importantes de esta definición:
\begin{itemize}
\item El producto de dos objetos en una categoría dada no tiene por qué
      existir.
\item De existir un producto, este solo queda determinado salvo isomorfismo. De hecho si $(A\times B, \pi_1, \pi_2)$ es un producto de $A$ y $B$
y $\phi : \arr{P}{A\times B}$ es un isomorfismo entonces
$(P, \pi_1 \circ \phi , \pi_2 \circ \phi)$ es otro producto de $A$
y $B$.
\end{itemize}


\subsection{Ejemplos}
\paragraph{\Set}
Dados dos conjuntos $A$ y $B$ siempre existe el producto categórico
y además coincide (salvo el isomorfismo
que comentamos antes) con el producto cartesiano de ambos conjuntos
y las proyecciones canónicas. Lo demostramos a continuación:

Sea $a : \arr{T}{A}, b : \arr{T}{B}$ un par de funciones. Podemos
definir la función $(a, b) : \arr{X}{A\times B}$ tal que
$f(x) = (a(x), b(x))$. Esta función cumple en efecto que
$\pi_1 \circ f = a$ y $\pi_2 \circ f = b$. Pero además es la única
que cumple esto pues
$f(x)=(\pi_1\circ f(x), \pi_2\circ f(x))=(f_1(x), f_2(x))$.

\paragraph{Otras estructuras matemáticas}
A continuación mostramos ejemplos de productos categóricos
en categorías conocidas:

\begin{itemize}
\item En $\Grp$ el producto categórico se corresponde con el
producto directo de grupos.
\item En $\Top$ se corresponde con el producto de espacios topológicos.
\item En $\Ring$ se corresponde con el producto de anillos.
\end{itemize}

\paragraph{Categoría de cuerpos}
En la categoría de cuerpos no existen los productos.
\textit{Demostrar}

\section{Objetos iniciales}
Al principio de la sección le dimos un papel especial al conjunto
de un solo elemento $*$. Nos gustaría caracterizar a ese objeto de
la categoría $\Set$ con alguna propiedad estrictamente categórica.
Esto es posible:

\begin{definition}
Sea $\C$ una categoría y $I$ un objeto. Diremos que $I$ es un objeto
inicial si dado cualquier objeto $T$ de la categoría solo existe un
elemento de $I$ definido sobre $T$.
\end{definition}

Esto es cierto sobre $* = \{ 1 \}$: la única flecha que llega a
a $*$ desde cualquier conjunto es la que vale constantemente 1, pero
una vez más esta definición se puede aplicar a otras categorías.

\subsection{Ejemplos}
\paragraph{En $\Grp$}
El objeto inicial en $\Grp$ es el grupo trivial. Solo existe una función
que vaya del grupo trivial a cualquier otro grupo y es la aplicación
nula. Resulta, además, que el objeto final de la categoría es también
el grupo trivial. Si en una categoría tenemos que un objeto
$O$ es a la vez inicial y final diremos que $O$ es un objeto nulo. Esto
es importante porque nos permite definir morfismos nulos.
% TODO: definir

\paragraph{En $\Ring$}
En la categoría de los anillos con unidad el objeto inicial es
$\Z$ y el único morfismo que existe de él en cualquier otro anillo
$R$ es el que lleva $1_\Z$ a $1_R$. No existe objeto final en la
categoría de anillos (se razona considerando la característica
del supuesto anillo final).

\section{Dualidad}
El concepto de dualidad que se encuentra frecuentemente en las
matemáticas puede ser analizado de forma muy general desde el punto
de vista categórico. Para introducir esta noción presentamos a
continuación la definición de categoría opuesta.

\subsection{La categoría opuesta}
Dada una categoría $\C$ podemos construir una categoría
$\C^{op}$ en la que definimos $\Obj{\C^{op}} = \Obj{\C}$,
$\Hom_{\C^{op}}(A, B) = \Hom_{\C}(B, A)$ y una operación de
composición dad por $f \circ_{op} g = g \circ f$. Veamos
la construcción realizada describe en efecto una categoría:
\begin{itemize}
\item La composición es asociativa: sean
      $f \in \Hom_{\C^{op}}(A, B)$, $g \in \Hom_{\C^{op}}(B, C)$
      y $h \in \Hom_{\C^{op}}(C, D)$. Tenemos que

\begin{multline*}
(h \circ_{op} g) \circ_{op} f = (g \circ h) \circ_{op} f \\
                              = f \circ (g \circ h)
                              = (f \circ g) \circ h
                              = (g \circ_{op} f)  \circ h
                              = h \circ_{op} (g \circ_{op} f)
\end{multline*}
\item Existen las identidades. Las identidades siguen siendo
      las mismas flechas que son identidad en $\C$. La demostración
      es inmediata.
\end{itemize}

La categoría opuesta nos otorga una herramienta para reaprovechar
definiciones realizadas anteriormente. Veremos ejemplos de propiedades
duales en las siguientes secciones.

\subsection{Epimorfismos}
En una sección anterior especificamos cuando una flecha
$f : \arr{A}{B}$ de una categoría $\C$ era un monomorfismo. La categoría
opuesta nos permite dualizar esta definición de la siguiente manera:

\begin{definition}
Dada una categoría $\C$ y una flecha $f: \arr{A}{B}$, diremos que
$f$ es un epimorfismo si y solo si $f$ es un monomorfismo en
$\C^{op}$.
\end{definition}

En este sentido decimos que las propiedades ``$f$ es un monomorfismo''
y ``$f$ es un epimorfismo'' son propiedades duales. Esta propiedad
se puede enunciar de forma sencilla también sin recurrir a
$\C^{op}$.

\begin{proposition}
$f$ es un epimorfismo si y solo si dado cualquier objeto $X$ y
cualquier par de flechas $g_1, g_2 : \arr{B, X}$
tenemos que $g_1 \circ f = g_1 \circ f \implies g_1 = g_2$.
Decimos también en este caso que $f$ se puede cancelar por la
derecha.
\end{proposition}

\subsubsection{Ejemplos}
\paragraph{En \Set}
En la categoría de los conjuntos los epimorfismos coinciden con las
aplicaciones sobreyectivas.

\paragraph{Isomorfismos}
Todo isomorfismo es un epimorfismo y un monomorfismo. El recíproco
es cierto en algunas categorías (como en $\Set$ o $\Grp$) pero no
lo es en general.

\paragraph{En \Ring}
Consideremos la categoría de anillos con unidad (en la
que los objetos son anillos con unidad y
las flechas son homomorfismos de anillos). En esta categoría
que $f : \arr{R}{S}$ es equivalente también a que $f$ sea una aplicación
inyectiva.

Sin embargo consideremos la inclusión
$i : \arr{\Z}{\Q}$, un anillo $R$
y un par de aplicaciones de anillos $g_1, g_2 : \arr{\Q}{R}$. Supongamos
que $g_1 \circ f = g_2 \circ f$. Ahora  $\forall x \in \Q$
tenemos que $\exists a, b \in \Z : x = \frac{a}{b}$ y entonces:

\begin{multline*}
g_1(x) = g_1(\frac{a}{b}) = g_1(a)g_1(b)^{-1} \\
       = (g_1 \circ i)(a)(g_1\circ i)(b)^{-1}
       = (g_2\circ i)(a)(g_2\circ i)(b)^{-1}
       = g_2(a)g_2(b)^{-1} = g_2(x)
\end{multline*}

Con lo que $g_1=g_2$. Esto prueba que $i : \arr{\Z}{\Q}$ es epimorfismo
y sin embargo no es sobreyectiva como aplicación. Un isomorfismo en
la categoría de anillos se corresponde con un isomorfismo de anillos
(que en particular es una biyección de conjuntos) y por tanto
en la categoría de anillos se pueden encontrar flechas que son
monomorfismos y epimorfismos pero no son isomorfismos.

\subsection{Objetos finales}
Existe también una propiedad dual a la de ser objeto inicial.

\begin{definition}
Dado un objeto $T$ de una categoría $C$ decimos que
$T$ es un objeto final si $T$ es un objeto inicial en la
categoría $C^{op}$.
\end{definition}

Podemos probar una caracterización que no hace referencia a
$\C^{op}$:

\begin{proposition}
El objeto $T$ es final si y solo si para cualquier otro objeto
$X$ existe una sola flecha tiene a $X$ como dominio y a $T$ como
codominio.
\end{proposition}

\subsubsection{Ejemplos}
\paragraph{En \Set}
En $\Set$ el conjunto vacío $\emptyset$ es un objeto final.

\paragraph{En \Grp}

El objeto final en $\Grp$ es el grupo de un solo elemento. Notemos
que coincide con el objeto inicial. Cuando en una categoría $\C$ coinciden
el objeto inicial y el final $T$ llamamos a $T$ objeto nulo. El objeto
nulo nos permite definir una aplicación nula (\textit{explicar esto mejor}).

Otro ejemplo de categoría en la que ocurre esto es en la de espacios vectoriales
sobre un cuerpo $K$.

\paragraph{En \Ring}
\textit{Probar que no existen objetos finales en la categoría de anillos}

\subsection{Coproducto}
Podemos repetir la definición de producto categórico sobre la categoría
opuesta y \textit{dando la vuelta} a las flechas para volver a la
categoría inicial $\C$ llegaríamos a la siguiente definición de lo que
llamaremos \textbf{coproducto}:

\begin{definition}
Sea $\C$ una categoría y $A$ y $B$ dos objetos de esta. Diremos que
existe el coproducto de $A$ y $B$
si existe una terna $(A+B, i_1, i_2)$
donde $P$ es un objeto de $\C$ y
$i_1 : \arr{P}{A}, i_2 : \arr{P}{B}$ son dos flechas tales
que para cualquier objeto $Y$ y sendas flechas $f_1 : \arr{A}{Y}$,
$f_2 : \arr{B}{Y}$ existe un morfismo
$f : \arr{P}{Y}$ tal que el siguiente diagrama es conmutativo:
\begin{center}
\begin{tikzcd}
&  Y  \arrow[rd, "f_2"]& \\
A \arrow[ur, "f_1"] \arrow{r}[swap]{i_1} &
 \arrow[u, dotted, "\exists! f"] P
 & B \arrow[l, "i_2"]
\end{tikzcd}
\end{center}
\end{definition}

\textit{Retocar esta sección para que sea más consistente con las otras}
\textit{Hablar de que es único salvo isomorfismo}.


\subsection{Funtores $\Hom(-, A)$}
Sea $\C$ una categoría y $A$ un objeto de esta.
Sea $f \in \Hom_{\C^{op}}(C, C')$. Definimos:

$$\Hom_{\C}(f, A) :
  \arr{
    \Hom_{\C}(C, A)
  }{
    \Hom_{\C}(C', A)
  }$$
$$\Hom_{\C}(f, A)(g) = g \circ f$$

(recordando que $g : \arr{C}{A}$ y $f : \arr{C'}{C}$ como
flechas de $\C$ y por tanto se pueden componer)

Se cumple además que:

\begin{itemize}
\item $\Hom_{\C}(1_C, A)(g) = g \circ 1_C = g$ para toda flecha
  $g\in \Hom_{\C}(C, A)$ y por tanto
  $\Hom_{\C}(1_C, A) = 1_{\Hom_{\C}(C, A)}$ para todo objeto $C$
  de $\C^{op}$.

\item Sean $f \in \Hom_{\C^{op}}(C, D)$ y $g : \Hom_{\C^{op}}(D, E)$
  entonces
  \begin{multline*}
    $$\Hom_{\C}(g \circ_{op} f, A)(h) = h \circ (g \circ_{op} f) \\
      h \circ (f \circ g) = (h \circ f) \circ g = \Hom_{\C}(g, A)(h \circ f) \\
      = \Hom_{\C}(g, A)(\Hom_{\C}(f, A)(h))$$
  \end{multline*}
  para toda $h \in \Hom_{\C}(C, A)$ y por tanto
  $\Hom_{\C}(g \circ_{op} f, A) = \Hom_{\C}(g, A) \circ \Hom_{\C}(f, A)$.
\end{itemize}

Esto nos muestra que $\Hom_\C(-, A)$ es un funtor
$\Hom_\C(-, A) : \arr{\C^{op}}{\Set}$, lo que no resulta
nada sorprendente teniendo en cuenta que
$\Hom_\C(-, A) = \Hom_{\C^{op}}(A, -)$ (que ya vimos que
era un funtor) tanto sobre los objetos
como sobre las flechas. Más aun, se puede demostrar
que $\Hom_\C(-, -) : \arr{\C^{op} \times \C}{\Set}$ es un bifuntor.

Cuando queremos hacer énfasis a que un funtor está actuando sobre
la categoría opuesta a una particular, decimos que es un funtor
contravariante. Por ejemplo, diríamos que $\Hom(-, A)$ es un funtor
contravariante.

\section{En programación}
A lo largo de esta sección aplicamos las construcciones que hemos visto
a la categoría \texttt{Hask}.

\subsection{Objetos iniciales y finales}
\texttt{Hask} tiene tanto objetos iniciales como objetos finales. El
objeto final es el tipo con un solo valor, al que se le suele llamar
\texttt{()} (pronunciado \texttt{Unit}). \texttt{()} es un tipo
cuyo único valor es \texttt{()} y para cada tipo \texttt{a} podemos
definir la función:
\begin{verbatim}
constUnit :: a -> ()
constUnit x = ()
\end{verbatim}
Por otro lado tenemos que el objeto inicial de \texttt{Hask} es el tipo
llamado \texttt{Void}, que no tiene ningún valor. En la biblioteca
estándar se encuentra una función llamada \texttt{absurd}
que tiene
como tipo \texttt{Void -> a} y es la única función con este tipo.
En cualquier caso, la función no puede ser llamada puesto que no
hay ningún valor con el que llamarla. La situación tanto como para
el objeto inicial como para el final es muy similar a la que se daba
en \Set.

\subsection{Productos}
\texttt{Hask} tiene productos. Dado dos tipos \texttt{A} y
\texttt{B} podemos construir el tipo \texttt{(A, B)} que tiene
como valores pares de valores de \texttt{A} y de \texttt{B}.
Las proyecciones se implementan en la biblioteca estándar de
Haskell de la siguiente forma:

\begin{verbatim}
fst :: (a, b) -> a
fst (x, y) = x

snd :: (a, b) -> b
fst (x, y) = y
\end{verbatim}

Si tenemos un par de funciones \cod{f1 :: X -> A} y
\cod{f2 :: X -> B} podemos definir la función \texttt{f}:
\begin{verbatim}
f :: X -> (A, B)
f x = (f1 x, f2 x)
\end{verbatim}

\subsection{Coproductos}
En la sección sobre funtores introdujimos el tipo \cod{Either}:
\begin{verbatim}
data Either a b = Left a | Right b
\end{verbatim}
Resulta que \cod{Either} es el coproducto en \cod{Hask}. Si
\cod{A} y \cod{B} son dos tipos de haskell entonces su coproducto
es \cod{Either A B} y las inyecciones canónicas son por un lado
\cod{Left :: A -> Either A B} y por otro lado
\cod{Right :: B -> Either A B}. Si tenemos un par de funciones
\cod{f1 :: A -> Y} y \cod{f2 :: B -> Y} tenemos que la única
función \cod{f :: Either A B -> Y} que se lleva bien con las inyecciones
es:
\begin{verbatim}
f :: Either A B -> Y
f (Left a) = f1 a
f (Right b) = f2 b
\end{verbatim}

Veamos que se lleva bien con las inyecciones:
\begin{verbatim}
(f . Left) a = f (Left a) = f1 a
;; por tanto f . Left = f1

(f . Right) b = f (Right b) = f2 b
;; por tanto f . Right = f2
\end{verbatim}
