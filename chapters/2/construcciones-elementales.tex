Dedicamos este capítulo al tratamiento de
algunas construcciones y definiciones elementales
sobre categorías. Veremos algunos ejemplos en los que
definiciones puramente categóricas nos permiten capturar nociones
específicas de estructuras matemáticas como los grupos o los
espacios topológicos. Además, introduciremos la categoría opuesta
como herramienta para razonar sobre la dualidad.

\section{Elementos}
Cuando hablamos de conjuntos es común hablar de sus elementos. Muchas
definiciones que se hacen sobre conjuntos y aplicaciones entre
ellos se hacen en base a los elementos de los conjuntos involucrados.
Dos ejemplos de este tipo de definiciones serían las definiciones de
aplicación inyectiva y de producto cartesiano.

\begin{definition*}
Una aplicación $f : \arr{A}{B}$ es inyectiva si dados
$a, a' \in A$, tenemos que $f(a) = f(a')$ implica $a = a'$.
\end{definition*}

\begin{definition*}
Dados dos conjuntos $A$ y $B$ definimos su producto cartesiano
como:
$$A\times B = \{ (a, b) : a \in A, \quad b \in B \}$$
\end{definition*}

Si intentamos trasladar las construcciones que hacemos sobre
conjuntos y las aplicaciones entre ellos
a construcciones sobre categorías arbitrarias,
será útil saber cómo trasladar la noción de elemento.

Si consideramos el conjunto con un solo elemento, al que llamaremos
$\uno = \{ x \}$, y las aplicaciones que
salen de él nos daremos cuenta de que podemos
identificar las aplicaciones $\arr{\uno}{A}$
(estas aplicaciones son de la forma $x \mapsto a \in A$)
con los elementos
de $A$. Dicho de otra forma $\Hom(*, A)$ y $A$ son
\emph{``lo mismo''} como conjuntos,
pero $\Hom(*, A)$ está formado por flechas y eso es algo de lo que
sí podemos hablar en el contexto de una categoría
arbitraria. Sin embargo, para realizar
este procedimiento de identificar los elementos de un conjunto con
un conjunto de flechas de la categoría hemos tenido que acudir a un
objeto especial de la categoría de conjuntos: $\uno$. Este procedimiento
es algo que quizá no podamos realizar en otras categorías pero nos
motiva a hacer una definición más general de
\textit{elemento categórico}.

\begin{definition}
En una categoría $\C$ llamaremos elemento de un objeto $A$
a cualquier flecha $x : \arr{T}{A}$ (sea cual sea el objeto
$T$). De forma paralela a como se hace con conjuntos, utilizaremos
la notación $x \in^T A$. Diremos también que $x$ es un elemento
de $A$ definido sobre $T$.
\end{definition}

Con esta noción de elemento notamos que $f : \arr{A}{B}$ lleva
elementos de $A$ a elementos de $B$ mediante la composición: si
$x \in^T A$ entonces $f \circ x \in^T B$. Esto motiva que en lo que
sigue omitamos a veces el signo de composición ($f x$
o $f(x)$ en lugar
de $f \circ x$) si nuestra intención es
que algunas flechas (en este caso $x$) se interpreten
como elementos categóricos.

Veremos como esta noción nos permite llevar a teoría de categorías
algunos conceptos originados sobre conjuntos.

\section{Monomorfismos}
\begin{definition}
Consideremos una categoría $\C$ y una flecha $f : \arr{A}{B}$ de ésta.
Diremos que $f$ es un monomorfismo si para cualquier objeto
$T$ de $\C$ y dados dos elementos
$x, y \in^T A$, tenemos que $fx = fy$ implica $x = y$.
\end{definition}

Nótese cómo esta definición es casi idéntica a la definición de
inyectividad sobre aplicaciones entre conjuntos. De hecho es sencillo
demostrar que la noción de monomorfismo sobre $\Set$ se corresponde
efectivamente con las aplicaciones inyectivas.


\paragraph{Ejemplos}
Esta noción se puede aplicar sobre otras categorías. En general
cuando consideramos categorías en la que los objetos son estructuras
matemáticas y las flechas son sus morfismos, los monomorfismos son
aquellas flechas tales que las aplicaciones entre conjuntos subyacentes
son inyectivas. Ejemplos de esto son las categorías
$\Grp$, $\Ring$, ...

\section{Isomorfismos}
\begin{definition}
Dada la categoría $\C$ y y una flecha $f: \arr{A}{B}$ diremos que
$f$ es un isomorfismo si existe una flecha $g: \arr{B}{A}$
tal que $f \circ g = 1_B$ y $g \circ f = 1_A$.
\end{definition}
Nótese que esta definición la podemos caracterizar en términos
de biyecciones sobre conjuntos de elementos.
\begin{proposition}
Sean $\C$ una categoría y $f: \arr{A}{B}$ una flecha de ésta.
Entonces $f$ es un isomorfismo si y solo si $f$ es una biyección entre
los elementos de $A$ y los elementos de $B$ definidos sobre $T$
para cualquier objeto $T$ de $\C$.
\end{proposition}
Esta caracterización es similar a la definición habitual
de biyección sobre conjuntos pero generalizando los elementos de
los conjuntos a elementos de los objetos definidos sobre todos
los objetos
de la categoría.

Es inmediato ver
que los isomorfismos de la categoría $\Set$ se corresponden con las
biyecciones y que los isomorfismos sobre $\Grp$, $\Ring$, $\Top$, ...
se corresponden con los isomorfismos de grupos,
isomorfismos de anillos y homeomorfismos de espacios topológicos,
respectivamente.

\section{Productos}
También es posible trasladar la definición de producto de conjuntos
a una categoría arbitraria mediante la siguiente definición.

\begin{definition}
Sea $\C$ una categoría y $A$ y $B$ dos objetos de ésta. Diremos que
existe el producto de $A$ y $B$ si
existe una terna $(A\times B, \pi_1, \pi_2)$
donde $A\times B$ es un objeto de $\C$ y
$\pi_1 : \arr{A\times B}{A}, \pi_2 : \arr{A\times B}{B}$ son dos flechas tales
que para cualesquiera elementos $x : \arr{T}{A}$ de $A$ e
$y: \arr{T}{B}$ de $B$, existe un único elemento
$(x,y): \arr{T}{A\times B}$ de $A\times B$ tal que $\pi_1 ((x,y))=x$ y $\pi_2((x,y))=y$ (de ahora en adelante notaremos $\pi_i(x, y)$ por
simplificar la notación).

Expresaremos esto diciendo que  el siguiente diagrama es conmutativo
para cada $T$:
\begin{center}
\begin{tikzcd}
&\arrow{ld}[swap]{x} T \arrow[d, dotted, "{\exists! (x, y)}"] \arrow[rd, "y"] & \\
A & \arrow[l, "\pi_1"] A\times B  \arrow{r}[swap]{\pi_2} & B
\end{tikzcd}
\end{center}

\end{definition}

Debemos resaltar dos aspectos importantes de esta definición.
\begin{itemize}
\item El producto de dos objetos en una categoría dada no tiene por qué
      existir.
\item De existir un producto, éste sólo queda determinado salvo isomorfismo. De hecho si $(A\times B, \pi_1, \pi_2)$ es un producto de $A$ y $B$
y $\phi : \arr{P}{A\times B}$ es un isomorfismo entonces
$(P, \pi_1 \circ \phi , \pi_2 \circ \phi)$ es otro producto de $A$
y $B$.
\end{itemize}

\subsection{Ejemplos}
\paragraph{\Set}
Dados dos conjuntos $A$ y $B$ siempre existe el producto de éstos
y además coincide (salvo el isomorfismo
que comentamos antes) con el producto cartesiano de ambos
junto a sus proyecciones canónicas. Desmotramos tal
afirmación a continuación.

Sean $x: \arr{T}{A}, y: \arr{T}{B}$ dos aplicaciones. Podemos
definir la aplicación $(x, y) :\arr{T}{A\times B}$ por
$(x, y)(t) = (x(t), y(t))$. Esta aplicación cumple en efecto que
$\pi_1 \circ (x, y) = x$ y $\pi_2 \circ y = y$. Pero además es la única
que cumple esto pues si tenemos otra aplicación
$f : \arr{T}{A\times B}$ cumpliendo $\pi_1 \circ f = x$ y
$\pi_2 \circ f = y$ sólo nos basta recordar que podemos
escribir $f$ como:
$f(t)=(\pi_1\circ f(t), \pi_2\circ f(t))=(x(t), y(t))$.
y por tanto $f = (x, y)$.

\paragraph{Otras estructuras matemáticas}
A continuación mostramos ejemplos de productos
en categorías conocidas.

\begin{itemize}
\item En $\Grp$ el producto se corresponde con el
producto directo de grupos.
\item En $\Top$ se corresponde con el producto de espacios topológicos.
\item En $\Ring$ se corresponde con el producto de anillos.
\end{itemize}

\section{Objetos finales}
Al principio de la sección le dimos un papel especial al conjunto
de un solo elemento $\uno$. Nos gustaría caracterizar a ese objeto de
la categoría $\Set$ con alguna propiedad estrictamente categórica.
Esto es posible tal y como vemos a continuación.

\begin{definition}
Sea $\C$ una categoría y $\final$ un objeto. Diremos que $\final$ es un objeto
final si dado cualquier objeto $T$ de la categoría existe un
único elemento de $\final$ definido sobre $T$.
\end{definition}

Comprobar que de existir el objeto $\final$ es único salvo isomorfismo
es inmediato.
\subsection{Ejemplos}
\paragraph{En $\Set$}
En la categoría de los conjuntos el objeto final es $\uno$,
el conjunto de un solo elemento.

\paragraph{En $\Grp$}
El grupo trivial es el objeto final de la categoría $\Grp$.
Esto es sencillo de comprobar: dado cualquier grupo $G$ tenemos
que el homomorfismo trivial $\arr{G}{\final}$ es el único morfismo
entre $G$ y $\final$.

\paragraph{En $\Ring$}
En la categoría de los anillos con unidad el objeto final es
también el anillo trivial.

\section{Dualidad}
El concepto de dualidad que se encuentra frecuentemente en
matemáticas puede ser analizado de forma muy general desde el punto
de vista categórico. Para introducir esta noción presentamos a
continuación la definición de categoría opuesta.

\subsection{La categoría opuesta}
\begin{definition}
Dada una categoría $\C$ definimos
$\C^{op}$ (a la que llamaremos categoría opuesta a $\C$)
como la categoría determinada por $\Obj{\C^{op}} = \Obj{\C}$,
$\Hom_{\C^{op}}(A, B) = \Hom_{\C}(B, A)$ y una operación de
composición dada por $f \circ_{op} g = g \circ f$.
\end{definition}

Comprobar que esta construcción es efectivamente una categoría
en sencillo.
\begin{itemize}
\item La composición es asociativa: sean
$f \in \Hom_{\C^{op}}(A, B)$, $g \in \Hom_{\C^{op}}(B, C)$
y $h \in \Hom_{\C^{op}}(C, D)$. Tenemos que
\begin{align*}
  (h \circ_{op} g) \circ_{op} f
    & = (g \circ h) \circ_{op} f
    = f \circ (g \circ h) \\
& = (f \circ g) \circ h
= (g \circ_{op} f)  \circ h
= h \circ_{op} (g \circ_{op} f)
\end{align*}
\item Existen las identidades: las identidades en $\C^{op}$
son las mismas flechas que en la categoría original $\C$.
\end{itemize}

La categoría opuesta nos otorga una herramienta para reaprovechar
definiciones realizadas anteriormente. Dada una propiedad $P$ que
se pueda o no cumplir en una categoría $\C$ llamamos propiedad
dual de $P$ a la propiedad de que se cumpla $P$ en la categoría opuesta
$\C^{op}$. Esto nos permite definir las siguientes propiedades.

\begin{itemize}
\item Ser epimorfismo es la propiedad dual de ser monomorfismo (es decir
$f$ es un epimorfismo en $\C$ si $f$ es un monomorfismo en la categoría opuesta
$\C^{op}$).
\item La propiedad dual de ser isomorfismo es ella misma. La propiedad
dual de ser único salvo isomorfismo es ella misma.
\item El coproducto es el dual del producto, es decir llamaremos
a $(A+B, i_1, i_2)$ el coproducto de $A$ y $B$ en $\C$ si
$(A+B, i_1, i_2)$ es el producto de $A$ y $B$ en $\C^{op}$.
\item La propiedad de ser objeto inicial es la dual a la de ser objeto final.
\end{itemize}

Merece la pena resaltar que ${\C^{op}}^{op} = \C$ y por tanto
la ``propiedad dual a la propiedad dual a $P$'' es la misma propiedad
$P$. Esto nos permite decir también, por ejemplo, que la propiedad
dual a ser epimorfismo es ser monomorfismo.

Comentamos estas propiedades en mayor profundidad en las siguientes
secciones.

\subsection{Epimorfismos}
Como ya hemos dicho, la definición de epimorfismo es la siguiente.
\begin{definition}
Dada una categoría $\C$ y una flecha $f: \arr{A}{B}$, diremos que
$f$ es un epimorfismo si y solo si $f$ es un monomorfismo en
$\C^{op}$.
\end{definition}
Esta propiedad
se puede caracterizar de forma sencilla desde dentro de $\C$.
\begin{proposition}
$f: \arr{A}{B}$ es un epimorfismo si y solo si dado cualquier objeto $X$ y
cualquier par de flechas $g_1, g_2 : \arr{B}{X}$,
tenemos que $g_1 \circ f = g_2 \circ f$ implica $g_1 = g_2$.
En este caso diremos también que $f$ se puede cancelar por la
derecha.
\end{proposition}

\subsubsection{Ejemplos}
\paragraph{En \Set}
En la categoría de los conjuntos los epimorfismos coinciden con las
aplicaciones sobreyectivas.

\paragraph{Isomorfismos}
Es sencillo probar que todo isomorfismo es a la vez
un epimorfismo y un monomorfismo.
El recíproco
es cierto en algunas categorías (como en $\Set$ o $\Grp$) pero no
lo es en general.

\paragraph{En \Ring}
Consideremos la categoría de anillos con unidad (en la
que los objetos son anillos con unidad y
las flechas son homomorfismos de anillos). En esta categoría
el hecho de que $f : \arr{R}{S}$ sea un monomorfismo
es equivalente a que $f$ sea una aplicación
inyectiva.

Consideremos la inclusión
$i : \arr{\Z}{\Q}$, un anillo $R$
y un par de aplicaciones de anillos $g_1, g_2 : \arr{\Q}{R}$. Supongamos
que $g_1 \circ i = g_2 \circ i$. Ahora para todo $x \in \Q$
tenemos que existen $a, b$ enteros tales que $x = \frac{a}{b}$ y entonces:
\begin{align*}
g_1(x) & = g_1(\frac{a}{b}) = g_1(a)g_1(b)^{-1}
       = (g_1 \circ i)(a)(g_1\circ i)(b)^{-1} \\
       & = (g_2\circ i)(a)(g_2\circ i)(b)^{-1}
       = g_2(a)g_2(b)^{-1} = g_2(x).
\end{align*}

Con lo que $g_1=g_2$. Esto prueba que $i : \arr{\Z}{\Q}$ es epimorfismo.
Claramente es también un monomorfismo. La categoría de anillos es
un ejemplo entonces de categoría en la que ser monomorfismo y
epimorfismo no es equivalente a ser isomorfismo.

\subsection{Objetos iniciales}
\begin{definition}
Dada una categoría $\C$ diremos que $\inicial$ es un objeto
inicial si $\inicial$ es un objeto final en la categoría
$\C^{op}$.
\end{definition}

Esta definición la podemos ver en exclusivamente en términos
de la categoría $\C$ de la siguiente forma:
\begin{proposition}
El objeto $\inicial$ es inicial si y solo si para cualquier otro objeto
$X$ existe una única flecha tiene a $\inicial$ como dominio y a $X$ como
codominio.
\end{proposition}

\subsubsection{Ejemplos}
\paragraph{En \Set}
En $\Set$ el conjunto vacío $\emptyset$ es un objeto inicial.

\paragraph{En \Grp}
Vimos anteriormente que el objeto final de $\Grp$ era el grupo
trivial. Resulta que el grupo trivial es también el objeto inicial
de la categoría. Cuando en una categoría coinciden el objeto inicial
$\inicial$ y el objeto final $\final$ llamamos a tal objeto el objeto
nulo y lo notamos por $\nulo$. La existencia de objetos nulos en una categoría nos garantiza
la existencia de una flecha distinguida entre cada par de objetos
$A$ y $B$ de la categoría, a la que llamaremos
flecha cero y la definimos por
$0_{A,B} = i \circ j : \arr{A}{B}$ donde
$j: \arr{A}{\nulo}$ y $i : \arr{\nulo}{B}$
son respectivamente la única flecha que va de $A$ al objeto nulo y
la única flecha que va del objeto nulo a $B$.


Otro ejemplo de categoría con objeto cero es la categoría de espacios
vectoriales sobre un cuerpo $K$.

\paragraph{En la categoría de cuerpos}
En la categoría de cuerpos no existen ni objetos finales ni objetos
iniciales. Para justificar por qué
supongamos que $K$ es un objeto final de la categoría
de cuerpos. La característica de $K$ es o bien 0 o bien un primo.
En cualquier caso si consideramos un cuerpo $K'$ con característica
distinta a la característica de $K$ es bien conocido que no existen
homomorfismos de cuerpos $\arr{K'}{K}$ y por tanto $K$ no
puede ser un objeto final. Para probar que no existen objetos iniciales
se razona de forma análoga.

\subsection{Coproducto}
Coproducto es la propiedad dual de producto. Si vemos qué significa
ser coproducto desde dentro de la misma categoría $\C$ obtenemos
la siguiente definición.

\begin{definition}
Sea $\C$ una categoría y $A$ y $B$ dos objetos de ésta. Diremos que
existe el coproducto de $A$ y $B$,
si existe una terna $(A+B, i_1, i_2)$,
donde $A+B$ es un objeto de $\C$, y
$i_1 : \arr{A}{A+B}, i_2 : \arr{B}{A+B}$ son dos flechas tales
que, para cualquier objeto $Y$, y sendas flechas $f_1 : \arr{A}{Y}$,
$f_2 : \arr{B}{Y}$, existe un único morfismo
$f : \arr{A+B}{Y}$ tal que el siguiente diagrama es conmutativo.
\begin{center}
\begin{tikzcd}
& &  Y & & \\
& & & &  \\
A \arrow[uurr, "f_1"] \arrow{rr}[swap]{i_1} & &
 \arrow{uu}[dotted, swap]{\exists! [f_1, f_2]} A+B
 & & B \arrow[ll, "i_2"] \arrow{uull}[swap]{f_2}
\end{tikzcd}
\end{center}
\end{definition}

\subsubsection{Ejemplos}
\paragraph{En \Set}
En $\Set$ el coproducto coincide con la unión disjunta.

\paragraph{En \Grp}
En $\Grp $ el coproducto coincide con el producto libre.

\subsection{Funtores $\Hom(-, A)$}
Dado un objeto $A$ de la categoría $\C$ definimos el funtor
$\Hom_\C(-, A)$ como
$$\Hom_\C(-, A) = \Hom_{\C^{op}}(A, -) : \arr{\C^{op}}{\Set}.$$

Sabemos cómo actúa $\Hom_\C(-, A)$ sobre las flechas de $\C^{op}$
pero si tenemos una flecha $f \in \Hom_\C(C, C')$ tenemos que:

$$\Hom_\C(f, A) = \Hom_{\C^{op}}(A, f) :
  \arr{\Hom_{\C^{op}}(A, C')}{\Hom_{\C^{op}}(A, C)},$$
ya que $f \in \Hom_{\C^{op}}(C', C)$. Dicho de otra forma:
$$\Hom_\C(f, A) : \arr{\Hom_{\C}(C', A)}{\Hom_{\C}(C, A)}$$
y su acción es:

$$\Hom_\C(f, A)(g) = g\circ f.$$
Además si tenemos dos flechas de $\C$ que se pueden componer
$f$ y $g$, tenemos que:

$$\Hom_\C(g \circ f, A) = \Hom_\C(f, A) \circ \Hom_\C(g, A),$$
con lo que la composición funciona \textit{al revés}. Diremos que
$F$ es un funtor contravariante sobre sobre $\C$ cuando $F$ sea
un funtor definido sobre $\C^{op}$. En este sentido
decimos que $\Hom_\C(-, A)$ es un funtor contravariante
sobre $\C$.

\subsection{Funtor opuesto}
Dado un funtor $F : \arr{\C}{\D}$ podemos definir el funtor opuesto
$F^{op} : \arr{\C^{op}}{\D^{op}}$ tal que $F^{op}C = FC$ y para cualquier
flecha $f \in \Hom_{\C^{op}}(C, C') = \Hom_{\C}(C', C)$,
$$F^{op}f = F f \in \Hom_\C(FC', FC) = \Hom_{\D^{op}}(FC, FC')$$

\section{En programación}
A lo largo de esta sección aplicamos alguna de
las construcciones que hemos visto
a la categoría $\Hask$.

\subsection{Objetos iniciales y finales}
$\Hask$ tiene tanto objetos iniciales como objetos finales. El
objeto final es el tipo con un solo valor, al que se le suele llamar
\cod{()} (pronunciado \cod{Unit}). El tipo \cod{()} tiene
como único valor el valor \cod{()} (tipo y valor
son homónimos en este caso) y para cada tipo \cod{a} podemos
definir la función:
\begin{verbatim}
constUnit :: a -> ()
constUnit x = ()
\end{verbatim}
Por otro lado tenemos que el objeto inicial de $\Hask$ es el tipo
llamado \cod{Void}, que no tiene ningún valor. En la biblioteca
estándar se encuentra una función llamada \cod{absurd}
que tiene
como tipo \cod{Void -> a} y es la única función con este tipo.
En cualquier caso, la función no puede ser llamada puesto que no
hay ningún valor con el que llamarla. La situación tanto como para
el objeto inicial como para el final es muy similar a la que se daba
en \Set.

\subsection{Productos}
$\Hask$ tiene productos. Dado dos tipos \cod{A} y
\cod{B} podemos construir el tipo \cod{(A, B)} que tiene
como valores pares de valores de \cod{A} y de \cod{B}.
Las proyecciones se implementan en la biblioteca estándar de
Haskell de la siguiente forma:

\begin{verbatim}
fst :: (a, b) -> a
fst (x, y) = x

snd :: (a, b) -> b
snd (x, y) = y
\end{verbatim}

Si tenemos un par de funciones \cod{f1 :: X -> A} y
\cod{f2 :: X -> B} podemos definir la función \cod{f}:
\begin{verbatim}
f :: X -> (A, B)
f x = (f1 x, f2 x)
\end{verbatim}

\subsection{Coproductos}
En la sección sobre funtores introdujimos el tipo \cod{Either}:
\begin{verbatim}
data Either a b = Left a | Right b
\end{verbatim}
Resulta que \cod{Either} es el coproducto en \cod{Hask}. Si
\cod{A} y \cod{B} son dos tipos de haskell entonces su coproducto
es \cod{Either A B} y las inyecciones canónicas son por un lado
\cod{Left :: A -> Either A B} y por otro lado
\cod{Right :: B -> Either A B}. Si tenemos un par de funciones
\cod{f1 :: A -> Y} y \cod{f2 :: B -> Y} tenemos que la única
función \cod{f :: Either A B -> Y} que se lleva bien con las inyecciones
es:
\begin{verbatim}
f :: Either A B -> Y
f (Left a) = f1 a
f (Right b) = f2 b
\end{verbatim}

Comprobar que se lleva bien con las inyecciones es inmediato:
\begin{verbatim}
(f . Left) a = f (Left a) = f1 a
-- por tanto f . Left = f1

(f . Right) b = f (Right b) = f2 b
-- por tanto f . Right = f2
\end{verbatim}

En la biblioteca estándar tenemos la función
\cod{either :: (a -> c) -> (b -> c) -> Either a b -> c}
que es precisamente la función que construye \cod{f} a partir
de \cod{f1, f2}.
