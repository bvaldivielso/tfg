En el primer capítulo introdujimos los funtores como los morfismos
entre las categorías. Ahora ha llegado el momento de dar un paso más
e introducir las transformaciones naturales como morfismos entre
funtores. Esta es una de las nociones que motivó la creación de la
teoría de categorías: se encuentran ejemplos de esta en múltiples
ramas de las matemáticas.

Procedemos con la definición:

\begin{definition}
  Dados dos funtores $F: \arr{\C}{\D}, G : \arr{\C}{\D}$ decimos
  que $\lambda : \nat{F}{G}$ es una transformación natural si $\lambda$
  asigna a cada objeto $C$ de $\C$ una flecha
  $\lambda_C : \arr{FC}{GC}$ de manera que para
  cualquier flecha $g : \arr{C}{C'}$ el siguiente diagrama
  es conmutativo:
\begin{center}
    \begin{tikzcd}
      F C \arrow{r}{\lambda_C} \arrow{d}[swap]{F g} & GC \arrow{d}{G g} \\
      F C' \arrow{r}{\lambda_{C'}} & G C'
    \end{tikzcd}
  \end{center}
\end{definition}

Los siguientes resultados nos permitirán considerar categorías en las
que los objetos son funtores entre dos categorías $\C$ y $\D$ y las
flechas son las transformaciones naturales entre los funtores:

\begin{proposition}
  \begin{enumerate}
  \item Dado cualquier funtor $F : \arr{\C}{\D}$ podemos definir
    la transformación natural $1_F : \nat{F}{F}$ dada por
    ${(1_F)}_C = 1_{(FC)}$
    para cada objeto $C$ de $\C$.
  \item Podemos componer transformaciones naturales de la siguiente
    forma: dados funtores $F, G, H : \arr{\C}{\D}$ y transformaciones
    naturales $\lambda : \nat{F}{G}$, $\sigma : \nat{G}{H}$
    definimos la transformación natural
    $\sigma \circ \lambda : \nat{F}{H}$ por
    $(\sigma\circ\lambda)_C = \sigma_C \circ \lambda_C$.
  \item La composición de transformaciones naturales
    es asociativa en el siguiente sentido: dados
    $F \xrightarrow{\lambda} G \xrightarrow{\sigma} H \xrightarrow{\tau}I$
    tenemos que $(\tau \circ \sigma) \circ \lambda = \tau \circ (\sigma \lambda)$
  \item Dado cualquier par de funtores $F,G : \arr{\C}{\D}$ y
    transformaciones naturales $\tau : \arr{F}{G}$,
    $\sigma : \arr{G}{F}$ tenemos que $1_F \circ \sigma = \sigma$
    y $\tau \circ 1_F = \tau$.
  \end{enumerate}

  En definitiva los funtores $\arr{\C}{\D}$ y las transformaciones
  naturales entre estos funtores forman una categoría. A esta categoría
  la llamaremos $\funccat{\C}{\D}$.
\end{proposition}


\begin{definition}
  Dados dos funtores $F, G : \arr{\C}{\D}$ diremos que la transformación
  natural $\lambda : \nat{F}{G}$ es un isomorfismo natural si
  $\lambda_C : \arr{FC}{GC}$ es un isomorfismo para todo objeto $C$
  de $\C$.
\end{definition}

Se puede justificar además que los isomorfismos naturales
son precisamente los isomorfismos en las categorías de funtores.

\subsection{Ejemplos}
\paragraph{El doble dual}
Consideremos la categoría de espacios vectoriales de dimensión
finita sobre un cuerpo $K$, a la que llamaremos $\VectK$. Podemos
definir el funtor $(-)^{**}: \arr{\VectK}{\VectK}$, al que llamaremos
doble dual, que lleva un espacio vectorial $V$ a su doble dual
$V^{**}$ y una aplicación lineal
$f: \arr{V}{W}$ a la correspondiente aplicación
lineal
$f^{**} : \arr{V^{**}}{W^{**}}$ definida por

$$f^{**}(g)(\phi) = g(\phi\circ f)$$
Podemos probar que el funtor doble dual y el funtor identidad
de la categoría $\VectK$ son naturalmente isomorfos. El isomorfismo
natural es $\lambda : \nat{1_{\VectK}}{(-)^{**}}$ definido por:

$$\lambda_V : \arr{V}{V^{**}}$$
$$\lambda_V(v)(\phi) = \phi(v)$$

\paragraph{Abelianización de grupos}
Definimos el funtor $(-)^{ab} : \arr{\Grp}{\Grp}$
como el funtor que lleva
cada grupo $G$ a su abelianización
$G^{ab} =\frac{G}{[G, G]}$

y cada homomorfismo de grupos $f : \arr{G}{H}$ al homomorfismo
de grupos dado por:

$$f^{ab} : \arr{G^{ab}}{H^{ab}}$$
$$f^{ab}(g[G, G]) = \pi_H (f(g))$$

Donde $\pi_H$ es la proyección al cociente.
La aplicación está bien definida porque los homomorfismos
de grupos llevan conmutadores en conmutadores y como
$H^{ab}$ es un grupo abeliano y el único conmutador de un
grupo abeliano es el 1 se tiene que
$[G, G] \subseteq \ker (\pi_H \circ f)$.


Considerando una vez más las proyecciones
$\pi_G : \arr{G}{G^{ab}}$ nos daremos cuenta
que $\pi : \nat{1_\Grp}{(-)^{ab}}$ es precisamente
una transformación natural.
