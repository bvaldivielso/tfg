En el primer capítulo introdujimos los funtores como los morfismos
entre las categorías. Ahora ha llegado el momento de dar un paso más
e introducir las transformaciones naturales como morfismos entre
funtores. Esta es una de las nociones que motivó la creación de la
teoría de categorías: se encuentran ejemplos de esta en múltiples
ramas de las matemáticas.

Procedemos con la definición:

\begin{definition}
  Dados dos funtores $F: \arr{\C}{\D}, G : \arr{\C}{\D}$ decimos
  que $\lambda : \arr{F}{G}$ es una transformación natural si $\lambda$
  asigna a cada objeto $C$ de $\C$ una flecha
  $\lambda_C : \arr{FC}{GC}$ de manera que para
  cualquier flecha $g : \arr{C}{C'}$ el siguiente diagrama
  es conmutativo:


  \begin{center}
    \begin{tikzcd}
      F C \arrow{r}{\lambda_C} \arrow{d}[swap]{F g} & GC \arrow{d}{G g} \\
      F C' \arrow{r}{\lambda_{C'}} & G C'
    \end{tikzcd}
  \end{center}
\end{definition}

Los siguientes resultados nos permitirán considerar categorías en las
que los objetos son funtores entre dos categorías $\C$ y $\D$ y las
flechas son las transformaciones naturales entre los funtores:

\begin{proposition}
  \begin{enumerate}
  \item Dado cualquier funtor $F : \arr{\C}{\D}$ podemos definir
    $({1_F})_C  = 1_{(FC)} : \arr{FC}{FC}$. $1_F$ es una transformación
    natural entre $1_F : \arr{F}{F}$.
  \item Podemos componer transformaciones naturales de la siguiente
    forma: dados funtores $F, G, H : \arr{\C}{\D}$ y transformaciones
    naturales $\lambda : \arr{F}{G}$, $\sigma : \arr{G}{H}$ podemos
    definir $\sigma \circ \lambda$ dada
    por sus componentes
    $(\sigma\circ\lambda)_C = \sigma_C \circ \lambda_C$.
    $\sigma\circ\lambda$ es una transformación natural
    $\sigma\circ\lambda : \arr{F}{H}$.
  \item La composición de transformaciones naturales
    es asociativa en el siguiente sentido: dados
    $F \xrightarrow{\lambda} G \xrightarrow{\sigma} H \xrightarrow{\tau}I$
    tenemos que $(\tau \circ \sigma) \circ \lambda = \tau \circ (\sigma \lambda)$
  \item Dado cualquier par de funtores $F,G : \arr{\C}{\D}$ y
    transformaciones naturales $\tau : \arr{F}{G}$,
    $\sigma : \arr{G}{F}$ tenemos que $1_F \circ \sigma = \sigma$
    y $\tau \circ 1_F = \tau$.
  \end{enumerate}

  En definitiva los funtores $\arr{\C}{\D}$ y las transformaciones
  naturales entre estos funtores forman una categoría. A esa categoría
  la llamaremos $[\C, \D]$.
\end{proposition}

\subsection{Ejemplos}
\paragraph{El doble dual}
Consideremos la categoría de espacios vectoriales sobre un cuerpo
$K$ a la que llamaremos $\VectK$. Podemos considerar los endofuntores
identidad $1_{\VectK} : \arr{\VectK}{\VectK}$ y el funtor doble dual:
$(-)^{**} : \arr{\VectK}{\VectK}$. El último de estos funtores actúa
así sobre los morfismos:

$$(-)^{**} : \arr{\Hom(V, W)}{\Hom(V^{**}, W^{**})}$$
$$f^{**}(g)(\phi) = g(\phi \circ f)$$

Para cada espacio vectorial sobre $K$ $V$ podemos además
definir un monomorfismo de espacios vectoriales
 (isomorfismo
en el caso finito-dimensional) en su doble dual:

\begin{equation*}
i_V : \arr{V}{V^{**}}
\end{equation*}
\begin{equation*}
i_V(v)(\phi) = \phi(v)
\end{equation*}

Pues $i$ es una transformación natural entre $1_K$ y
$(-)^{**}$. Más aun, si nos restringimos al caso finito dimensional,
debido a que todas las componentes de la transformación natural
son isomorfismos, decimos que ambos funtores son naturalmente isomorfos.

\paragraph{Abelianización de grupos}
Dado un grupo $G$ definimos su abelianización $G^{ab}$ como
$G^{ab} = \frac{G}{[G, G]}$. Llamemos $\pi_G : \arr{G}{G^{ab}}$
a la proyección sobre el cociente. Para cualquier morfismo
de grupos $f : \arr{G}{H}$ tenemos que la aplicación
$\pi_H \circ f$ contiene en su nucleo a $[G, G]$ (ya que
$H^{ab}$ es abeliano y un morfismo de un grupo a un grupo abeliano
lleva los conmutadores al 0) y por tanto factoriza a través
de $G^{ab}$ permitiéndonos definir una aplicación
$f^{ab} : \arr{G^{ab}}{H^{ab}}$ y comprobar
que $(-)^{ab}$ es un funtor. La proyección de $\pi$ es una transformación
natural $\pi : \arr{ 1_{\Grp} }{ (-)^{ab} }$.
