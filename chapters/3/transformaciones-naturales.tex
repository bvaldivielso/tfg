En el primer capítulo introdujimos la noción de funtor
como morfismo
entre categorías. Llegados a este punto
podemos dar un paso más
e introducir el concepto de transformación
natural como el morfismo
entre funtores.

Procedemos con la definición:

\begin{definition}
  Dados dos funtores $F: \arr{\C}{\D}$ y $G : \arr{\C}{\D}$ decimos
  que $\lambda : \nat{F}{G}$ es una transformación natural si $\lambda$
  asigna a cada objeto $C$ de $\C$ una flecha
  $\lambda_C : \arr{FC}{GC}$ de manera que para
  cualquier flecha $g : \arr{C}{C'}$ el siguiente diagrama
  es conmutativo:
\begin{center}
    \begin{tikzcd}
      F C \arrow{r}{\lambda_C} \arrow{d}[swap]{F g} & GC \arrow{d}{G g} \\
      F C' \arrow{r}{\lambda_{C'}} & G C'
    \end{tikzcd}
  \end{center}
\end{definition}

\subsubsection{Ejemplos}
\paragraph{El doble dual}
Consideremos la categoría de espacios vectoriales de dimensión
finita sobre un cuerpo $K$, a la que llamaremos $\VectK$. Podemos
definir el funtor $(-)^{**}: \arr{\VectK}{\VectK}$, al que llamaremos
doble dual, que lleva un espacio vectorial $V$ a su doble dual
$V^{**}$ y una aplicación lineal
$f: \arr{V}{W}$ a la correspondiente aplicación
lineal
$f^{**} : \arr{V^{**}}{W^{**}}$ definida por
$$f^{**}(g)(\phi) = g(\phi\circ f)$$
Podemos probar que existe una transformación
natural entre el funtor doble dual y el funtor identidad
de la categoría $\VectK$. La transformación
natural es $\lambda : \nat{1_{\VectK}}{(-)^{**}}$ definido por:
$$\lambda_V : \arr{V}{V^{**}}$$
$$\lambda_V(v)(\phi) = \phi(v)$$
Nótese además que para cada espacio vectorial $V$ se tiene
que $\lambda_V$ es un isomorfismo de espacios vectoriales.

\paragraph{Abelianización de grupos}
Definimos el funtor $(-)^{ab} : \arr{\Grp}{\Grp}$
como el funtor que lleva
cada grupo $G$ a su abelianización
$G^{ab} =\frac{G}{[G, G]}$
y cada homomorfismo de grupos $f : \arr{G}{H}$ al homomorfismo
de grupos dado por:
$$f^{ab} : \arr{G^{ab}}{H^{ab}}$$
$$f^{ab}(g[G, G]) = \pi_H (f(g))$$
Donde $\pi_H$ es la proyección al cociente.
La aplicación está bien definida porque los homomorfismos
de grupos llevan conmutadores en conmutadores y como
$H^{ab}$ es un grupo abeliano y el único conmutador de un
grupo abeliano es el 1 se tiene que
$[G, G] \subseteq \ker (\pi_H \circ f)$.

Comprobar que $\pi: \nat{1_\Grp}{(-)^{ab}}$ es una transformación natural
se reduce a comprobar que para cada homomorfismo de grupos
$f: \arr{G}{H}$ el siguiente diagrama es conmutativo:
\begin{center}
  \begin{tikzcd}
    G \arrow{r}{\pi_G} \arrow{d}{f} & G^{ab} \arrow{d}{f^{ab}} \\
    H \arrow{r}{\pi_H} & H^{ab}
  \end{tikzcd}
\end{center}

\subsection{Categorías de funtores}
Los siguientes resultados nos permitirán componer
transformaciones naturales y considerar categorías en las
que los objetos son funtores entre dos categorías $\C$ y $\D$ y las
flechas son precisamente
las transformaciones naturales entre estos funtores:

\begin{proposition}
  \begin{enumerate}
  \item Dado cualquier funtor $F : \arr{\C}{\D}$ podemos definir
    la transformación natural $1_F : \nat{F}{F}$ dada por
    ${(1_F)}_C = 1_{(FC)}$
    para cada objeto $C$ de $\C$.
  \item Podemos componer transformaciones naturales de la siguiente
    forma: dados funtores $F, G, H : \arr{\C}{\D}$ y transformaciones
    naturales $\lambda : \nat{F}{G}$, $\sigma : \nat{G}{H}$
    definimos la transformación natural
    $\sigma \circ \lambda : \nat{F}{H}$ por
    $(\sigma\circ\lambda)_C = \sigma_C \circ \lambda_C$.
  \item La composición de transformaciones naturales
    es asociativa en el siguiente sentido: dados
    $F \xrightarrow{\lambda} G \xrightarrow{\sigma} H \xrightarrow{\tau}I$
    tenemos que $(\tau \circ \sigma) \circ \lambda = \tau \circ (\sigma \lambda)$
  \item Dado cualquier par de funtores $F,G : \arr{\C}{\D}$ y
    transformaciones naturales $\tau : \arr{F}{G}$,
    $\sigma : \arr{G}{F}$ tenemos que $1_F \circ \sigma = \sigma$
    y $\tau \circ 1_F = \tau$.
  \end{enumerate}

  En definitiva los funtores $\arr{\C}{\D}$ y las transformaciones
  naturales entre estos funtores forman una categoría. Notaremos
  esta categoría
  como $\funccat{\C}{\D}$.
\end{proposition}

\subsection{Isomorfismos naturales}
\begin{definition}
  Dados dos funtores $F, G : \arr{\C}{\D}$ diremos que la transformación
  natural $\lambda : \nat{F}{G}$ es un isomorfismo natural si
  $\lambda_C : \arr{FC}{GC}$ es un isomorfismo para todo objeto $C$
  de $\C$.
\end{definition}
Podemos relacionar los isomorfismos naturales y las categorías
de funtores de la siguiente manera:
\begin{proposition}
  Sean $F, G : \arr{\C}{\D}$ dos funtores. Entonces
  $\phi: \nat{F}{G}$
  es un isomorfismo natural si y solo si es un isomorfismo
  en la categoría $\funccat{\C}{\D}$.
\end{proposition}
\begin{proof}
  Supongamos que $\phi: \nat{F}{G}$ es un isomorfismo natural. Definimos
  $\psi: \nat{G}{F}$ como $\psi_C = \phi_C^{-1}$ (podemos
  considerar la inversa de $\phi_C$ gracias a que $\phi_C$ es un isomorfismo de
  $\C$ por ser $\phi$ un isomorfismo natural). Comprobar la
  naturalidad de $\psi$ consiste en ver que para cualquier
  flecha $f: \arr{C}{C'}$ se tiene que el siguiente diagrama es
  conmutativo:
  \begin{center}
    \begin{tikzcd}
      GC \arrow{r}{\psi_C} \arrow{d}[swap]{Gf} & FC' \arrow{d}{Ff} \\
      GC' \arrow{r}{\psi_{C'}} & FC'
    \end{tikzcd}
  \end{center}
  Es decir, $\psi_{C'} \circ Gf = Ff \psi_C$. Esto es cierto sí y solo
  sí $Gf \circ \phi_C = \phi_{C'} \circ Ff$ (componiendo a la izquierda
  con el isomorfismo $\phi_{C'}$ y a la derecha con el isomorfismo
  $\phi_C$), pero esta última igualdad se deduce de la naturalidad
  de $\phi: \nat{F}{G}$. Entonces
  $\psi$ es la inversa de $\phi$ y por tanto
  $\phi$ es un isomorfismo en la categoría de funtores
  $\funccat{\C}{\D}$.


  Supongamos ahora que $\phi: \nat{F}{G}$ es un isomorfismo en la
  categoría de funtores $\funccat{\C}{\D}$. Entonces $\phi$ tiene
  una inversa $\psi: \nat{G}{F}$ (es decir
  $\psi \circ \phi = 1_F$ y $\phi \circ \psi = 1_G$)
  por lo que para cada objeto
  $C$ de $\C$ tenemos que $\phi_C \circ \psi_C = 1_{GC}$ y
  además $\phi_C \circ \psi_C = 1_{FC}$, es decir, $\phi_C$
  es un isomorfismo en $\C$. Esto prueba que
  $\phi: \nat{F}{G}$ es un isomorfismo natural.
\end{proof}

\subsection{Transformaciones naturales a través de funtores}
Sean $F: \arr{\C}{\D}$ y $G: \arr{\C}{\D}$ dos funtores y
$\tau : \nat{F}{G}$ una transformación natural entre ellos.
Entonces:
\begin{itemize}
\item Dado un funtor $H: \arr{\D}{\D'}$ definimos
$(H\tau)_A = H(\tau_A)$ para cada objeto $A$ de
$\C$. $H\tau$ es una transformación natural
$H\tau : \nat{HF}{HG}$.
\item Dado un funtor $K: \arr{\C'}{\C}$ definimos
$(\tau K)_A' = \tau_{KA'}$ para cualquier objeto
$A'$ de $\C'$. $\tau K$ es una transformación natural
$\tau K : \nat{FK}{GK}$.
\end{itemize}
