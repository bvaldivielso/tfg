\subsection{Transformaciones Naturales}
Para construir una transformación natural entre dos endofuntores de
$\Hask$ $F$ y $G$ necesitamos, en primer lugar, una función de tipo
\cod{FA -> GA} para \cod{A} cualquier tipo del lenguaje.

Usamos el sistema de polimorfismo de Haskell para suplir nuestra primera
necesidad. El lenguaje nos permite definir una función de
tipo \cod{F a -> G a} (una función polimórfica en la variable de
tipo \cod{a}) que, especializando a cada tipo del lenguaje, será
la componente en cada objeto de $\Hask$ de nuestra transformación natural.

Un ejemplo de este tipo de función polimórfica sería:
\begin{verbatim}
discardLeft :: Either b a -> Maybe a
discardLeft (Left stringQueNoQuiero) = Nothing
discardLeft (Right v) = Just v
\end{verbatim}
(recordemos que es \cod{Either b}, y no \cod{Either} a secas,
la instancia de la clase \cod{Functor})

En segundo lugar, necesitamos que se cumplan las condiciones de
naturalidad. Podríamos emplear nuestro tiempo en demostrar que
la función \cod{discardLeft} que hemos definido cumple
tales condiciones pero eso no
será necesario: el sistema de polimorfismo de
Haskell (\textit{parametric polymorphism}) nos lo garantiza.
Haskell nos impone la restricción de que
una función polimórfica en el la variable de tipo \cod{a}
esté implementada
con la misma expresión. Por ejemplo, en Haskell no es posible
escribir una función polimórfica de esta forma:
\begin{verbatim}
polimorfica :: a -> a
polimorfica v = v

polimorfica :: Int -> Int
polimorfica n = n + 1
\end{verbatim}
Esta restricción, junto con otras características
del polimorfismo paramétrico, permite comprobar que cualquier
función de tipo \cod{F a->G a} con \cod{F} y \cod{G} funtores es
una transformación natural \cite{theorems-for-free}.

Mostramos a continuación algunos ejemplos de transformaciones naturales
que podemos escribir entre los funtores de Haskell que hemos visto
anteriormente.

\subsubsection{Ejemplos}
\paragraph{Entre \cod{Maybe} y \cod{Either}}
La función \cod{discardLeft :: Either b a -> Maybe a} que hemos definido
antes es un ejemplo de transformación natural. La interpretación
que dimos de estos tipos en el primer capítulo fue la siguiente:
\begin{itemize}
\item Los valores de la forma \cod{Nothing :: Maybe a} representan
  situaciones de error durante un cómputo. Los valores de la forma
  \cod{Just v :: Maybe a} representan un cómputo exitoso que tiene
  como al valor \cod{v :: a} como valor de retorno.
\item Los valores de la forma \cod{Left error :: Either b a} representan
  situaciones de error durante un cómputo y además aportan más información
  sobre el error (por ejemplo si la variable \cod{b} estuviera
  especializada a \cod{String} podríamos tener un valor de error
  que fuera \cod{Left "No fue posible conectar al servidor"}. Los
  valores de la forma \cod{Right v :: Either b a} representan cómputos
  exitosos con valor de retorno \cod{ v :: a }.
\end{itemize}

En este sentido podemos implementar la función \cod{discardLeft} como
una aplicación que recibe un cómputo posiblemente fallido y descarta
la información asociada al fallo en caso de que la haya. En caso de
computación exitosa deja invariante su resultado.

Podemos escribir una transformación natural en el sentido inverso
fijando un valor de tipo \cod{b}. Por ejemplo

\begin{verbatim}
maybeToRight :: b -> Maybe a -> Either b a
maybeToRight defaultError Nothing = Left defaultError
maybeToRight defaultError (Just v) = Right v
\end{verbatim}

Y \cod{maybeToRight} no es una transformación natural pero para
cualquier valor \cod{x :: B} tenemos que
\cod{maybeToRight x :: Maybe a -> Either B a} sí que lo es. Bajo
la interpretación de cómputos propensos a error, esta transformación
natural añade una justificación del error a un cómputo fallido
sin contexto del fallo.

\paragraph{Transformaciones naturales al funtor constante}
En Haskell podemos definir el funtor constante de la siguiente
manera:
\begin{verbatim}
data Const b a = Const b

instance Functor (Const b) where
  fmap f (Const b) = Const b
\end{verbatim}

Este es un funtor distinto a los anteriores que hemos
definido en Haskell demás en el sentido de que no utiliza
la variable de tipo de ninguna forma. \textit{Escribir esto mejor}.


Supongamos que tenemos una función polimórfica de tipo
\cod{f :: F a -> B} donde \cod{B} es un tipo fijo. Esta función
se puede extender a otra función
\cod{fConst :: F a -> Const B a} de la siguiente forma:
\begin{verbatim}
-- recordamos f :: F a -> B
fConst :: F a -> Const B a
fConst x = Const (f x)
\end{verbatim}
El tipo de \cod{fConst} (es de la forma \cod{F a -> G a} donde tanto
\cod{F} como \cod{G} son funtores) nos garantiza que es una transformación
natural. En particular cumple las condiciones de naturalidad. Si
escribimos estas condiciones de naturalidad en haskell llegamos a:

\begin{verbatim}
-- dado g :: c -> c'
fConst . fmap g = fmap g . fConst
\end{verbatim}

donde el \cod{fmap} de la izquierda es el de \cod{F} y el de
la izquierda es el de \cod{Const B}. Pero el de \cod{Const b}
deja invariante el resultado. Esto nos lleva a comprobar que:
\cod{f . fmap g = f}. Es decir, si modificamos un valor de
nuestro funtor con alguna función de la forma \cod{fmap g} no
estaremos variando la métrica. Un ejemplo de esto lo podemos
ver con la función \cod{length :: [a] -> Int}.
