El lema de Yoneda es uno de los primeros resultados
inesperados que surgen a raíz de la teoría de categorías. Una
de las consecuencias más importantes de este resultado es
el \textit{embebimiento} (cambiar \textcolor{red}{yo suelo usar encaje}) de Yoneda, que nos permite
ver una categoría $\C$ cualquiera dentro de la categoría de
funtores $\funccat{\C}{\Set}$. Necesitaremos algunos resultados
previos para enunciar el lema:

\textcolor{red}{yo sacaría la definicion de $\Nat$ de la proposición. Diciendo algo así: Vamos a escribir $\Nat(-,-)$ en lugar de $Hom_{[\C,\D]}(-,-)$}

\begin{proposition*}
  Dada una categoría $\C$ y un objeto suyo
  $A$,
  $$\Nat(\Hom(A, -), -) : \arr{\C \times \funccat{\C}{\Set}}{\Set}$$
  es un bifuntor donde $\Nat(F, G)$ es el conjunto de transformaciones
  naturales entre los funtores $F$ y $G$.
\end{proposition*}
\begin{proof}
  Consideramos el funtor identidad
  $1_{\funccat{\C}{\Set}} :\arr{\funccat{\C}{\Set}}{\funccat{\C}{\Set}}$ y el funtor
  $\Hom(*, -) : \arr{C^{op}}{\funccat{\C}{\Set}}$ (demostrar)
  el producto de estos nos define un funtor:
  $$\Hom(*, -)\times 1_{\funccat{\C}{\Set}} :
  \arr{\C\times \funccat{\C}{\Set}}{ \funccat{\C^{op}}{\Set} \times \funccat{\C}{\Set} }$$
  pero sabemos que
  $$\Nat(-, -) = \Hom_{\funccat{\C}{\Set}}(-, -) :
    \arr{\funccat{\C^{op}}{\Set}\times\funccat{\C}{\Set}}{\Set}$$
    es un funtor. La composición de estos dos funtores
    que hemos dicho es el funtor que buscamos.

 \textcolor{red}{esto es un poco mucho caótico, por lo pronto el nombre $Hom(*,-) $ es horrible, luego en los dos funtores que vas a componer no aparece $A$ por ningún lado...por último has dicho quien es $\Nat(-,-)$ por tanto $\Nat(F,-):[\C ,\Set]\rightarrow \Set$ estaría claro para cualquier $F\in [\C ,\Set]^{op}=[\C^{op},\Set]$, en particular estaría claro quien es $\Nat(\Hom(A, -), -)$ pero como funtor de  $\funccat{\C}{\Set}$ en $\Set$  }
\end{proof}

\textcolor{red}{me quedo aquí para darte tiempo a que repases esto.......recuerda que la semana que viene sólo estoy aquí el lunes......creo que antes de seguir deberías quitar todos los comentarios en rojo, yo posiblemente no vuelva a leer los cambios que hagas a menos que me indiques que lea algo en particular.....}



Necesitaremos probar que podemos definir el siguiente funtor:

\begin{proposition*}
  Sea $\C$ una categoría y $Ev(-, -) : \arr{\C\times\funccat{\C}{\Set}}{\Set}$
  dado por $Ev(C, F) = FC$. $E$ es un funtor y lo llamamos
  \textbf{funtor de evaluación}.
\end{proposition*}
\begin{proof}
  Sean $\sigma : \arr{F}{G}$ y $\tau : \arr{G}{H}$ transformaciones naturales entre funtores
  $F, G, H : \arr{\C}{\Set}$. Sean además $f : \arr{C}{D}$,
  $g : \arr{D}{E}$ flechas de $\C$.
  Podemos definir la acción de $Ev$ sobre las flechas de
  $\arr{\C\times\funccat{\C}{\Set}}{\Set}$ como
  $Ev(f, \sigma) = \sigma_D \circ Ff : \arr{FC}{GD}$. Veamos que $E$ se comporta bien
  respecto a las composiciones:
  \begin{multline*}
    Ev(g \circ f, \tau\circ\sigma) =
    (\tau \circ \sigma)_E\circ F(g\circ f) =
    \tau_E\circ\sigma_E\circ Fg \circ Ff = \\
  \tau_E\circ Gg \circ\sigma_D \circ Ff =
  \tau_E\circ Gg \circ Ev(f, \sigma) =\\
  Ev(g, \tau) \circ Ev(f, \sigma)
  \end{multline*}

  El hecho de que $Ev(1_C, 1_F) = 1_{FC}$ se prueba de forma sencilla
  y por tanto $Ev : \arr{\C\times\funccat{\C}{\Set}}{\Set}$ es un funtor

  \textit{Explicar mejor la demostración}
\end{proof}

Ahora que conocemos estos dos funtores podemos enunciar el lema
de Yoneda de forma sencilla.

\begin{theorem}[Lema de Yoneda]
  Sea $\C$ una categoría y $A$ un objeto de esta. Los funtores
  $$(A, F) \mapsto \Nat(\Hom(A, -), F)$$
  $$(A, F) \mapsto FA$$
  son naturalmente isomorfos.
\end{theorem}
\begin{proof}
  Sea $a \in F A$. Definimos la transformación natural entre
  $\lambda_a : \arr{\Hom(A, -)}{F}$ componente a componente de
  la siguiente forma: $(\lambda_a)_C(f) = f(a)$.
  \textit{Continuar mañana}
\end{proof}
