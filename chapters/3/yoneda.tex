El lema de Yoneda es uno de los primeros resultados
inesperados que surgen a raíz del
estudio de la teoría de categorías. Una
de las consecuencias más importantes de este resultado es
el encaje de Yoneda, que nos permite
ver una categoría $\C$ arbitraria dentro de la categoría de
funtores $\funccat{\C^{op}}{\Set}$. Necesitaremos algunos resultados
previos para poder enunciar el lema.

A lo largo de las siguientes
secciones usaremos la notación $\Nat(F, G)$ en lugar de
$\Hom_{\funccat{\C}{\D}}(F, G)$ para referirnos al conjunto de
transformaciones naturales entre dos funtores
$F, G: \arr{\C}{\D}$. Probamos la funtorialidad de uno de los
ingredientes principales del lema de Yoneda.
\begin{proposition*}
  Dadas una categoría $\C$ y un objeto $A$ de esta definimos:
  $$L : \arr{\C\times\funccat{\C}{\Set}}{\Set}$$
  $$L(A, F) = \Nat(\Hom(A, -), F)$$
  $L$ es un funtor.
\end{proposition*}
\begin{proof}
  Llamando
  $H: \arr{\C^{op}}{\funccat{\C}{\Set}}$ al funtor dado por
  $H(A) = \Hom_\C(A, -)$
  podemos notar que nuestro funtor $L$ es precisamente
  $$L :\C\times\funccat{\C}{\Set}
  \xrightarrow{H^{op}\times 1_{\funccat{\C}{\Set}}}
  (\funccat{C}{\Set})^{op} \times \funccat{C}{\Set}
  \xrightarrow{\Nat} \Set$$
\end{proof}
Probamos la funtorialidad
del otro ingrediente principal del lema de Yoneda.
\begin{proposition*}
  Sea $\C$ una categoría y $Ev(-, -) : \arr{\C\times\funccat{\C}{\Set}}{\Set}$
  dado por $Ev(C, F) = FC$. $Ev$ es un funtor y lo llamamos
  \textbf{funtor de evaluación}.
\end{proposition*}
\begin{proof}
  Sean $\sigma : \arr{F}{G}$ y $\tau : \arr{G}{H}$ transformaciones naturales entre funtores
  $F, G, H : \arr{\C}{\Set}$. Sean además $f : \arr{C}{D}$,
  $g : \arr{D}{E}$ flechas de $\C$.
  Definimos la acción de $Ev$ sobre las flechas de
  $\arr{\C\times\funccat{\C}{\Set}}{\Set}$ como
  $Ev(f, \sigma) = \sigma_D \circ Ff : \arr{FC}{GD}$. Veamos que $Ev$ se comporta bien
  respecto a las composiciones:
  \begin{align*}
    Ev(g \circ f, \tau\circ\sigma) &
     = (\tau \circ \sigma)_E\circ F(g\circ f)
     = \tau_E\circ\sigma_E\circ Fg \circ Ff \\
     & \stackrel{*}{=}
  \tau_E\circ Gg \circ\sigma_D \circ Ff
  = Ev(g, \tau) \circ Ev(f, \sigma)
  \end{align*}

  Donde la igualdad $(*)$ se deduce de la
  naturalidad de $\sigma : \nat{F}{G}$.
  Que $Ev(1_C, 1_F) = 1_{FC}$ se prueba de forma sencilla,
  confirmando que $Ev : \arr{\C\times\funccat{\C}{\Set}}{\Set}$ es un funtor
\end{proof}
Ya estamos preparados para enunciar el lema.
\begin{theorem}[Lema de Yoneda]
  Sea $\C$ una categoría. Entonces los funtores $L$ y $Ev$ son naturalmente isomorfos.
  En particular, para cada objeto $A$ de $\C$ y cada funtor $F : \arr{\C}{\Set}$
  se tiene una biyección natural
  $$\Nat(\Hom(A, -), F) \cong FA$$
\end{theorem}
\begin{proof}
  Definimos la aplicación
  $$\phi_{A, F} : \arr{\Nat(\Hom(A, -), F)}{FA}$$
  $$\phi_{A, F}(\tau) = \tau_A (1_A)$$
  (recordamos que $\tau_A$ por ser $\tau$ transformación natural entre $\Hom(A, -)$ y $F$ es
  induce una aplicación de conjuntos $\tau_A : \arr{\Hom(A, A)}{F A}$).

  Veamos que $\phi_{A, F}$ es una biyección. En primer lugar veremos que es sobreyectiva. Sea $x \in FA$
  Definimos la transformación natural $\lambda_x : \nat{\Hom(A, -)}{F}$ por
  $$(\lambda_x)_C : \arr{\Hom(A, C)}{F C}$$
  $$(\lambda_x)_C(f) = Ff(x)$$

  Comprobar que $\lambda_x$ es en efecto una transformación natural es sencillo. Pero
  $\phi_{A, F}(\lambda_x) = (\lambda_x)_A (1_A) = F1_A(x) = x$, luego $\phi_{A, F}$
  es sobreyectiva.

  Veamos ahora que $\phi_{A, F}$ es inyectiva. Supongamos que tenemos
  dos transformaciones naturales $\tau, \tau' : \nat{\Hom(A, -)}{F}$
  tales que $\phi_{A, F}(\tau) = \phi_{A, F}(\tau')$. Para todo $f \in \Hom(A, C)$
  la naturalidad de $\tau$ nos garantiza la conmutatividad
  del siguiente diagrama:
  \begin{center}
    \begin{tikzcd}
      \Hom(A, A) \arrow{d}[swap]{\Hom(A, f)} \arrow{r}{\tau_A}
            & FA \arrow{d}{Ff} \\

      \Hom(A, C) \arrow{r}{\tau_C} & F C
    \end{tikzcd}
  \end{center}


  Es decir $\tau_C \circ \Hom(A, f) = F f \circ \tau_A$. Aplicando
  esta flecha sobre el valor $1_A$ tenemos que:
$$\tau_C(f) = \tau_C(f \circ 1_A) = \tau_C(\Hom(A, f)(1_A)) = F f(\tau_A(1_A)) = Ff(\phi_{A, F}(\tau))$$
  De la misma manera se puede obtener que $\tau'_C(f)  = Ff(\phi_{A, F}(\tau'))$, pero entonces para cualquier objeto $C$ y para cualquier flecha $f \in \Hom(A, C)$
  se cumple:
  $$\tau_C(f) = Ff (\phi_{A,F}(\tau)) = Ff(\phi_{A, F}(\tau')) = \tau'_C(f)$$
  con lo que $\tau = \tau'$ y $\phi_{A, F}$ es biyectiva.
  Merece la pena apuntar cuál es la aplicación inversa de $\phi_{A, F}$.
  Unas cuentas sencillas nos permiten comprobar que:
  $$\phi_{A,F}^{-1} : \arr{FA}{\Nat(\Hom(A, -), F)}$$
  $$\phi_{A,F}^{-1}(a)_C(f) = (Ff)(a)$$

  Probar que $\phi_{A, F}$ es natural en $A$ y $F$ es lo que queda para ver
  que $\phi : \nat{L}{Ev}$ es un isomorfismo natural. La demostración es
  rutinaria y no se incluye en el presente texto.
\end{proof}

El lema de Yoneda se puede interpretar en términos de elementos
categóricos. Si tenemos una categoría $\C$, un objeto $A$ de esta
y un funtor $F : \arr{\C}{\Set}$, el lema de Yoneda nos permite
decir que los elementos categóricos de $F$ definidos sobre
el funtor $\Hom(A, -)$ son esencialmente los mismos que los
elementos del conjunto $F A$.

Obtenemos un importante corolario del Lema de Yoneda cuando tomamos
$F = \Hom(B, -)$ con $B$ otro objeto de $\C$.

\begin{theorem}
  Dada una categoría $\C$
  el funtor $T : \arr{C^{op}}{\funccat{\C}{\Set}}$ dado por
  $T(A) = \Hom_\C(A, -)$ es fiel y pleno.
\end{theorem}
\begin{proof}
  $T$ actúa sobre flechas
  $f \in \Hom_{\C^{op}}(A, B)$
  de la siguiente manera:
  $$T(f) = \Hom_\C(f, -) : \nat{T(A) = \Hom(A, -)}{\Hom(B, -) = T(B)}$$
  $$T(f)_C(g) = g \circ f$$

  Repitiendo la demostración del lema de Yoneda con
  $F = \Hom(B, -)$ vemos que la aplicación inducida por
  $T$ sobre los conjuntos $\Hom_{\C^{op}}(A, B)$ es precisamente la
  biyección:
  $$\phi_{A, F}^{-1} : \arr{
    \Hom_{\C^{op}}(A, B) = FA
  }{
    \Nat(\Hom(A, -), F) = \Nat(T(A), T(B))
  }$$
  que definimos anteriormente.
  Luego $T$ es inyectiva y sobreyectiva sobre los conjuntos
  $\Hom$ y
  por tanto es un funtor fiel y pleno.
\end{proof}

Aplicando esta misma proposición sobre la categoría $\C^{op}$ llegamos
al resultado dual:

\begin{theorem}
  Dada una categoría $\C$, el funtor que asigna a cada objeto
  $A$ de la categoría el funtor $\Hom(-, A)$ es un funtor
  $\arr{\C}{\funccat{\C^{op}}{\Set}}$ fiel y pleno.
\end{theorem}

Los funtores descritos en los dos últimos teoremas son conocidos
como \textbf{encajes de Yoneda}.

Los encajes de Yoneda son una herramienta fundamental para la teoría
de categorías. Nos permiten ver la categoría $\C$ dentro de la
categoría de funtores $\funccat{\C^{op}}{\Set}$. En este sentido
la categoría $\funccat{\C^{op}}{\Set}$ se puede ver como una extensión
de la categoría $\C$, identificando cada objeto $A$ de $\C$
con el funtor $\Hom(-, A)$ y cada flecha $f : \arr{A}{B}$ con la
transformación natural asociada
$\Hom(-, f) : \nat{\Hom(-, A)}{\Hom(-, B)}$. El hecho
de que el encaje de
Yoneda sea fiel y pleno nos dice que esta identificación es buena
en el sentido de que las flechas entre objetos $A$ y $B$ de $\C$
son \textit{las mismas} que las flechas (transformaciones naturales)
entre los funtores
$\Hom(-, A)$ y $\Hom(-, B)$

Cuando en el capítulo 2 introdujimos la noción
de elemento categórico subrayamos que dados $x \in^T A$ y
$f : \arr{A}{B}$ entonces $f \circ x \in^T B$.
Que $f$ lleve elementos de $A$ en elementos de $B$
motivó que utilizáramos la notación $f(x)$ en lugar de $f \circ x$.
Esta notación podría llevarnos a interpretar $f$ como una familia
de aplicaciones
$f_T : \arr{\Hom(T, A)}{\Hom(T, B)}$ para objeto $T$ de $\C$,
pero esta familia es precisamente
la imagen de $f$ a través del encaje de Yoneda: $f_T = \Hom(-, f)_T$.
Recíprocamente,
dada cualquier familia de aplicaciones
$g_T : \arr{\Hom(T, A)}{\Hom(T, B)}$ natural en $T$, en el sentido
de que para cualquier $h : \arr{S}{T}$ se dé
$g_T(x) \circ h = g_S(x \circ h)$, tendríamos que $g$ es una transformación
natural $g: \nat{\Hom(-, A)}{\Hom(-, B)}$ y el
encaje de Yoneda nos permitiría realizar la identificación $g : \arr{A}{B}$.

Todo esto nos permite decir que dar una flecha $f: \arr{A}{B}$ en una categoría
arbitraria $\C$ es esencialmente lo mismo que dar una aplicación que lleva
elementos de $A$ a elementos de $B$ de una forma coherente (la condición
de naturalidad).
