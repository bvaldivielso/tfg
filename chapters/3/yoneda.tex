El lema de Yoneda es uno de los primeros resultados
inesperados que surgen a raíz de la teoría de categorías. Una
de las consecuencias más importantes de este resultado es
el \textit{embebimiento} (cambiar) de Yoneda, que nos permite
ver una categoría $\C$ cualquiera dentro de la categoría de
funtores $[\C, \Set]$. Necesitaremos algunos resultados
previos para enunciar el lema:

\begin{proposition*}
  Dada una categoría $\C$ y un objeto suyo
  $A$,
  $$\Nat(\Hom(A, -), -) : \arr{\C \times [\C, \Set]}{\Set}$$
  es un bifuntor donde $\Nat(F, G)$ es el conjunto de transformaciones
  naturales entre los funtores $F$ y $G$.
\end{proposition*}
\begin{proof}
  Consideramos el funtor identidad
  $1_{[\C, \Set]} :\arr{[\C, \Set]}{[\C, \Set]}$ y el funtor
  $\Hom(*, -) : \arr{C}{[\C, \Set]^{op}}$ (demostrar)
  el producto de estos nos define un funtor:
  $$\Hom(*, -)\times 1_{[\C, \Set]} :
  \arr{\C\times [\C, \Set]}{ [\C, \Set]^{op} \times [\C, \Set] }$$
  pero sabemos que
  $$\Nat(-, -) = \Hom_{[\C, \Set]}(-, -) :
    \arr{[\C, \Set]^{op}\times[\C, \Set]}{\Set}$$
    es un funtor. La composición de estos dos funtores
    que hemos dicho es el funtor que buscamos.
\end{proof}


Necesitaremos probar que podemos definir el siguiente funtor:

\begin{proposition*}
  Sea $\C$ una categoría y $Ev(-, -) : \arr{\C\times[\C, \Set]}{\Set}$
  dado por $Ev(C, F) = FC$. $E$ es un funtor y lo llamamos
  \textbf{funtor de evaluación}.
\end{proposition*}
\begin{proof}
  Sean $\sigma : \arr{F}{G}$ y $\tau : \arr{G}{H}$ transformaciones naturales entre funtores
  $F, G, H : \arr{\C}{\Set}$. Sean además $f : \arr{C}{D}$,
  $g : \arr{D}{E}$ flechas de $\C$.
  Podemos definir la acción de $Ev$ sobre las flechas de
  $\arr{\C\times[\C, \Set]}{\Set}$ como
  $Ev(f, \sigma) = \sigma_D \circ Ff : \arr{FC}{GD}$. Veamos que $E$ se comporta bien
  respecto a las composiciones:
  \begin{multline*}
    Ev(g \circ f, \tau\circ\sigma) =
    (\tau \circ \sigma)_E\circ F(g\circ f) =
    \tau_E\circ\sigma_E\circ Fg \circ Ff = \\
  \tau_E\circ Gg \circ\sigma_D \circ Ff =
  \tau_E\circ Gg \circ Ev(f, \sigma) =\\
  Ev(g, \tau) \circ Ev(f, \sigma)
  \end{multline*}

  El hecho de que $Ev(1_C, 1_F) = 1_{FC}$ se prueba de forma sencilla
  y por tanto $Ev : \arr{\C\times[\C, \Set]}{\Set}$ es un funtor

  \textit{Explicar mejor la demostración}
\end{proof}

\begin{theorem}[Lema de Yoneda]
  Sea $\C$ una categoría y $A$ un objeto de esta. Los funtores
  $$(A, F) \mapsto \Nat(\Hom(A, -), F)$$
  $$(A, F) \mapsto FA$$
  son naturalmente isomorfos.
\end{theorem}
\begin{proof}
  Sea $a \in F A$. Definimos la transformación natural entre
  $\lambda_a : \arr{\Hom(A, -)}{F}$ componente a componente de
  la siguiente forma: $(\lambda_a)_C(f) = f(a)$.
  \textit{Continuar mañana}
\end{proof}
