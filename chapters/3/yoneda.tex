El lema de Yoneda es uno de los primeros resultados
inesperados que surgen a raíz de la teoría de categorías. Una
de las consecuencias más importantes de este resultado es
el encaje de Yoneda, que nos permite
ver una categoría $\C$ cualquiera dentro de la categoría de
funtores $\funccat{\C}{\Set}$. Necesitaremos algunos resultados
previos para poder enunciar el lema.

A lo largo de las siguientes
secciones usaremos la notación $\Nat(-, -)$ en lugar de
$\Hom_{\funccat{\C}{\D}}(-, -)$ para referirnos al conjunto de
transformaciones naturales entre dos funtores. Definimos
$L : \arr{\C\times\funccat{\C}{\Set}}{\Set}$ por
$L(A, F) = \Nat(\Hom(A, -), F)$. Probaremos que $L$ es un funtor.

\begin{proposition*}
  $L$ definido anteriormente es un funtor.
\end{proposition*}
\begin{proof}
  Definimos el funtor
  $H: \arr{\C^{op}}{\funccat{\C}{\Set}}$ dado por
  $H(A) = \Hom_\C(A, -)$. Nuestro funtor $L$ es precisamente

  $$L :\C\times\funccat{\C}{\Set}
  \xrightarrow{H^{op}\times 1_{\funccat{\C}{\Set}}}
  (\funccat{C}{\Set})^{op} \times \funccat{C}{\Set}
  \xrightarrow{\Nat} \Set$$
\end{proof}
Proseguimos con otro funtor $\arr{\C\times\funccat{\C}{\Set}}{\Set}$:

\begin{proposition*}
  Sea $\C$ una categoría y $Ev(-, -) : \arr{\C\times\funccat{\C}{\Set}}{\Set}$
  dado por $Ev(C, F) = FC$. $Ev$ es un funtor y lo llamamos
  \textbf{funtor de evaluación}.
\end{proposition*}
\begin{proof}
  Sean $\sigma : \arr{F}{G}$ y $\tau : \arr{G}{H}$ transformaciones naturales entre funtores
  $F, G, H : \arr{\C}{\Set}$. Sean además $f : \arr{C}{D}$,
  $g : \arr{D}{E}$ flechas de $\C$.
  Definimos la acción de $Ev$ sobre las flechas de
  $\arr{\C\times\funccat{\C}{\Set}}{\Set}$ como
  $Ev(f, \sigma) = \sigma_D \circ Ff : \arr{FC}{GD}$. Veamos que $E$ se comporta bien
  respecto a las composiciones:
  \begin{multline*}
    Ev(g \circ f, \tau\circ\sigma) =
    (\tau \circ \sigma)_E\circ F(g\circ f) =\\
    \tau_E\circ\sigma_E\circ Fg \circ Ff =^*
  \tau_E\circ Gg \circ\sigma_D \circ Ff =
  \tau_E\circ Gg \circ Ev(f, \sigma) =\\
  Ev(g, \tau) \circ Ev(f, \sigma)
  \end{multline*}

  Donde $*$ se deduce del hecho de que $\sigma : \nat{F}{G}$ es una transformación
  natural.
  Que $Ev(1_C, 1_F) = 1_{FC}$ se prueba de forma sencilla,
  confirmando que $Ev : \arr{\C\times\funccat{\C}{\Set}}{\Set}$ es un funtor
\end{proof}

Ahora que conocemos estos dos funtores podemos enunciar el lema
de Yoneda de forma sencilla.

\begin{theorem}[Lema de Yoneda]
  Sea $\C$ una categoría. Los funtores $L$ y $Ev$ son naturalmente isomorfos.
  En particular, para cada objeto $A$ de $\C$ y cada funtor $F : \arr{\C}{\Set}$
  se tiene una biyección
  $$\Nat(\Hom(A, -), F) \cong FA$$
\end{theorem}
\begin{proof}
  Definimos la aplicación de conjuntos
  $$\phi_{A, F} : \arr{\Nat(\Hom(A, -), F)}{FA}$$
  $$\phi_{A, F}(\tau) = \tau_A (1_A)$$
  (recordamos que $\tau_A$ por ser $\tau$ transformación natural entre $\Hom(A, -)$ y $F$ es
  una aplicación de conjuntos $\tau_A : \arr{\Hom(A, A)}{F A}$).

  Veamos que $\phi_{A, F}$ es una biyección. En primer lugar veremos que es sobreyectiva. Sea $x \in FA$
  Definimos la transformación natural $\lambda_x : \nat{\Hom(A, -)}{F}$ por
  $$(\lambda_x)_C : \arr{\Hom(A, C)}{F C}$$
  $$(\lambda_x)_C(f) = f(x)$$

  Comprobar que $\lambda_x$ es en efecto una transformación natural es sencillo. Pero
  $\phi_{A, F}(\lambda_x) = (\lambda_x)_A (1_A) = 1_A(x) = x$, luego $\phi_{A, F}$
  es sobreyectiva.

  Veamos ahora que $\phi_{A, F}$ es inyectiva. Supongamos que tenemos
  dos transformaciones naturales $\tau, \tau' : \nat{\Hom(A, -)}{F}$
  tales que $\phi_{A, F}(\tau) = \phi_{A, F}(\tau')$. Para todo $f \in \Hom(A, C)$
  la naturalidad de $\tau$ nos garantiza la conmutatividad
  del siguiente diagrama:

  \begin{center}
    \begin{tikzcd}
      \Hom(A, A) \arrow{d}[swap]{\Hom(A, f)} \arrow{r}{\tau_A}
            & FA \arrow{d}{Ff} \\

      \Hom(A, C) \arrow{r}{\tau_C} & F C
    \end{tikzcd}
  \end{center}


  Es decir $\tau_C \circ \Hom(A, f) = F f \circ \tau_A$. Aplicando
  esta flecha sobre el valor $1_A$ tenemos que:

  $$\tau_C(f) = \tau_C(f \circ 1_A) = \tau_C(\Hom(A, f)(1_A)) = F f(\tau_A(1_A)) = Ff(\phi_{A, F}(\tau))$$
  De la misma manera se puede obtener que $\tau'_C(f)  = Ff(\phi_{A, F}(\tau'))$

  pero entonces para cualquier objeto $C$ y para cualquier flecha $f \in \Hom(A, C)$
  se cumple:

  $$\tau_C(f) = Ff (\phi_{A,F}(\tau)) = Ff(\phi_{A, F}(\tau')) = \tau'_C(f)$$
  con lo que $\tau = \tau'$ y $\phi_{A, F}$ es biyectiva.


  Probar que $\phi_{A, F}$ es natural en $A$ y $F$ es lo que queda para ver
  que $\phi : \nat{L}{Ev}$ es un isomorfismo natural. La demostración es
  rutinaria y no se incluye en el presente texto.
\end{proof}
