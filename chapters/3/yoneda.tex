El lema de Yoneda es uno de los primeros resultados
inesperados que surgen a raíz de la teoría de categorías. Una
de las consecuencias más importantes de este resultado es
el \textit{embebimiento} (cambiar) de Yoneda, que nos permite
ver una categoría $\C$ cualquiera dentro de la categoría de
funtores $[\C, \Set]$. Empezamos enunciando y demostrando el lema:

\begin{theorem}[Lema de Yoneda (versión básica solo con biyecciones, escribir mañana sobre la otra)]
  Sea $\C$ una categoría, $A$ un objeto de esta
  y $F : \arr{\C}{\Set}$ un funtor. El conjunto de transformaciones
  naturales entre $\Hom(A, -)$ y $F$ está en biyección con el
  conjunto $F A$.
\end{theorem}
\begin{proof}
  Sea $a \in F A$. Definimos la transformación natural entre
  $\lambda_a : \arr{\Hom(A, -)}{F}$ componente a componente de
  la siguiente forma: $(\lambda_a)_C(f) = f(a)$.
  \textit{Continuar mañana}
\end{proof}
