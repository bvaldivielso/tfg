Comentamos brevemente algunas funcionalidades de haskell
que facilitarán la comprensión de los ejemplos utilizados.

\section{Expresiones y tipos}
Toda expresión en Haskell tiene un tipo. Los siguientes
son ejemplos de expresiones junto con sus tipos:
\begin{minted}{haskell}
True :: Bool
False :: Bool
'c' :: Char
"hola" :: String
3 :: Int
\end{minted}
En general la sintaxis \cod{a :: A} se utiliza en haskell para
afirmar que la expresión \cod{a} tiene tipo \cod{A}. Que toda
expresión en haskell esté tipada no significa que tengamos que
ser explícito acerca de estos tipos, el compilador es capaz
en la mayoría de las ocasiones de inferir por nosotros cual
es el tipo de todas las expresiones de un programa. Será habitual,
sin embargo, que especifiquemos un tipo aunque no sea necesario,
para ayudar al lector.

\section{Funciones}
En haskell las funciones son ciudadanos de primera clase
(\textit{first class citizens}) en el sentido de que son valores
como cualquier otro. Esto significa, entre otras cosas, que las
funciones tienen un tipo.

Proponemos ejemplos de funciones a continuación:
\begin{minted}{haskell}
suma1 :: Int -> Int
suma1 x = x + 1

tres :: Int
tres = suma 2

esA :: Char -> Bool
esA c = if c == 'A'
        then True
        else False
\end{minted}
El tipo de \cod{suma1} indica que la función recibe un parámetro
de tipo entero y devuelve un valor de tipo entero. Se pueden
definir también funciones de múltiples parámetros:
\begin{minted}{haskell}
suma :: Int -> Int -> Int
suma x y = x + y

cinco :: Int
cinco = suma 3 2
\end{minted}
El tipo de \cod{suma} puede resulta sorprendente
a quien esté acostumbrado a trabajar con lenguajes imperativos.
Haskell solo considera funciones de un parámetro. La trampa es que
\cod{suma} es una función que recibe un parámetro de tipo \cod{Int}
y devuelve \emph{otra función} que recibe un parámetro \cod{Int}
y devuelve un valor de tipo \cod{Int}. El siguiente ejemplo es
equivalente al anterior:
\begin{minted}{haskell}
suma :: Int -> Int -> Int
suma x y = suma x + y

sumaTres :: Int -> Int
sumaTres = suma 3

cinco :: Int
cinco = sumaTres 2
\end{minted}
Entonces cuando haskell evalúa la expresión
\begin{minted}{haskell}
suma 3 2
\end{minted}
lo está haciendo asociado a la izquierda:
\begin{minted}{haskell}
(suma 3) 2
\end{minted}
La técnica de transformar una función que recibe varios parámetros
en una función que recibe un parámetro y devuelve otra función
se llama currificación. Haskell hace currificación automática
de todas las funciones.

De hecho no existe ninguna diferencia entre el tipo de la función
\cod{suma} tal como estaba definido anteriormente y el tipo:
\begin{minted}{haskell}
suma :: Int -> (Int -> Int)
suma x y = x + y
\end{minted}

\section{Expresiones lambda}
Existe un tipo especial de expresión en haskell llamado expresión
lambda y que sirve para definir funciones. Ponemos un ejemplo
a continuación:
\begin{minted}{haskell}
suma1 :: Int -> Int
suma1 = (\x -> x + 1)
\end{minted}
La expresión construye la función que recibe un valor \cod{x}
y devuelve el valor \cod{x+1}. Las expresiones lambda sirven
también para definir funciones que reciben múltiples parámetros:
\begin{minted}{haskell}
multiplica :: Int -> Int -> Int
multiplica = (\x y -> x * y)
\end{minted}

Es importante notar que la expresión \cod{\\x -> f x} es
equivalente a la expresión \cod{f}.

\section{Definir tipos}
Haskell permite al usuario definir sus propios tipos. Veamos
un ejemplo sencillo:
\begin{minted}{haskell}
data Dni = Dni Int Char
\end{minted}
Esta línea define el tipo \cod{Dni} (izquierda) y el constructor
homónimo (derecha) que acepta como parámetros un entero y un carácter.
Mostramos a continuación ejemplos de valores del tipo \cod{Dni}.
\begin{minted}{haskell}
(Dni 44669279 'J') :: Dni
(Dni 12345678 'C') :: Dni
\end{minted}
Todos los valores del tipo \cod{Dni} son de la forma
\cod{(Dni i l)} donde \cod{i :: Int} y \cod{l :: Char}. Además,
podemos considerar al constructor \cod{Dni} una función con tipo
\cod{Dni :: Int -> Char -> Dni}. Nótese que
\cod{Dni} antes de los dos puntos representa al constructor pero
que el \cod{Dni} de la derecha representa al tipo.

Como todos los valores del tipo \cod{Dni} son de la forma especificada
arriba, haskell nos permite definir funciones sobre ese tipo de la
siguiente manera:
\begin{minted}{haskell}
suma3AlNumero :: Dni -> Int
suma3AlNumero (Dni numero caracter) = numero + 3

-- devuelve el dni completo como una cadena
devuelveCadena :: Dni -> String
devuelveCadena (Dni n c) = (show n) ++ (show c)
\end{minted}

Podemos definir tipos con más de un constructor. Por ejemplo:
\begin{minted}{haskell}
data PersonaConAltura = Hombre Int | Mujer Int | Otro String Int
\end{minted}
donde definimos el tipo \cod{PersonaConAltura} y los constructores
\cod{Hombre :: Int -> PersonaConAltura},
\cod{Mujer :: Int -> PersonaConAltura},
\cod{Otro :: String -> Int -> PersonaConAltura}. Todo valor de
\cod{PersonaConAltura} será o bien de la forma
\cod{Hombre n}, o bien \cod{Mujer n}, o bien \cod{Otro o n}, donde
\cod{n :: Int} y \cod{o :: String}. Esto hace que haskell
nos permita definir funciones sobre este tipo como:
\begin{minted}{haskell}
comentarioSobreAltura :: PersonaConAltura -> String
comentarioSobreAltura (Hombre altura) = if altura < 174
                                        then "Tienes altura inferior a la media"
                                        else "Tienes altura superior a la media"
comentarioSobreAltura (Mujer altura) = if altura < 163
                                       then "Tienes altura inferior a la media"
                                       else "Tienes altura superior a la media"
comentarioSobreAltura (Otro genero x) =
            "No tengo informacion sobre la altura media en el género " ++ genero
\end{minted}
Esta funcionalidad de hacer distinción de casos sobre los constructores
de un tipo es conocida como \textit{pattern matching}. También se puede
hacer \textit{pattern matching} dentro de una expresión con la siguiente
sintaxis:
\begin{minted}{haskell}
genero :: PersonaConAltura -> String
genero persona = case persona of
    Hombre _ -> "Hombre"
    Mujer _ -> "Mujer"
    Otro g _ -> g

-- equivalentemente podríamos haber implementado la función como
genero (Hombre _) = "Hombre"
genero (Mujer _) = "Mujer"
genero (Otro g _) = g
\end{minted}
donde usamos el carácter \cod{\_} para indicar (tanto a lectores como
al compilador) que no usaremos esa variable.

Nótese que podemos definir tipos \textit{recursivos} en el siguiente
sentido:
\begin{minted}{haskell}
data Peano = Cero | Succ Peano
\end{minted}
Aquí estamos definiendo el tipo \cod{Peano} que tiene dos constructores:
\cod{Cero :: Peano} (no admite parámetros) y \cod{Succ :: Peano -> Peano}.
Podemos hacer \textit{pattern matching} sobre este tipo:
\begin{minted}{haskell}
aNumero :: Peano -> Int
aNumero Cero = 0
aNumero (Succ peano) = 1 + (aNumero peano)

esCero :: Peano -> Bool
esCero Cero = True
esCero (Succ _) = False
\end{minted}

Existe otra sintaxis para definir tipos con un solo constructor y que reciben
un solo parámetro:
\begin{minted}{haskell}
newtype EnvuelveEntero = EnvuelveEntero Int
\end{minted}
que para los propósitos de este trabajo consideraremos igual a la sintaxis
que usa \cod{data} en lugar de \cod{newtype}.

Cuando usamos la sintaxis:
\begin{minted}{haskell}
data Dni = Dni {
  numero :: Int
  letra :: Char
  }
\end{minted}
No solo estamos definiendo el tipo \cod{Dni} junto con el constructor de tipo
\cod{Dni :: Int -> Char -> Dni} sino que además estamos definiendo las
proyecciones evidente \cod{numero :: Dni -> Int} y \cod{letra :: Dni -> Char}.

\section{Funciones con tipos variables}
Podemos definir funciones polimórficas como la siguiente:
\begin{minted}{haskell}
aplicar :: (a -> b) -> a -> b
aplicar f x = f x

seis :: Int
seis = aplicar suma1 5

hombre :: String
hombre = aplicar genero (Hombre 180)
\end{minted}
La función \cod{aplicar} se puede especializar a tipos concretos como
\cod{a = Int, b = Int} en el caso del valor \cod{seis} o a los
tipos \cod{a = PersonaConAltura, b = String} en el caso del
valor \cod{hombre}.


\section{Tipos Parametrizados}
Podemos definir tipos parametrizados por otros tipos. Por ejemplo:
\begin{minted}{haskell}
data ParConNombre a = ParConNombre String a a
\end{minted}
esto define todos los tipos \cod{ParConNombre A} donde \cod{A} es
un tipo fijo pero arbitrario. Nótese que \cod{ParConNombre} no
es un tipo, pero \cod{ParConNombre Int} sí que lo es.
Un ejemplo de valor de este tipo sería \cod{ParConNombre "tresdos" 3 2 :: ParConNombre Int}.

El constructor \cod{ParConNombre :: String -> a -> a -> ParConNombre a}
es ahora una función con tipos variables. Se puede estar parametrizado por más
de un tipo:
\begin{minted}{haskell}
data Par a b = Par a b

-- ejemplos de valores
Par 3 'c' :: Par Int Char
Par 'c' 3 :: Par Char Int
Par 4 5 :: Par Int Int
\end{minted}

\section{Listas}
Podemos definir listas en haskell de forma similar a como
definimos el tipo \cod{Peano}:
\begin{minted}{haskell}
data Lista a = ListaVacia | NoVacia a Lista

-- ejemplos de valores
ListaVacia :: Lista Int
NoVacia 3 ListaVacia :: Lista Int
NoVacia 3 (NoVacia 4 ListaVacia) :: Lista Int
\end{minted}
y podemos definir funciones sobre este tipo:
\begin{minted}{haskell}
sumarLista :: Lista Int -> Int
sumarLista ListaVacia = 0
sumarLista (NoVacia x otraLista) = x + (sumarLista otraLista)
\end{minted}
No necesitamos definir este tipo por nuestra cuenta
porque ya está definido en la biblioteca estándar de haskell
y cuenta con una sintaxis más intuitiva:
\begin{minted}{haskell}
sumarLista :: [Int] -> Int
sumarLista [] = 0
sumarLista (x:xs) = x + (sumarLista xs)
\end{minted}
Cuando queramos hacer referencia a que \cod{[]} es un tipo
parametrizado a veces escribiremos \cod{[] Int} en lugar
de \cod{[Int]}.

\section{Typeclasses}
Haskell soporta una funcionalidad similar a las interfaces de java
conocida como typeclasses. Una typeclass define una interfaz que
cualquier tipo que quiera ser instancia de esa clase debe implementar.
Por ejemplo:
\begin{minted}{haskell}
data Tamaño = Grande | Pequeño

class TieneTamaño t where
  tamaño :: t -> Tamaño

instance TieneTamaño Int where
  tamaño x = if x > 179 then Grande else Pequeño

instance TieneTamaño String where
  tamaño s = if (length s) > 10 then Grande else Pequeño

instance TieneTamaño PersonaConAltura
  tamaño (Hombre altura) = tamaño altura
  tamaño (Mujer altura) = tamaño altura
  tamaño (Otro _ altura) = tamaño altura
\end{minted}
Nótese como hemos implementado la instancia de \cod{Tamaño} de
\cod{PersonaConAltura} aprovechando la instancia de \cod{Int}.
Podemos escribir funciones que solo acepten instancias de
esta typeclass:
\begin{minted}{haskell}
nombreTamaño :: Tamaño -> String
nombreTamaño Grande = "grande"
nombreTamaño Pequeño = "pequeño"

decirTamaño :: TieneTamaño a => String -> a -> String
decirTamaño nombre a =
  nombre ++ " tiene tamaño " ++ nombreTamaño (tamaño a)

-- ejemplos
decirTamaño "tres" 3 -- tres tiene tamaño pequeño
decirTamaño "Braulio" (Hombre 180) -- Braulio tiene tamaño grande
decirTamaño "cadena de texto" "eeeeeeeeeeee" -- ...
\end{minted}
De hecho, también se pueden definir clases que solo puedan ser
instanciadas por instancias de otras clases:
\begin{minted}{haskell}
class TieneTamaño t => TieneTamañoNumerico t where
  tamañoNumerico :: t -> Int

instance TieneTamañoNumerico Int where
  tamañoNumerico x = x

instance TieneTamañoNumerico Char where
  tamañoNumerico _ = 100 -- Error: Char no es instancia de TieneTamaño
\end{minted}
