\subsection{Definición}
De la misma manera que para los grupos se definen los homomorfismos
de grupos, para los anillos los homomorfismos de anillos y para los
espacios topológicos las funciones continuas, también
podemos asociar a las categorías una noción de morfismos que preserva
su estructura. A estos morfismos de categorías les llamaremos
\emph{funtores}. Un funtor $F$ de una categoría
$\C$ en una categoría $\D$ (que notaremos por $F : \arr{\C}{\D}$)
tendrá que llevar objetos de $\C$ en objetos de $\D$ y flechas
de $\C$ en flechas de $\D$ preservando la estructura de la
categoría en el siguiente sentido:
\begin{enumerate}
\item $F$ respeta los dominios y los codominios:
si $f : \arr{A}{B}$ es una flecha
de la categoría $\mathcal{C}$ entonces
$F f : \arr{F A}{F B}$ es la correspondente flecha asociada
en $\D$.
\item $F$ preserva las identidades. Dicho de otra forma si $C$ es un
objeto de $\mathcal{C}$ entonces $F 1_C = 1_{F C}$.
\item $F$ respeta la composición: si tenemos $f: \arr{A}{B}$ y
$g : \arr{B}{C}$ tenemos que $F (g\circ f) = F g \circ F f$.
\end{enumerate}

Aunque la acción sobre los objetos
no determina por completo a un funtor, habrá ocasiones en las que
el lector podrá completar por sí mismo fácilmente la acción de este
sobre las flechas. En tales casos nos limitaremos a referirnos
al funtor describiendo su acción sobre los objetos.

A lo largo del trabajo mostraremos diagramas en la que los nodos
son objetos de una categoría y las aristas dirigidas
que los unen son flechas.
Diremos que el diagrama es conmutativo cuando para cada par
de objetos $C$ y $C'$ del diagrama la composición
de las flechas que conforman un camino entre $C$ y $C'$ es independiente
del camino utilizado. Un ejemplo sería el siguiente diagrama:

\begin{center}
  \begin{tikzcd}
    FA \arrow{r}{Ff} \arrow{rd}[swap]{F(g \circ f)} & FB \arrow{d}{Fg} \\
    & FC
  \end{tikzcd}
\end{center}

Donde afirmar que el diagrama es conmutativo se traduce en
que $F g \circ F f = F (g\circ f)$. Según McLane en \cite{mclane}

\begin{quotation}
  Una parte considerable de la efectividad de los métodos
  categóricos reside en el hecho de que los diagramas
  conmutativos en representan muy fielmente las acciones de
  las flechas en un contexto determinado.
\end{quotation}


\subsection{Ejemplos}
\paragraph{Funtores Identidad}
Para cada categoría $\C$ podemos definir el funtor identidad
$1_{\C} : \arr{C}{C}$ que deja invariante tanto a los objetos
como a las flechas. Comprobar las propiedades de los funtores
es trivial.

\paragraph{Funtores subyacentes a $\Set$}
Podemos considerar el funtor $U : \arr{\Grp}{\Set}$
que asigna a cada grupo su conjunto subyacente y a cada flecha la
aplicación entre los conjuntos subyacentes (cada homomorfismo de
grupos es también una aplicación entre ambos conjuntos). Es rutinario
comprobar que se respetan los axiomas de funtores. Existen también
funtores subyacentes desde la categoría de anillos, espacios topológicos
o retículos por ejemplo.

\paragraph{Grupos libres}
Podemos definir un funtor $F : \arr{\Set}{\Grp}$ de la siguiente
forma: a cada conjunto $X$ le asignamos el grupo libre sobre $X$
(al que llamaremos $F X$) y a
cada aplicación $f : \arr{X}{Y}$ entre conjuntos le asignamos el
único homomorfismo de grupos $F f: \arr{F X}{F Y}$ que extiende a $f$.
Comprobar que
$F$ es en efecto un funtor es sencillo.

\paragraph{Funtores $\Hom$}
Ya hemos dicho que para cada par de objetos $A, B$ de
una categoría $\C$ tenemos que
$\Hom(A, B)$ es un conjunto. Fijado un objeto $A$ de
$\C$ tenemos que $\Hom(A, -)$ nos permite asociar a cada
objeto $B$ de la categoría $\C$ un conjunto $\Hom(A, B)$.

Veamos que $\Hom(A, -)$ es un funtor
$\Hom(A, -) : \C \rightarrow \Set$. Ya conocemos la acción
sobre los objetos ahora tenemos que encontrar como actúa
el funtor sobre las flechas. Supongamos que tenemos
$f : \arr{B}{C}$ una flecha de $\C$. Definimos
$\Hom(A, f) : \arr{\Hom(A, B)}{\Hom(A, C)}$
($\Hom(A, f)$ es una aplicación entre los conjuntos
$\Hom(A, B)$ y $\Hom(A, C)$ es decir a cada función $\arr{A}{B}$
le asigna una función $\arr{A}{C}$) por
$$\Hom(A, f)(g) = f \circ g $$.

Probamos a modo de ejemplo
que se cumplen los axiomas de los funtores. En primer lugar
supongamos que $g : \arr{C}{D}$ y $h : \arr{D}{E}$ entonces tenemos
que probar
$$\Hom(A, h \circ g) = \Hom(A, h) \circ \Hom(A, g) :
\arr{\Hom(A, B)}{\Hom(A, C)}$$

Para probar tal cosa suponemos $f \in \Hom(A, B)$ y entonces:
\begin{multline*}
\Hom(A, h\circ g)(f) = (h \circ g) \circ f = h \circ (g \circ f) \\
= h \circ \Hom(A, g)(f) = \Hom(A, h)(\Hom(A, g)(f)) \\
= (\Hom(A, h) \circ \Hom(A, g))(f)
\end{multline*}

Y por tanto se comporta bien respecto a la composición. Veamos que se
comporta bien respecto a la identidad. Sea $f \in \Hom(A, B)$ entonces
$$\Hom(A, 1_B)(f) = 1_B \circ f = f$$ y por tanto
$\Hom(A, 1_B) = 1_{\Hom(A, B)}$.

Teniendo en cuenta $\Hom(A, -)$ se comporta bien respecto a la
composición y lleva identidades en identidades concluimos que es
un funtor.

\paragraph{Bifuntores}
Llamamos bifuntor a un funtor de
la forma $F : \arr{\C_1\times\C_2}{\D}$. Un ejemplo de
bifuntor sería $-\times - : \arr{\Set\times\Set}{\Set}$ que
a cada par de conjuntos le asigna su producto cartesiano.

\paragraph{Composición de funtores}
Dados dos funtores $F : \arr{\C}{\D}$ y $G: \arr{\D}{\E}$ podemos
definir $F \circ G : \arr{\C}{\E}$ tal que $(F\circ G)C = F(G(C))$
y $(F\circ G)(f) = F(Gf) : \arr{FGC}{FGC'}$ donde $C, C'$ son objetos
de $\C$ y $f: \arr{C}{C'}$ una flecha de esta categoría. $F\circ G$
es un funtor. Además esta composición de funtores es asociativa
y los funtores identidad se comportan como identidades frente
a esta composición.

\paragraph{Producto de funtores}
Dados dos funtors $F : \arr{\C}{\D}$ y $G: \arr{\C'}{\D'}$
se define el funtor $F\times G: \arr{\C\times\C'}{\D\times\D'}$ de
la manera evidente.

%% \paragraph{Endofuntores en Haskell}
%% Es habitual que en lenguajes de programación fuertemente tipados exista
%% una noción de \emph{constructor de tipo} o \emph{generics}. Uno de los usos más
%% habituales de estos constructores de tipo son los contenedores. En
%% \verb~C++~ por ejemplo tenemos el constructor de tipo \verb~vector~, que no es
%% un tipo en si mismo pero que asigna a cada tipo un nuevo tipo. Por
%% ejemplo al tipo \verb~int~ le asigna el tipo \verb~vector<int>~ que representa
%% un vector con elementos de tipo \verb~int~.

%% En Haskell existen también constructores de tipos. Viéndolo desde el
%% punto de vista de \verb~Hask~ un constructor de tipos es una forma de asignar
%% objetos de la categoría a otros objetos de la categoría. Esto nos hace
%% pensar que un funtor de \verb~Hask~ a \verb~Hask~ (cuando un funtor va de una
%% categoría en sí misma le llamamos endofuntor) será un constructor de
%% tipos que además cumple una serie de propiedades adicionales.

%% Pondremos un ejemplo de endofuntor en \verb~Hask~. De forma similar a
%% \verb~vector~ podemos considerar el constructor de tipos \verb~[]~ de Haskell
%% que a cada tipo (por ejemplo \verb~Int~) le hace corresponder otro tipo
%% (en este caso \verb~[Int]~, una lista de enteros). Nos hace falta ahora
%% definir cómo a una función \verb~f :: a -> b~ se le asigna una función
%% de tipo \verb~[a] -> [b]~ de manera que se cumplan los axiomas
%% de los functores, pero eso es sencillo: extenderemos una función \verb~f~
%% sobre listas aplicando \verb~f~ elemento a elemento. Escribimos en haskell
%% la definición de esta extensión y escribimos algunos ejemplos:

%% \begin{verbatim}
%% map :: (a -> b) -> ([a] -> [b])
%% map f [] = []
%% map f (x:xs) = (f x):(map f xs)


%% add_1 :: Int -> Int
%% add_1 x = x + 1

%% map add_1 [] ; Esto da como resultado []
%% map add_1 [1,2,3,4,5] ; [2,3,4,5,6]

%% ; en general map f [x1, x2,x3...] = [f x1, f x2, f x3, ...]
%% \end{verbatim}

%% Veamos que se cumplen los axiomas de los funtores. En primer lugar
%% veamos que la identidad va a la identidad:

%% \begin{verbatim}
%% map id [] ; esto es []
%% map id [x1, ..., xn]; esto es [id x1, ..., id xn] = [x1, ..., xn]
%% \end{verbatim}

%% por lo que la función \verb~map id~ es la identidad. Veamos que la función
%% además respeta las composiciones:

%% \begin{verbatim}
%% ((map f) . (map g)) [x1, ..., xn] = map f (map g [x1, ..., xn])
%%                                   = map f [g x1, ..., g xn]
%%                                   = [ f (g x1), ..., f (g xn) ]
%%                                   = [ (f . g) x1, ..., (f . g) xn ]
%%                                   = map (f . g) [x1, ..., xn]
%% \end{verbatim}
