\subsection{Categorías}
\paragraph{$\Hask$}
En el contexto del lenguaje de programación Haskell (aunque esta
construcción es análoga en otros lenguajes de programación
fuertemente tipados) se suele hablar de la categoría
$\Hask$  en la que los objetos son los tipos del lenguaje
(por ejemplo \verb~Int~, \verb~String~ o \verb~Double~) y
las flechas son las funciones entre esos tipos. Por ejemplo
la función \verb~length :: String -> Int~ vista en
$\Hask$ sería una flecha
$\texttt{length} \in \Hom_{\Hask}(\texttt{String}, \texttt{Int})$.
Como operador de composición tenemos la composición habitual de
funciones, que en haskell se nota con \texttt{.} (un punto) y se
define de la siguiente forma:
\begin{verbatim}
(.) :: (b -> c) -> (a -> b) -> (a -> c)
(.) f g argumentWithTypeA = f (g argumentWithTypeA)
\end{verbatim}
Además tenemos la función \texttt{id} que nos define las
flechas identidad en $\Hask$:

\begin{verbatim}
id :: a -> a
id x = x
\end{verbatim}
Nótese que aun teniendo esta colección de objetos, de flechas, la
operación de composición (que es asociativa) y las identidades tenemos
que $\Hask$ no es una categoría. Esto se debe
entre otras cosas a algunas
peculiaridades del comportamiento del
valor especial \texttt{undefined} de Haskell. Salvando el uso de
este valor, la teoría de categorías es un modelo ampliamente aceptado
para el estudio de $\Hask$. No presumimos en este trabajo
de que $\Hask$ cumpla bajo toda circunstancia los axiomas de las
categorías pero eso no nos impedirá utilizar la teoría de categorías
para analizar y razonar sobre construcciones hechas sobre Haskell
(siendo conscientes de la imperfección del modelo).
Una justificación de que $\Hask$ es indistinguible de una
categoría restringiéndose a un subconjunto del lenguaje lo encontramos
en \cite{fastandloose}.

A lo largo del trabajo veremos como especializando construcciones
categóricas a $\Hask$ obtendremos aplicaciones de la teoría
de la categorías a la programación.

\paragraph{\texttt{Pipes}}
\textcolor{red}{Hablar de este otro ejemplo de categorías}


\subsection{Funtores}
\paragraph{Endofuntores en $\Hask$}
Llamamos endofuntor a un funtor que va de una categoría $\C$ en sí misma.
Sabiendo esto podemos comenzar a entender en qué consiste un endofuntor
en $\Hask$: una forma de asignar tipos a otros tipos (los objetos
de $\Hask$) y transformar funciones en otras funciones (las flechas
de $\Hask$) de manera que se preserven algunas relaciones.

Es habitual encontrar mecanismos que permiten asignar tipos a otros tipos
en los lenguajes de programación usados hoy día. En \verb~C++~
tenemos como ejemplo concreto \verb~vector~ que a cada tipo
(por ejemplo \verb~int~, un entero)
le asigna otro tipo (\verb~vector<int>~, un vector de enteros). En
general las templates de \texttt{C++} permiten realizar este tipo de
construcciones. En \verb~java~ los Generics cumplen una función similar.

También es habitual en los lenguajes de programación modernos
que las funciones
sean \textit{ciudadanos de primera clase} (\textit{first class citizens}),
es decir, se puede operar sobre las funciones como se opera sobre
cualquier otro valor del lenguaje. En este contexto es natural que
surjan funciones que reciben funciones como parámetro y devuelven
otras funciones. Python es un ejemplo de lenguaje en el que
se encuentran estas \textit{higher order functions} (se usan
tan habitualmente que hasta se incorporó en el
lenguaje sintaxis específica para ellas \cite{decorators}).

En la biblioteca estándar de Haskell existe una \verb~typeclass~
(el mecanismo de polimorfismo de Haskell, que para los propósitos
de este trabajo podemos suponer similar a las interfaces de java)
que sirve para dotar de comportamiento funtorial a los constructores
de tipo que implementemos. La typeclass
se llama, convenientemente, \verb~Functor~ \cite{haskell-functor} y se
define de la siguiente manera:

\begin{verbatim}
class Functor F where
  fmap :: (a -> b) -> (F a -> F b)
\end{verbatim}

Si queremos que nuestro constructor de tipo
sea un \verb~Functor~ tendremos
que implementar sobre él una función llamada \verb~fmap~ que reciba
una función de tipo \verb~a -> b~ y nos devuelva una función
de tipo \verb~F a -> F b~ donde \verb~F~ es nuestro constructor de tipo.

Veremos ejemplos a continuación que aclararán la situación pero si
tuviéramos que trazar un paralelismo con \texttt{C++} podríamos decir
que \verb~vector~ (que sería \verb~F~ del código en Haskell) sería una
instancia de la typeclass \verb~Functor~ si implementáramos una función
llamada \verb~fmap~ que recibe como parámetro una función que va
de un tipo cualquiera \verb~A~ a un tipo cualquiera \verb~B~ y
devuelve una función que va del tipo \verb~vector<A>~ (análogo a
\verb~F a~ en el código en Haskell) a \verb~vector<B>~.

Haskell no comprueba que tu implementación de \texttt{fmap}
verifica los axiomas de los funtores. Esa tarea se delega al
desarrollador. Los axiomas de los funtores en
haskell teniendo en cuenta los
operadores de composición \verb~.~ y la función identidad son:

\begin{verbatim}
; f :: a -> b
; g :: b -> c

fmap (g . f) = (fmap g) . (fmap f) :: F a -> F c

fmap id = id :: F a -> F a
\end{verbatim}

Proponemos algunos ejemplos de instancias de la typeclass
\verb~Functor~
que se encuentran en la biblioteca estándar
de haskell.

\paragraph{Maybe}
La definición de \verb~Maybe~ es la siguiente:

\begin{verbatim}
data Maybe a = Just a | Nothing
\end{verbatim}

Este tipo se usa constantemente en Haskell. Representa el resultado
de computaciones que podrían fallar o podrían no tener solución en
casos concretos. Podemos poner un ejemplo de
utilización de este tipo: \verb~head_safe~, una
función que devuelve el primer elemento de una lista:

\begin{verbatim}
head_safe :: [a] -> Maybe a
head_safe [] = Nothing
head_safe (x:xs) = Just x
\end{verbatim}

Decidimos que el tipo de retorno de \verb~head_safe~ sea \verb~Maybe a~ puesto
que este cómputo puede no tener solución en caso de que la lista no
tenga elementos. En otros lenguajes la función \verb~head~ lanzaría una
excepción si se le pasara una lista vacía, pero en Haskell se puede
codificar la naturaleza propensa a fallos
del resultado en el sistema de tipos.

Resulta que se puede dotar al constructor de tipos \verb~Maybe~ de
un comportamiento funtorial. Mostramos a continuación su
implementación de la typeclass \verb~Functor~

\begin{verbatim}
instance Functor Maybe where
  fmap f (Just x) = Just (f x)
  fmap f Nothing = Nothing
\end{verbatim}

Lo que hace esta función \verb~fmap~ es extender funciones de tipo
\verb~a -> b~ a una función de tipo \verb~Maybe a -> Maybe b~. Esta función
no hace nada si recibe un \verb~Nothing~ (representante del fallo
en el cómputo) y aplica la función al contenido del valor \verb~Just x~.

Podemos comprobar que se cumplen los axiomas de los funtores.

\begin{verbatim}
; veamos fmap id = id :: Maybe a -> Maybe a
; fmap id x = id x = x
; si x = (Just y) entonces
(fmap id) x = fmap id (Just y) = Just (id y) = (Just y) = x

; si x = Nothing
(fmap id) Nothing = Nothing

; veamos que se comporta bien con la composición.
; una vez más supongamos que x = (Just y)
(fmap (f . g)) x = fmap (f . g) (Just y)
                 = Just ( (f . g) y )
                 = Just (f (g y))
                 = (fmap f) (Just (g y))
                 = ( (fmap f) . (fmap g) ) (Just y)
                 = ( (fmap f) . (fmap g) ) x

; si x = Nothing
(fmap (f . g)) x = (fmap (f . g)) Nothing
                 = Nothing
                 = (fmap f) Nothing
                 = (fmap f) ((fmap g) Nothing)
                 = ((fmap f) . (fmap g)) Nothing
\end{verbatim}

\paragraph{Either}
La definición de \verb~Either~ es la siguiente:

\begin{verbatim}
data Either a b = Left a | Right b
\end{verbatim}

El tipo \verb~Either~ en haskell se usa para representar cómputos que pueden
devolver valores de dos tipos distintos. Un ejemplo muy habitual
de \verb~Either~ es representar cómputos que, al igual que \verb~Maybe~,
podrían ser erróneos, pero dando detalles sobre el error en caso
de error.

Proponemos un ejemplo artificial que aun así muestra para qué se
podría usar este tipo. Supongamos que tenemos un sistema con usuarios
registrados y en nuestra empresa queremos premiar la fidelidad
de nuestros usuarios en edad laboral. Supongamos además que tenemos
dos tipos de premio: uno para adultos jóvenes y otro para el resto
de personas en edad laboral.

Utilizando \verb~Maybe~ para resolver el problema nos quedaría un código
de la siguiente forma:

\begin{verbatim}
data PremiosTrabajadores = PremiosJovenes | PremiosMayores

dar_premio :: Int -> Maybe PremiosAdultos
dar_premio age
  | age < 16 = Nothing
  | 16 <= age < 40 = Just PremiosJovenes
  | 40 <= age < 65 = Just PremiosMayores
  | 65 <= age = Nothing
\end{verbatim}

Este código cumple su propósito de decirnos qué premio
le corresponde al usuario en caso de que efectivamente le
toque un premio. Sin embargo, lo que no devuelve la función
es el motivo por el que el cliente no es elegible para este.
Para conseguir que la función devuelva ese tipo de información
podemos usar \verb~Either~.

\begin{verbatim}
data PremiosTrabajadores = PremiosJovenes | PremiosMayores

dar_premio :: Int -> Either String PremiosAdultos
dar_premio edad
  | edad < 16 = Left "Demasiado joven para estar en edad laboral"
  | 16 <= edad < 40 = Right PremiosJovenes
  | 40 <= edad < 65 = Right PremiosMayores
  | 65 <= edad = Left "Demasiado mayor para estar en edad laboral"
\end{verbatim}

¿Es \verb~Either~ un funtor? La respuesta es que no, porque de entrada
\verb~Either~ es un constructor de tipo con dos parametros y para que
un constructor de tipo sea un funtor necesitamos que solo tenga
un parámetro. Entonces \verb~Either~ no es un funtor, pero resulta que
\verb~Either a~ donde \verb~a~ es algún (cualquier) tipo fijo de Haskell sí
es un funtor. Es decir si consideramos fijo el primer tipo
\verb~(Either a)~ es un constructor de tipos que admite un tipo
como parámetro y además se puede implementar una instancia de
\verb~Functor~ sobre él de la siguiente forma:

\begin{verbatim}
instance Functor (Either a) where
  fmap f (Left x) = Left x
  fmap f (Right x) = Right (f x)
\end{verbatim}

Esta instancia de \verb~Functor~ es similar a la de \verb~Maybe~: si el valor
es de los de \emph{error} no se hace nada con él. Si es de los valores
\emph{buenos} se transforma mediante la función \verb~f~. Veamos que efectivamente
esta instancia de \verb~Functor~ cumple con las leyes:

\begin{verbatim}
; la identidad va a la identidad:
; supongamos x = (Left y)

fmap id x = fmap id (Left y) = Left y = x

; supongamos x = Right y
fmap id x = fmap id (Right y) = Right (id y) = Right y = x

; probemos ahora que se lleva bien con la composición
fmap (f . g) (Left y) = Left y = (fmap f) (Left y)
                      = (fmap f) (fmap g (Left y))
                      = (fmap f) . (fmap g) (Left y)

fmap (f . g) (Right y) = Right ( (f . g) y )
                       = fmap f (Right (g y))
                       = (fmap f . fmap g) (Right y)
\end{verbatim}

Veamos un ejemplo de utilización de la instancia de \verb~Functor~ de
\verb~Either a~ siguiendo con el ejemplo que utilizamos antes. Imaginemos
que tenemos una función que asocia los distintos premios a sus títulos.
Por ejemplo:

\begin{verbatim}
titulos_premios :: PremiosTrabajadores -> String
titulos_premios PremioJovenes = "Semana de senderismo"
titulos_premios PremioMayores = "Cata de Vinos"
\end{verbatim}

Entonces si quisiéramos una función que a partir de la edad de
un usuario nos devolviera qué mensaje mostrarle en la interfaz
con respecto al premio podríamos hacer lo siguiente:

\begin{verbatim}
mensaje_premio :: Int -> String
mensaje_premio edad =
  case resultado of
    (Left mensajeError) -> "Error: " ++ mensajeError
    (Right tituloPremio) -> "Enhorabuena has conseguido una " ++ tituloPremio

  where
    resultado :: Either String String
    resultado = fmap titulos_premios (dar_premio edad)
\end{verbatim}

\paragraph{List}
\paragraph{Reader}
Definimos \texttt{Reader} de la siguiente forma:
\begin{verbatim}
data Reader a b = Reader (a -> b)
\end{verbatim}

Esta definición quiere decir
que un valor de \textit{tipo} \texttt{Reader a b} es
de la forma \texttt{Reader g} donde \texttt{g} es
una función que recibe un valor de tipo \texttt{a} como
parámetro y devuelve un valor de tipo \texttt{b}. De la misma manera
que hicimos con \texttt{Either} podemos fijar la primera variable
de tipo e implementar una instancia de \texttt{Functor} para
\texttt{(Reader a)}:

\begin{verbatim}
instance Functor (Reader a) where
  ; fmap :: (b -> c) -> (Reader a b) -> (Reader a c)
  fmap f (Reader g) = Reader (f . g)
\end{verbatim}

Podemos probar que esta instancia de \texttt{Functor} cumple
las leyes de los funtores:

\begin{verbatim}
; f :: b -> c
; g :: c -> d
; a  = (Reader h) :: (Reader a b) y entonces
; h :: a -> b

fmap (g . f) a = fmap (g. f) (Reader h)
               = Reader ( (g . f) . h)
               = Reader ( g . (f . h))
               = fmap g (Reader (f . h))
               = fmap g (fmap f (Reader h))
               = (fmap g . fmap f) (Reader h)
               = (fmap g . fmap f) a

; y por tanto fmap (g . f) = fmap g . fmap f
; en las mismas condiciones:

fmap id a = fmap id (Reader h)
          = Reader (id . h)
          = Reader h
          = a

;; y por tanto fmap id = id
\end{verbatim}

Veremos más adelante en el trabajo las aplicaciones
de \texttt{Reader}. Creemos también importante resaltar las similitudes
entre \texttt{Reader} y los funtores $\Hom$ descritos en los ejemplos
matemáticos de funtores. Formalizaremos esta relación
más adelante en el trabajo
cuando hablemos de objetos $\Hom$ internos.
