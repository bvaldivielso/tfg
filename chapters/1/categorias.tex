\subsection{Definición}
Tradicionalmente las matemáticas están fundamentadas en una teoría
de conjuntos. Cuando partimos de una teoría de conjuntos no hace
falta (\textit{no se puede}) definir \textit{qué} es un conjunto. Ocurre de
forma similar con los conceptos de \emph{elemento} y \emph{pertenece},
que son básicos en la teoría. La teoría de categorías se puede utilizar
también para fundamentar las matemáticas, y en este sentido no se
podrían dar definiciones, en términos de otros conceptos,
de nociones como \emph{categoría, objeto, flecha o composición}.
Siguiendo una línea de trabajo similar podríamos decir que tenemos
una categoría si:
\begin{itemize}
\item Conocemos sus \textit{objetos},
  que denotamos $A,B,C,\ldots.$
\item Conocemos sus \textit{flechas},
  que denotamos $f,g,h,\ldots.$
\item Para cada flecha $f$ conocemos su dominio $A$ y su codominio $B$
  , que serán objetos de $\mathcal{C}$ y que notaremos por
  $f:A\rightarrow B$ o bien $A\xrightarrow{f} B$.
\item Para cada dos \emph{flechas componibles}
  $A\xrightarrow{f} B\xrightarrow{g} C$
  conocemos su composición
  $g\circ f :A\rightarrow C$.
\end{itemize}
Todos estos datos, que determinan una categoría $\mathcal{C}$, tienen que cumplir las siguientes propiedades o axiomas:
\begin{enumerate}
\item La composición es asociativa, en el siguiente sentido: si $f : A \longrightarrow B, g: B \longrightarrow C$ y $h : C \longrightarrow D$ se ha de cumplir
  $(h \circ g) \circ f = h \circ (g \circ f)$.
\item Existen identidades, esto es: para cada objecto $C$ existe una
  flecha, a la que llamaremos identidad en $C$ y que denotaremos $1_C : C \longrightarrow C$, que cumple que
  para cualquiera flechas $f: X \longrightarrow C$ y  $g : C \longrightarrow Y$ se tiene
  $1_C \circ f = f$  y  $g \circ 1_C = g$.
\end{enumerate}

Con esta aproximación inicial a la teoría de categorías y con el
suficiente esfuerzo se puede evitar hacer uso de la teoría de conjuntos.
En vista de que este no es un trabajo que pretenda tratar sobre la
fundamentación de las matemáticas no seguiremos un enfoque tan estricto
y consideraremos que para cada par de objetos $A$ y $B$ de la categoría
$\C$ las flechas entre $A$ y $B$ forman un conjunto, al que llamaremos
$\Hom_\C(A, B)$ o simplemente $\Hom(A, B)$ cuando esté claro a que
categoría $\C$ nos referimos. Por otro lado, y en vista de
esta notación, la operación de composición induce una aplicación
$\circ : \arr{\Hom(B, C) \times \Hom(A, B)}{\Hom(A, C)}$ para cada
terna de objetos $A, B$ y $C$. Las categorías en las que se puede
hablar de los conjuntos $\Hom(A, B)$ son conocidas en la literatura
como categorías \emph{localmente pequeñas} y serán las únicas
categorías consideradas a lo largo de este trabajo. Denotaremos por
$\Obj{\C}$ y $\Arr{\C}$ a la \emph{clase} de todos los objetos
y a la clase de todas las flechas de la categoría $\C$ respectivamente,
sin profundizar
en la noción precisa de clase utilizada por estar fuera
de nuestro objetivo.

Mostramos a continuación algunos ejemplos de categorías.

\subsubsection{Ejemplos}
\paragraph{Conjuntos}
Uno de los más típicos ejemplos de categorías es $\Set$,
la categoría de los conjuntos. En esta categoría
cada conjunto es un objeto y cada
aplicación $f$ entre el conjunto $A$ y
el conjunto $B$ es una flecha $f : \arr{A}{B}$. La composición
es la composición habitual de aplicaciones y las identidades
$1_C : C \longrightarrow C$ son las aplicaciones identidad en
cada conjunto $C$.
Comprobar que se cumplen los axiomas de las categorías es una tarea
rutinaria. No se puede asumir en general que los
objetos de una categoría
forman un conjunto por ejemplos como este:
no tiene sentido hablar del conjunto de todos los conjuntos.

\paragraph{Otras estructuras matemáticas}
Gran parte de las estructuras que se estudian en matemáticas forman
categorías si consideramos sus morfismos como flechas.
Podemos dar multitud de ejemplos de este tipo.
\begin{itemize}
\item \texttt{Grp}: la categoría en la que los objetos son grupos
  y las flechas son los homomorfismos de grupos.
\item \texttt{Top}: la categoría en la que los objetos son espacios
  topológicos y las flechas son funciones continuas.
\item \texttt{Ring}: la categoría en la que los objetos son
  anillos y las flechas son homomorfismos de anillos.
\end{itemize}
La lista sigue y sigue.
\paragraph{Monoides}
Proponemos este ejemplo para evitar la asumción de que en una categoría
los objetos deben ser estructuras matemáticas y las flechas entre ellos
aplicaciones que preservan la estructura. Definimos una categoría
con un solo objeto al que llamaremos $*$. El conjunto de flechas
será $Hom(*, *) = \mathbb{Z}$ y
$\circ : Hom(*, *)\times Hom(*, *)\rightarrow Hom(*, *)$ quedará definido
por $f \circ g = f + g$ donde la suma es la habitual de los enteros.

Es trivial ver que los axiomas se cumplen:
\begin{enumerate}
\item La composición es asociativa: dadas $n, m, k : * \longrightarrow *$
(el único tipo de flechas que se puede componer, el único tipo de
flechas que hay) sabemos que
$n \circ (m \circ k) = n + (m + k) = (n + m) + k = (n \circ m) \circ k$.

\item Existe la identidad para cada objeto: solo existe un objeto y a su
identidad la llamaremos 0. Es trivial ver que $f \circ 0 = f$ y que
$0 \circ g = g$ en este contexto.
\end{enumerate}
En general esta construcción que acabamos de aplicar a $(\mathbb{Z}, +)$
se puede utilizar con cualquier monoide. Toda categoría con un solo
objeto se puede interpretar como un monoide (y viceversa):
la asociatividad de la
composición garantiza la asociatividad de la operación monoidal y
la existencia de la flecha identidad garantiza la existencia del
elemento neutro del monoide. En este sentido podemos considerar que
las categorías son una generalización de los monoides.

\subsection{Categoría Producto}
Dado un par de categorías $\C$ y $\D$ podemos construir la categoría
producto $\C\times\D$ de la siguiente forma:
\begin{itemize}
\item los objetos serán de la forma $(C, D)$ donde $C$ es un objeto
  de $\C$ y $D$ es un objeto de $\D$;
\item las flechas serán de la forma $(f, g) : \arr{(C, D)}{(C', D')}$,
  donde $f: \arr{C}{C'}$ es una flecha de $\C$ y $g : \arr{D}{D'}$
  es una flecha de $\D$;
\item la composición actúa componente a componente
  $(f, g) \circ (f', g') = (f \circ f', g \circ g')$, siempre
  que $f$ y $f'$, y $g$ y $g'$ se puedan componer.
\end{itemize}

Las identidad del objeto $(C, D)$ es claramente la flecha
$(1_C, 1_D)$. Es trivial comprobar que $\C\times\D$ cumple
los axiomas de una categoría.
