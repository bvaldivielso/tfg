\subsection{Definición}

Una categoría $\mathcal{C}$ queda determinada por dos colecciones:

\begin{enumerate}
\item $\Obj{\C}$ a cuyos elementos nos referiremos como
  \emph{objetos} de $\mathcal{C}$.
\item $\Arr{\C}$ a la que nos referiremos como las
  \emph{flechas} de $\mathcal{C}$. A cada flecha se le puede asignar
  un par de objetos: al primero de ellos le llamaremos \emph{origen}
  (o dominio) y al otro \emph{destino} (o \emph{codominio}). Con la notación
  $f : \arr{A}{B}$ estaremos afirmando que $f$ es una flecha
  que tiene como origen el objeto $A$ y como destino el objeto $B$.
  No supondremos en general que $\mathcal{A}r(\mathcal{C})$ es un
  conjunto pero a lo largo de este texto sí que asumiremos que
  $Hom(A, B) = \{ f: f : A \longrightarrow B \}$ lo es. A las categorías
  en las que $Hom(A, B)$ son conjuntos se les suele llamar categorías
  localmente pequeñas. En este texto solo trataremos categorías
  localmente pequeñas.
\end{enumerate}
Para poder hablar de categorías necesitamos también una operación de composición $\circ$ que funcione de la siguiente
forma:

\begin{enumerate}
\item Para cada terna de objetos $A, B, C$ de la categoría $\mathcal{C}$
  diremos que los pares de flechas de $Hom(B, C)\times Hom(A, B)$ se
  pueden componer y tendremos que
  $\circ : Hom(B, C)\times Hom(A, B) \rightarrow Hom(A, C)$. Dicho
  de otra forma si $f: A\longrightarrow B$ y $g : B \longrightarrow C$
  tenemos que $g \circ f : A \longrightarrow C$.
\item $\circ$ es asociativa en el siguiente sentido: si
  $f : A \longrightarrow B, g: B \longrightarrow C,$
  $h : C \longrightarrow D$ tenemos que
  $(h \circ g) \circ f = h \circ (g \circ f)$.
\item Para cada objecto $C$ de la categoría $\mathcal{C}$ existe una
  flecha a la que llamaremos $1_C : C \longrightarrow C$ tal que
  para cualquier flecha $f: X \longrightarrow C$ tenemos que
  $1_C \circ f = f$ y para cualquier flecha $g : C \longrightarrow Y$
  se cumple $g \circ 1_C = g$ para cualquiera que sean los objetos
  $X, Y$ de $\mathcal{C}$.
\end{enumerate}

\textcolor{red}{Voy a reescribirte esto, creo que la forma en que se introduce una categoría da una idea de la escuela que se sigue o de la filosofía con que se afronta esta teoría. Te voy a introducir en categorías como lo haría yo. TU ELIGES LA FORMA DEFINITIVA. No haré esto con el resto del trabajo pero el comienzo es importante.}

Tradicionalmente las matemáticas están fundamentadas en  \emph{una teoría de conjuntos} y bajo esta fundamentación el concepto de \emph{conjunto} es básico. Entendemos lo que es un conjunto pero no tratamos de dar una definición formal de este. Ocurre lo mismo con los conceptos de \emph{elemento} y \emph{pertenece} que son básicos en esta teoría.  En la actualidad se puede utilizar la \emph{teoría de categorías} para fundamentar las matemáticas y en este sentido los conceptos de \emph{categoría, objeto, flecha} y \emph{composición} serían los conceptos básicos que se intentan entender sin dar una definición formal de estos. Siguiendo esta idea podemos decir que tendremos una categoría $\mathcal{C}$ si:
\begin{itemize}
	\item Conocemos sus objetos, que denotamos $A,B,C,\ldots.$
	\item Conocemos sus flechas, que denotamos $f,g,h,\ldots.$
	\item Para cada flecha $f$ conocemos su dominio $A$ y su codominio $B$, que serán objetos de $\mathcal{C}$, escribiremos $f:A\rightarrow B$ o bien $A\xrightarrow{f} B$.
	\item Para cada dos \emph{flechas componibles} $A\xrightarrow{f} B\xrightarrow{g} C$ conocemos su composición $g\circ f :A\rightarrow C$.
\end{itemize}
Todos estos datos, que determinan una categoría $\mathcal{C}$, tienen que cumplir las siguientes propiedades o axiomas:
\begin{enumerate}
	\item La composición es asociativa, en el siguiente sentido: si $f : A \longrightarrow B, g: B \longrightarrow C$ y $h : C \longrightarrow D$ se ha de cumplir
	$(h \circ g) \circ f = h \circ (g \circ f)$.
	\item Existen identidades, esto es: para cada objecto $C$ existe una
	flecha, a la que llamaremos identidad en $C$ y que denotaremos $1_C : C \longrightarrow C$, que cumple que
	para cualquiera flechas $f: X \longrightarrow C$ y  $g : C \longrightarrow Y$ se tiene
	$1_C \circ f = f$  y  $g \circ 1_C = g$.
\end{enumerate}

Con esta aproximación a la teoría de categorías no haríamos uso de la teoría de conjuntos. Sin embargo vamos a optar por una aproximación \emph{no tan categórica}. En toda esta memoria vamos a asumir que para cada par de objetos $A$ y $B$, de la categoría $\mathcal{C}$, las flechas de $\mathcal{C}$ con dominio $A$ y codominio $B$ forman un conjunto, que denotaremos $Hom_{\mathcal{C}}(A,B)$ o simplemente $Hom(A,B)$. De manera que la composición en $\mathcal{C}$ determina aplicaciones $\circ : Hom(B, C)\times Hom(A, B) \rightarrow Hom(A, C)$ para cada terna de objetos $A,B,C$ de $\mathcal{C}$. En este sentido, en esta memoria trataremos sólo con categorías \emph{localmente pequeñas}. Denotaremos $\Obj{\C}$ y $\Arr{\C}$ a las \emph{clases} de todos los objetos y todas las flechas respectivamente de $\C$.




Mostramos a continuación algunos ejemplos de categorías.

\subsubsection{Ejemplos}
\paragraph{Conjuntos}
Uno de los más típicos ejemplos de categorías es $\Set$,
la categoría de los conjuntos. En esta categoría
cada conjunto es un objeto (no se puede asumir que
$\mathcal{O}b(\mathcal{C})$ es un conjunto por ejemplos como este:
no tiene sentido hablar del conjunto de todos los conjuntos) y cada
aplicación $f$ entre conjuntos con dominio el conjunto $A$ y como codominio
el conjunto $B$ es una flecha $f : A \longrightarrow B$. La composición
es la composición habitual de aplicaciones y las identidades
$1_C : C \longrightarrow C$ son las aplicaciones identidad en
cada conjunto $C$.
Comprobar que se cumplen los axiomas de las categorías es una tarea
rutinaria.

\paragraph{Otras estructuras matemáticas}
Gran parte de las estructuras que se estudian en matemáticas forman
categorías si consideramos sus morfismos como flechas.
Podemos dar multitud de ejemplos de este tipo:

\begin{itemize}
\item \texttt{Grp}: la categoría en la que los objetos son grupos
  y las flechas son los homomorfismos de grupos.
\item \texttt{Top}: la categoría en la que los objetos son espacios
  topológicos y las flechas son funciones continuas.
\item \texttt{Ring}: la categoría en la que los objetos son
  anillos y las flechas son homomorfismos de anillos.
\end{itemize}
La lista sigue y sigue.

\paragraph{Monoides}
Proponemos este ejemplo para evitar la asumción de que en una categoría
los objetos deben ser estructuras matemáticas y las flechas entre ellos
aplicaciones que preservan la estructura. Definimos una categoría
con un solo objeto al que llamaremos $*$. El conjunto de flechas
será $Hom(*, *) = \mathbb{Z}$ y
$\circ : Hom(*, *)\times Hom(*, *)\rightarrow Hom(*, *)$ quedará definido
por $f \circ g = f + g$ donde la suma es la habitual de los enteros.

Es trivial ver que los axiomas se cumplen:
\begin{enumerate}
\item La composición es asociativa: dadas $n, m, k : * \longrightarrow *$
(el único tipo de flechas que se puede componer, el único tipo de
flechas que hay) sabemos que
$n \circ (m \circ k) = n + (m + k) = (n + m) + k = (n \circ m) \circ k$.

\item Existe la identidad para cada objeto: solo existe un objeto y a su
identidad la llamaremos 0. Es trivial ver que $f \circ 0 = f$ y que
$0 \circ g = g$ en este contexto.
\end{enumerate}
En general esta construcción que acabamos de aplicar a $(\mathbb{Z}, +)$
se puede aplicar a cualquier monoide. Toda categoría con un solo
objeto se puede interpretar como un monoide (y viceversa):
la asociatividad de la
composición garantiza la asociatividad de la operación monoidal y
la existencia de la flecha identidad garantiza la existencia del
elemento neutro del monoide. En este sentido podemos considerar que
las categorías son una generalización de los monoides.

\subsection{Categoría Producto}
Dado un par de categorías $\C$ y $\D$ podemos construir la categoría
producto $\C\times\D$ de la siguiente forma:
\begin{itemize}
\item Los objetos serán de la forma $(C, D)$ donde $C$ es un objeto
  de $\C$ y $D$ es un objeto de $\D$.
\item Las flechas serán de la forma $(f, g) : \arr{(C, D)}{(C', D')}$
  donde $f: \arr{C}{C'}$ es una flecha de $\C$ y $g : \arr{D}{D'}$
  es una flecha de $\D$.
\item La composición actúa componente a componente
  $(f, g) \circ (f', g') = (f \circ f', g \circ g')$ siempre
  que $f$ y $f'$, y $g$ y $g'$ se puedan componer.
\end{itemize}

Las identidad del objeto $(C, D)$ es claramente la flecha
$(1_C, 1_D)$. Es trivial comprobar que $\C\times\D$ cumple
los axiomas de una categoría.
