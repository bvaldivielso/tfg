\documentclass[12pt]{report}

\usepackage[utf8]{inputenc}
\usepackage{graphicx}
\usepackage{amsmath}
\usepackage{amsfonts}
\usepackage{amsthm} % unnumbered environments


\usepackage{tikz-cd} % diagramas conmutativos




\usepackage[hidelinks]{hyperref} % para hiperreferencias

\usepackage{color} %para el color

% pies de página y esas cosas
\usepackage{fancyhdr}
\pagestyle{fancy}

% cambiar chapter por capítulo
\renewcommand{\chaptername}{Capítulo}

% algunas macros que usaré frecuentemente
% categorías genéricas
\newcommand{\C}[0]{\mathcal{C}}
\newcommand{\D}[0]{\mathcal{D}}
\newcommand{\E}[0]{\mathcal{E}}
\newcommand{\Obj}[1]{\mathcal{O}b(#1)}
\newcommand{\Arr}[1]{\mathcal{A}r(#1)}
\newcommand{\arr}[2]{#1\longrightarrow #2}
\newcommand{\nat}[2]{#1\Rightarrow #2}

\DeclareMathOperator{\Hom}{Hom}
\DeclareMathOperator{\Nat}{Nat}

% macros para nombres de categorías
\newcommand{\Set}[0]{\texttt{Set}}
\newcommand{\Grp}[0]{\texttt{Grp}}
\newcommand{\Ring}[0]{\texttt{Ring}}
\newcommand{\Top}[0]{\texttt{Top}}
\newcommand{\Hask}[0]{\texttt{Hask}}
\newcommand{\VectK}[0]{\texttt{Vect-}K}

% algebra
\newcommand{\Z}[0]{\mathbb{Z}}
\newcommand{\Q}[0]{\mathbb{Q}}

% short code snippets
\newcommand{\cod}[1]{\texttt{#1}}

\newcommand{\uno}{*}
\newcommand{\final}{*}
\newcommand{\inicial}{\oslash}
\newcommand{\nulo}{\textbf{0}}

\newcommand{\funccat}[2]{#2^{#1}}


% bibliografía
\usepackage{biblatex}
\addbibresource{references.bib}

\setcounter{secnumdepth}{3} % que numere las subsecciones también

\graphicspath{ {images/} }

% ambientes: teoremas, definiciones
\newtheorem{definition}{Definición}
\newtheorem*{definition*}{Definición}
\newtheorem{proposition}{Proposición}
\newtheorem*{proposition*}{Proposición}
\newtheorem{theorem}{Teorema}


\title{
	{Patrones de diseño inspirados por teoría de categorías}\\
	{\large Universidad de Granada}\\
}
\author{Braulio Valdivielso Martínez}
\date{\today}

\begin{document}
\maketitle

\chapter*{Abstract}
Abstract goes here

\chapter*{Dedication}
To mum and dad

\chapter*{Declaration}
I declare that..

\chapter*{Acknowledgements}
I want to thank...

\tableofcontents

\chapter{Categorías y funtores}
\section{Categorías}
\subsection{Definición}
Tradicionalmente las matemáticas están fundamentadas en una teoría
de conjuntos. Cuando partimos de una teoría de conjuntos no hace
falta (\textit{no se puede}) definir \textit{qué} es un conjunto. Ocurre de
forma similar con los conceptos de \emph{elemento} y \emph{pertenece},
que son básicos en la teoría. La teoría de categorías se puede utilizar
también para fundamentar las matemáticas, y en este sentido no se
podrían dar definiciones, en términos de otros conceptos,
de nociones como \emph{categoría, objeto, flecha o composición}.
Siguiendo una línea de trabajo similar podríamos decir que tenemos
una categoría si:
\begin{itemize}
\item Conocemos sus \textit{objetos},
  que denotamos $A,B,C,\ldots.$
\item Conocemos sus \textit{flechas},
  que denotamos $f,g,h,\ldots.$
\item Para cada flecha $f$ conocemos su dominio $A$ y su codominio $B$
  , que serán objetos de $\mathcal{C}$ y que notaremos por
  $f:A\rightarrow B$ o bien $A\xrightarrow{f} B$.
\item Para cada dos \emph{flechas componibles}
  $A\xrightarrow{f} B\xrightarrow{g} C$
  conocemos su composición
  $g\circ f :A\rightarrow C$.
\end{itemize}
Todos estos datos, que determinan una categoría $\mathcal{C}$, tienen que cumplir las siguientes propiedades o axiomas:
\begin{enumerate}
\item La composición es asociativa, en el siguiente sentido: si $f : A \longrightarrow B, g: B \longrightarrow C$ y $h : C \longrightarrow D$ se ha de cumplir
  $(h \circ g) \circ f = h \circ (g \circ f)$.
\item Existen identidades, esto es: para cada objecto $C$ existe una
  flecha, a la que llamaremos identidad en $C$ y que denotaremos $1_C : C \longrightarrow C$, que cumple que
  para cualquiera flechas $f: X \longrightarrow C$ y  $g : C \longrightarrow Y$ se tiene
  $1_C \circ f = f$  y  $g \circ 1_C = g$.
\end{enumerate}

Con esta aproximación inicial a la teoría de categorías y con el
suficiente esfuerzo se puede evitar hacer uso de la teoría de conjuntos.
En vista de que este no es un trabajo que pretenda tratar sobre la
fundamentación de las matemáticas no seguiremos un enfoque tan estricto
y consideraremos que para cada par de objetos $A$ y $B$ de la categoría
$\C$ las flechas entre $A$ y $B$ forman un conjunto, al que llamaremos
$\Hom_\C(A, B)$ o simplemente $\Hom(A, B)$ cuando esté claro a que
categoría $\C$ nos referimos. Por otro lado, y en vista de
esta notación, la operación de composición induce una aplicación
$\circ : \arr{\Hom(B, C) \times \Hom(A, B)}{\Hom(A, C)}$ para cada
terna de objetos $A, B$ y $C$. Las categorías en las que se puede
hablar de los conjuntos $\Hom(A, B)$ son conocidas en la literatura
como categorías \emph{localmente pequeñas} y serán las únicas
categorías consideradas a lo largo de este trabajo. Denotaremos por
$\Obj{\C}$ y $\Arr{\C}$ a la \emph{clase} de todos los objetos
y a la clase de todas las flechas de la categoría $\C$ respectivamente,
sin profundizar
en la noción precisa de clase utilizada por estar fuera
de nuestro objetivo.

Mostramos a continuación algunos ejemplos de categorías.

\subsubsection{Ejemplos}
\paragraph{Conjuntos}
Uno de los más típicos ejemplos de categorías es $\Set$,
la categoría de los conjuntos. En esta categoría
cada conjunto es un objeto y cada
aplicación $f$ entre el conjunto $A$ y
el conjunto $B$ es una flecha $f : \arr{A}{B}$. La composición
es la composición habitual de aplicaciones y las identidades
$1_C : C \longrightarrow C$ son las aplicaciones identidad en
cada conjunto $C$.
Comprobar que se cumplen los axiomas de las categorías es una tarea
rutinaria. No se puede asumir en general que los
objetos de una categoría
forman un conjunto por ejemplos como este:
no tiene sentido hablar del conjunto de todos los conjuntos.

\paragraph{Otras estructuras matemáticas}
Gran parte de las estructuras que se estudian en matemáticas forman
categorías si consideramos sus morfismos como flechas.
Podemos dar multitud de ejemplos de este tipo.
\begin{itemize}
\item \texttt{Grp}: la categoría en la que los objetos son grupos
  y las flechas son los homomorfismos de grupos.
\item \texttt{Top}: la categoría en la que los objetos son espacios
  topológicos y las flechas son funciones continuas.
\item \texttt{Ring}: la categoría en la que los objetos son
  anillos y las flechas son homomorfismos de anillos.
\end{itemize}
La lista sigue y sigue.
\paragraph{Monoides}
Proponemos este ejemplo para evitar la asumción de que en una categoría
los objetos deben ser estructuras matemáticas y las flechas entre ellos
aplicaciones que preservan la estructura. Definimos una categoría
con un solo objeto al que llamaremos $*$. El conjunto de flechas
será $Hom(*, *) = \mathbb{Z}$ y
$\circ : Hom(*, *)\times Hom(*, *)\rightarrow Hom(*, *)$ quedará definido
por $f \circ g = f + g$ donde la suma es la habitual de los enteros.

Es trivial ver que los axiomas se cumplen:
\begin{enumerate}
\item La composición es asociativa: dadas $n, m, k : * \longrightarrow *$
(el único tipo de flechas que se puede componer, el único tipo de
flechas que hay) sabemos que
$n \circ (m \circ k) = n + (m + k) = (n + m) + k = (n \circ m) \circ k$.

\item Existe la identidad para cada objeto: solo existe un objeto y a su
identidad la llamaremos 0. Es trivial ver que $f \circ 0 = f$ y que
$0 \circ g = g$ en este contexto.
\end{enumerate}
En general esta construcción que acabamos de aplicar a $(\mathbb{Z}, +)$
se puede utilizar con cualquier monoide. Toda categoría con un solo
objeto se puede interpretar como un monoide (y viceversa):
la asociatividad de la
composición garantiza la asociatividad de la operación monoidal y
la existencia de la flecha identidad garantiza la existencia del
elemento neutro del monoide. En este sentido podemos considerar que
las categorías son una generalización de los monoides.

\subsection{Categoría Producto}
Dado un par de categorías $\C$ y $\D$ podemos construir la categoría
producto $\C\times\D$ de la siguiente forma:
\begin{itemize}
\item los objetos serán de la forma $(C, D)$ donde $C$ es un objeto
  de $\C$ y $D$ es un objeto de $\D$;
\item las flechas serán de la forma $(f, g) : \arr{(C, D)}{(C', D')}$,
  donde $f: \arr{C}{C'}$ es una flecha de $\C$ y $g : \arr{D}{D'}$
  es una flecha de $\D$;
\item la composición actúa componente a componente
  $(f, g) \circ (f', g') = (f \circ f', g \circ g')$, siempre
  que $f$ y $f'$, y $g$ y $g'$ se puedan componer.
\end{itemize}

Las identidad del objeto $(C, D)$ es claramente la flecha
$(1_C, 1_D)$. Es trivial comprobar que $\C\times\D$ cumple
los axiomas de una categoría.

\section{Funtores}
\subsection{Definición}
De la misma manera que para los grupos se definen los homomorfismos
de grupos, para los anillos los homomorfismos de anillos y para los
espacios topológicos las funciones continuas, también
podemos asociar a las categorías una noción de morfismos que preserva
su estructura. A estos morfismos de categorías les llamaremos
\emph{funtores}. Un funtor $F$ de una categoría
$\C$ en una categoría $\D$ (que notaremos por $F : \arr{\C}{\D}$)
tendrá que llevar objetos de $\C$ en objetos de $\D$ y flechas
de $\C$ en flechas de $\D$ preservando la estructura de la
categoría en el siguiente sentido:
\begin{enumerate}
\item $F$ respeta los dominios y los codominios:
si $f : \arr{A}{B}$ una flecha
de la categoría $\mathcal{C}$ entonces
$F f : \arr{F A}{F B}$ es la correspondente flecha asociada
en $\D$.
\item $F$ preserva las identidades. Dicho de otra forma si $C$ es un
objeto de $\mathcal{C}$ entonces $F 1_C = 1_{F C}$.
\item $F$ respeta la composición: si tenemos $f: \arr{A}{B}$ y
$g : \arr{B}{C}$ tenemos que $F (g\circ f) = F g \circ F f$.
\end{enumerate}

Aunque la acción sobre los objetos
no determina por completo a un funtor, habrá ocasiones en las que
el lector podrá completar por sí mismo fácilmente la acción de este
sobre las flechas. En tales casos nos limitaremos a referirnos
al funtor describiendo su acción sobre los objetos.

La ley de la composición se puede interpretar como que $F$ lleva
diagramas conmutativos a diagramas conmutativos. \textcolor{red}{para hacer este comentario aquí tendrías que decir lo que  es un diagrama conmutativo ?¿?¿ Sí, diré qué es un diagrama conmutativo (antes de esta parte para que se entienda el comentario).}


\subsection{Ejemplos}
\paragraph{Funtores Identidad}
Para cada categoría $\C$ podemos definir el funtor identidad
$1_{\C} : \arr{C}{C}$ que deja invariante tanto a los objetos
como a las flechas. Comprobar las propiedades de los funtores
es trivial.

\paragraph{Funtores subyacentes \textcolor{red}{a $\Set$ ???}}
Podemos considerar el funtor $U : \arr{\Grp}{\Set}$
que asigna a cada grupo su conjunto subyacente y a cada flecha la
aplicación entre los conjuntos subyacentes (cada homomorfismo de
grupos es también una aplicación entre ambos conjuntos). Es rutinario
comprobar que se respetan los axiomas de funtores. Existen también
funtores subyacentes desde la categoría de anillos, espacios topológicos
o retículos por ejemplo.

\paragraph{Grupos libres}
Podemos definir un funtor $F : \arr{\Set}{\Grp}$ de la siguiente
forma: a cada conjunto $X$ le asignamos el grupo libre sobre $X$
(al que llamaremos $F X$) y a
cada aplicación $f : \arr{X}{Y}$ entre conjuntos le asignamos el
único homomorfismo de grupos $F f: \arr{F X}{F Y}$ que extiende a $f$.
Comprobar que
$F$ es en efecto un funtor es sencillo.

\paragraph{Funtores $\Hom$}
Ya hemos dicho que para cada par de objetos $A, B$ de
una categoría $\C$ tenemos que
$\Hom(A, B)$ es un conjunto. Fijado un objeto $A$ de
$\C$ tenemos que $\Hom(A, -)$ nos permite asociar a cada
objeto $B$ de la categoría $\C$ un conjunto $\Hom(A, B)$.

Veamos que $\Hom(A, -)$ es un funtor
$\Hom(A, -) : \C \rightarrow \Set$. Ya conocemos la acción
sobre los objetos ahora tenemos que encontrar como actúa
el funtor sobre las flechas. Supongamos que tenemos
$f : \arr{B}{C}$ una flecha de $\C$. Definimos
$\Hom(A, f) : \arr{\Hom(A, B)}{\Hom(A, C)}$
($\Hom(A, f)$ es una aplicación entre los conjuntos
$\Hom(A, B)$ y $\Hom(A, C)$ es decir a cada función $\arr{A}{B}$
le asigna una función $\arr{A}{C}$) por
$$\Hom(A, f)(g) = f \circ g $$.

Probamos a modo de ejemplo
que se cumplen los axiomas de los funtores. En primer lugar
supongamos que $g : \arr{C}{D}$ y $h : \arr{D}{E}$ entonces tenemos
que probar
$$\Hom(A, h \circ g) = \Hom(A, h) \circ \Hom(A, g) :
\arr{\Hom(A, B)}{\Hom(A, C)}$$

Para probar tal cosa suponemos $f \in \Hom(A, B)$ y entonces:
\begin{multline*}
\Hom(A, h\circ g)(f) = (h \circ g) \circ f = h \circ (g \circ f) \\
= h \circ \Hom(A, g)(f) = \Hom(A, h)(\Hom(A, g)(f)) \\
= (\Hom(A, h) \circ \Hom(A, g))(f)
\end{multline*}

Y por tanto se comporta bien respecto a la composición. Veamos que se
comporta bien respecto a la identidad. Sea $f \in \Hom(A, B)$ entonces
$$\Hom(A, 1_B)(f) = 1_B \circ f = f$$ y por tanto
$\Hom(A, 1_B) = 1_{\Hom(A, B)}$.

Teniendo en cuenta $\Hom(A, -)$ se comporta bien respecto a la
composición y lleva identidades en identidades concluimos que es
un funtor.

\paragraph{Bifuntores}
Llamamos bifuntor a un funtor de
la forma $F : \arr{\C_1\times\C_2}{\D}$. Un ejemplo de
bifuntor sería $-\times - : \arr{\Set\times\Set}{\Set}$ que
a cada par de conjuntos le asigna su producto cartesiano.

\paragraph{Composición de funtores}
Dados dos funtores $F : \arr{\C}{\D}$ y $G: \arr{\D}{\E}$ podemos
definir $F \circ G : \arr{\C}{\E}$ tal que $(F\circ G)C = F(G(C))$
y $(F\circ G)(f) = F(Gf) : \arr{FGC}{FGC'}$ donde $C, C'$ son objetos
de $\C$ y $f: \arr{C}{C'}$ una flecha de esta categoría. $F\circ G$
es un funtor. Además esta composición de funtores es asociativa
y los funtores identidad se comportan como identidades frente
a esta composición.

\paragraph{Producto de funtores}
\textit{Hablar de esto}

%% \paragraph{Endofuntores en Haskell}
%% Es habitual que en lenguajes de programación fuertemente tipados exista
%% una noción de \emph{constructor de tipo} o \emph{generics}. Uno de los usos más
%% habituales de estos constructores de tipo son los contenedores. En
%% \verb~C++~ por ejemplo tenemos el constructor de tipo \verb~vector~, que no es
%% un tipo en si mismo pero que asigna a cada tipo un nuevo tipo. Por
%% ejemplo al tipo \verb~int~ le asigna el tipo \verb~vector<int>~ que representa
%% un vector con elementos de tipo \verb~int~.

%% En Haskell existen también constructores de tipos. Viéndolo desde el
%% punto de vista de \verb~Hask~ un constructor de tipos es una forma de asignar
%% objetos de la categoría a otros objetos de la categoría. Esto nos hace
%% pensar que un funtor de \verb~Hask~ a \verb~Hask~ (cuando un funtor va de una
%% categoría en sí misma le llamamos endofuntor) será un constructor de
%% tipos que además cumple una serie de propiedades adicionales.

%% Pondremos un ejemplo de endofuntor en \verb~Hask~. De forma similar a
%% \verb~vector~ podemos considerar el constructor de tipos \verb~[]~ de Haskell
%% que a cada tipo (por ejemplo \verb~Int~) le hace corresponder otro tipo
%% (en este caso \verb~[Int]~, una lista de enteros). Nos hace falta ahora
%% definir cómo a una función \verb~f :: a -> b~ se le asigna una función
%% de tipo \verb~[a] -> [b]~ de manera que se cumplan los axiomas
%% de los functores, pero eso es sencillo: extenderemos una función \verb~f~
%% sobre listas aplicando \verb~f~ elemento a elemento. Escribimos en haskell
%% la definición de esta extensión y escribimos algunos ejemplos:

%% \begin{verbatim}
%% map :: (a -> b) -> ([a] -> [b])
%% map f [] = []
%% map f (x:xs) = (f x):(map f xs)


%% add_1 :: Int -> Int
%% add_1 x = x + 1

%% map add_1 [] ; Esto da como resultado []
%% map add_1 [1,2,3,4,5] ; [2,3,4,5,6]

%% ; en general map f [x1, x2,x3...] = [f x1, f x2, f x3, ...]
%% \end{verbatim}

%% Veamos que se cumplen los axiomas de los funtores. En primer lugar
%% veamos que la identidad va a la identidad:

%% \begin{verbatim}
%% map id [] ; esto es []
%% map id [x1, ..., xn]; esto es [id x1, ..., id xn] = [x1, ..., xn]
%% \end{verbatim}

%% por lo que la función \verb~map id~ es la identidad. Veamos que la función
%% además respeta las composiciones:

%% \begin{verbatim}
%% ((map f) . (map g)) [x1, ..., xn] = map f (map g [x1, ..., xn])
%%                                   = map f [g x1, ..., g xn]
%%                                   = [ f (g x1), ..., f (g xn) ]
%%                                   = [ (f . g) x1, ..., (f . g) xn ]
%%                                   = map (f . g) [x1, ..., xn]
%% \end{verbatim}

\section{En programación}
\subsection{Categorías}
\paragraph{$\Hask$}
En el contexto del lenguaje de programación Haskell (aunque esta
construcción es análoga en otros lenguajes de programación
fuertemente tipados) se suele hablar de la categoría
$\Hask$  en la que los objetos son los tipos del lenguaje
(por ejemplo \verb~Int~, \verb~String~ o \verb~Double~) y
las flechas son las funciones entre esos tipos. Por ejemplo
la función \verb~length :: String -> Int~ vista en
$\Hask$ sería una flecha
$\texttt{length} \in \Hom_{\Hask}(\texttt{String}, \texttt{Int})$.
Como operador de composición tenemos la composición habitual de
funciones, que en haskell se nota con \texttt{.} (un punto) y se
define de la siguiente forma:
\begin{verbatim}
(.) :: (b -> c) -> (a -> b) -> (a -> c)
(.) f g argumentWithTypeA = f (g argumentWithTypeA)
\end{verbatim}
Además tenemos la función \texttt{id} que nos define las
flechas identidad en $\Hask$:

\begin{verbatim}
id :: a -> a
id x = x
\end{verbatim}
Nótese que aun teniendo esta colección de objetos, de flechas, la
operación de composición (que es asociativa) y las identidades tenemos
que $\Hask$ no es una categoría. Esto se debe
entre otras cosas a algunas
peculiaridades del comportamiento del
valor especial \texttt{undefined} de Haskell. Salvando el uso de
este valor, la teoría de categorías es un modelo ampliamente aceptado
para el estudio de $\Hask$. No presumimos en este trabajo
de que $\Hask$ cumpla bajo toda circunstancia los axiomas de las
categorías pero eso no nos impedirá utilizar la teoría de categorías
para analizar y razonar sobre construcciones hechas sobre Haskell
(siendo conscientes de la imperfección del modelo).
Una justificación de que $\Hask$ es indistinguible de una
categoría restringiéndose a un subconjunto del lenguaje lo encontramos
en \cite{fastandloose}.

A lo largo del trabajo veremos como especializando construcciones
categóricas a $\Hask$ obtendremos aplicaciones de la teoría
de la categorías a la programación.

\paragraph{\texttt{Pipes}}
\textcolor{red}{Hablar de este otro ejemplo de categorías}


\subsection{Funtores}
\paragraph{Endofuntores en $\Hask$}
Llamamos endofuntor a un funtor que va de una categoría $\C$ en sí misma.
Sabiendo esto podemos comenzar a entender en qué consiste un endofuntor
en $\Hask$: una forma de asignar tipos a otros tipos (los objetos
de $\Hask$) y transformar funciones en otras funciones (las flechas
de $\Hask$) de manera que se preserven algunas relaciones.

Es habitual encontrar mecanismos que permiten asignar tipos a otros tipos
en los lenguajes de programación usados hoy día. En \verb~C++~
tenemos como ejemplo concreto \verb~vector~ que a cada tipo
(por ejemplo \verb~int~, un entero)
le asigna otro tipo (\verb~vector<int>~, un vector de enteros). En
general las templates de \texttt{C++} permiten realizar este tipo de
construcciones. En \verb~java~ los Generics cumplen una función similar.

También es habitual en los lenguajes de programación modernos
que las funciones
sean \textit{ciudadanos de primera clase} (\textit{first class citizens}),
es decir, se puede operar sobre las funciones como se opera sobre
cualquier otro valor del lenguaje. En este contexto es natural que
surjan funciones que reciben funciones como parámetro y devuelven
otras funciones. Python es un ejemplo de lenguaje en el que
se encuentran estas \textit{higher order functions} (se usan
tan habitualmente que hasta se incorporó en el
lenguaje sintaxis específica para ellas \cite{decorators}).

En la biblioteca estándar de Haskell existe una \verb~typeclass~
(el mecanismo de polimorfismo de Haskell, que para los propósitos
de este trabajo podemos suponer similar a las interfaces de java)
que sirve para dotar de comportamiento funtorial a los constructores
de tipo que implementemos. La typeclass
se llama, convenientemente, \verb~Functor~ \cite{haskell-functor} y se
define de la siguiente manera:

\begin{verbatim}
class Functor F where
  fmap :: (a -> b) -> (F a -> F b)
\end{verbatim}

Si queremos que nuestro constructor de tipo
sea un \verb~Functor~ tendremos
que implementar sobre él una función llamada \verb~fmap~ que reciba
una función de tipo \verb~a -> b~ y nos devuelva una función
de tipo \verb~F a -> F b~ donde \verb~F~ es nuestro constructor de tipo.

Veremos ejemplos a continuación que aclararán la situación pero si
tuviéramos que trazar un paralelismo con \texttt{C++} podríamos decir
que \verb~vector~ (que sería \verb~F~ del código en Haskell) sería una
instancia de la typeclass \verb~Functor~ si implementáramos una función
llamada \verb~fmap~ que recibe como parámetro una función que va
de un tipo cualquiera \verb~A~ a un tipo cualquiera \verb~B~ y
devuelve una función que va del tipo \verb~vector<A>~ (análogo a
\verb~F a~ en el código en Haskell) a \verb~vector<B>~.

Haskell no comprueba que tu implementación de \texttt{fmap}
verifica los axiomas de los funtores. Esa tarea se delega al
desarrollador. Los axiomas de los funtores en
haskell teniendo en cuenta los
operadores de composición \verb~.~ y la función identidad son:

\begin{verbatim}
; f :: a -> b
; g :: b -> c

fmap (g . f) = (fmap g) . (fmap f) :: F a -> F c

fmap id = id :: F a -> F a
\end{verbatim}

Proponemos algunos ejemplos de instancias de la typeclass
\verb~Functor~
que se encuentran en la biblioteca estándar
de haskell.

\paragraph{Maybe}
La definición de \verb~Maybe~ es la siguiente:

\begin{verbatim}
data Maybe a = Just a | Nothing
\end{verbatim}

Este tipo se usa constantemente en Haskell. Representa el resultado
de computaciones que podrían fallar o podrían no tener solución en
casos concretos. Podemos poner un ejemplo de
utilización de este tipo: \verb~head_safe~, una
función que devuelve el primer elemento de una lista:

\begin{verbatim}
head_safe :: [a] -> Maybe a
head_safe [] = Nothing
head_safe (x:xs) = Just x
\end{verbatim}

Decidimos que el tipo de retorno de \verb~head_safe~ sea \verb~Maybe a~ puesto
que este cómputo puede no tener solución en caso de que la lista no
tenga elementos. En otros lenguajes la función \verb~head~ lanzaría una
excepción si se le pasara una lista vacía, pero en Haskell se puede
codificar la naturaleza propensa a fallos
del resultado en el sistema de tipos.

Resulta que se puede dotar al constructor de tipos \verb~Maybe~ de
un comportamiento funtorial. Mostramos a continuación su
implementación de la typeclass \verb~Functor~

\begin{verbatim}
instance Functor Maybe where
  fmap f (Just x) = Just (f x)
  fmap f Nothing = Nothing
\end{verbatim}

Lo que hace esta función \verb~fmap~ es extender funciones de tipo
\verb~a -> b~ a una función de tipo \verb~Maybe a -> Maybe b~. Esta función
no hace nada si recibe un \verb~Nothing~ (representante del fallo
en el cómputo) y aplica la función al contenido del valor \verb~Just x~.

Podemos comprobar que se cumplen los axiomas de los funtores.

\begin{verbatim}
; veamos fmap id = id :: Maybe a -> Maybe a
; fmap id x = id x = x
; si x = (Just y) entonces
(fmap id) x = fmap id (Just y) = Just (id y) = (Just y) = x

; si x = Nothing
(fmap id) Nothing = Nothing

; veamos que se comporta bien con la composición.
; una vez más supongamos que x = (Just y)
(fmap (f . g)) x = fmap (f . g) (Just y)
                 = Just ( (f . g) y )
                 = Just (f (g y))
                 = (fmap f) (Just (g y))
                 = ( (fmap f) . (fmap g) ) (Just y)
                 = ( (fmap f) . (fmap g) ) x

; si x = Nothing
(fmap (f . g)) x = (fmap (f . g)) Nothing
                 = Nothing
                 = (fmap f) Nothing
                 = (fmap f) ((fmap g) Nothing)
                 = ((fmap f) . (fmap g)) Nothing
\end{verbatim}

\paragraph{Either}
La definición de \verb~Either~ es la siguiente:

\begin{verbatim}
data Either a b = Left a | Right b
\end{verbatim}

El tipo \verb~Either~ en haskell se usa para representar cómputos que pueden
devolver valores de dos tipos distintos. Un ejemplo muy habitual
de \verb~Either~ es representar cómputos que, al igual que \verb~Maybe~,
podrían ser erróneos, pero dando detalles sobre el error en caso
de error.

Proponemos un ejemplo artificial que aun así muestra para qué se
podría usar este tipo. Supongamos que tenemos un sistema con usuarios
registrados y en nuestra empresa queremos premiar la fidelidad
de nuestros usuarios en edad laboral. Supongamos además que tenemos
dos tipos de premio: uno para adultos jóvenes y otro para el resto
de personas en edad laboral.

Utilizando \verb~Maybe~ para resolver el problema nos quedaría un código
de la siguiente forma:

\begin{verbatim}
data PremiosTrabajadores = PremiosJovenes | PremiosMayores

dar_premio :: Int -> Maybe PremiosAdultos
dar_premio age
  | age < 16 = Nothing
  | 16 <= age < 40 = Just PremiosJovenes
  | 40 <= age < 65 = Just PremiosMayores
  | 65 <= age = Nothing
\end{verbatim}

Este código cumple su propósito de decirnos qué premio
le corresponde al usuario en caso de que efectivamente le
toque un premio. Sin embargo, lo que no devuelve la función
es el motivo por el que el cliente no es elegible para este.
Para conseguir que la función devuelva ese tipo de información
podemos usar \verb~Either~.

\begin{verbatim}
data PremiosTrabajadores = PremiosJovenes | PremiosMayores

dar_premio :: Int -> Either String PremiosAdultos
dar_premio edad
  | edad < 16 = Left "Demasiado joven para estar en edad laboral"
  | 16 <= edad < 40 = Right PremiosJovenes
  | 40 <= edad < 65 = Right PremiosMayores
  | 65 <= edad = Left "Demasiado mayor para estar en edad laboral"
\end{verbatim}

¿Es \verb~Either~ un funtor? La respuesta es que no, porque de entrada
\verb~Either~ es un constructor de tipo con dos parametros y para que
un constructor de tipo sea un funtor necesitamos que solo tenga
un parámetro. Entonces \verb~Either~ no es un funtor, pero resulta que
\verb~Either a~ donde \verb~a~ es algún (cualquier) tipo fijo de Haskell sí
es un funtor. Es decir si consideramos fijo el primer tipo
\verb~(Either a)~ es un constructor de tipos que admite un tipo
como parámetro y además se puede implementar una instancia de
\verb~Functor~ sobre él de la siguiente forma:

\begin{verbatim}
instance Functor (Either a) where
  fmap f (Left x) = Left x
  fmap f (Right x) = Right (f x)
\end{verbatim}

Esta instancia de \verb~Functor~ es similar a la de \verb~Maybe~: si el valor
es de los de \emph{error} no se hace nada con él. Si es de los valores
\emph{buenos} se transforma mediante la función \verb~f~. Veamos que efectivamente
esta instancia de \verb~Functor~ cumple con las leyes:

\begin{verbatim}
; la identidad va a la identidad:
; supongamos x = (Left y)

fmap id x = fmap id (Left y) = Left y = x

; supongamos x = Right y
fmap id x = fmap id (Right y) = Right (id y) = Right y = x

; probemos ahora que se lleva bien con la composición
fmap (f . g) (Left y) = Left y = (fmap f) (Left y)
                      = (fmap f) (fmap g (Left y))
                      = (fmap f) . (fmap g) (Left y)

fmap (f . g) (Right y) = Right ( (f . g) y )
                       = fmap f (Right (g y))
                       = (fmap f . fmap g) (Right y)
\end{verbatim}

Veamos un ejemplo de utilización de la instancia de \verb~Functor~ de
\verb~Either a~ siguiendo con el ejemplo que utilizamos antes. Imaginemos
que tenemos una función que asocia los distintos premios a sus títulos.
Por ejemplo:

\begin{verbatim}
titulos_premios :: PremiosTrabajadores -> String
titulos_premios PremioJovenes = "Semana de senderismo"
titulos_premios PremioMayores = "Cata de Vinos"
\end{verbatim}

Entonces si quisiéramos una función que a partir de la edad de
un usuario nos devolviera qué mensaje mostrarle en la interfaz
con respecto al premio podríamos hacer lo siguiente:

\begin{verbatim}
mensaje_premio :: Int -> String
mensaje_premio edad =
  case resultado of
    (Left mensajeError) -> "Error: " ++ mensajeError
    (Right tituloPremio) -> "Enhorabuena has conseguido una " ++ tituloPremio

  where
    resultado :: Either String String
    resultado = fmap titulos_premios (dar_premio edad)
\end{verbatim}

\paragraph{List}
\paragraph{Reader}
Definimos \texttt{Reader} de la siguiente forma:
\begin{verbatim}
data Reader a b = Reader (a -> b)
\end{verbatim}

Esta definición quiere decir
que un valor de \textit{tipo} \texttt{Reader a b} es
de la forma \texttt{Reader g} donde \texttt{g} es
una función que recibe un valor de tipo \texttt{a} como
parámetro y devuelve un valor de tipo \texttt{b}. De la misma manera
que hicimos con \texttt{Either} podemos fijar la primera variable
de tipo e implementar una instancia de \texttt{Functor} para
\texttt{(Reader a)}:

\begin{verbatim}
instance Functor (Reader a) where
  ; fmap :: (b -> c) -> (Reader a b) -> (Reader a c)
  fmap f (Reader g) = Reader (f . g)
\end{verbatim}

Podemos probar que esta instancia de \texttt{Functor} cumple
las leyes de los funtores:

\begin{verbatim}
; f :: b -> c
; g :: c -> d
; a  = (Reader h) :: (Reader a b) y entonces
; h :: a -> b

fmap (g . f) a = fmap (g. f) (Reader h)
               = Reader ( (g . f) . h)
               = Reader ( g . (f . h))
               = fmap g (Reader (f . h))
               = fmap g (fmap f (Reader h))
               = (fmap g . fmap f) (Reader h)
               = (fmap g . fmap f) a

; y por tanto fmap (g . f) = fmap g . fmap f
; en las mismas condiciones:

fmap id a = fmap id (Reader h)
          = Reader (id . h)
          = Reader h
          = a

;; y por tanto fmap id = id
\end{verbatim}

Veremos más adelante en el trabajo las aplicaciones
de \texttt{Reader}. Creemos también importante resaltar las similitudes
entre \texttt{Reader} y los funtores $\Hom$ descritos en los ejemplos
matemáticos de funtores. Formalizaremos esta relación
más adelante en el trabajo
cuando hablemos de objetos $\Hom$ internos.


\chapter{Construcciones elementales}
Dedicamos este capítulo al tratamiento de construcciones elementales
sobre categorías. En este capítulo podremos ver cómo un marco tan general
como el de la teoría de categorías permite realizar definiciones
\textit{Continuar introducción de construcciones elementales luego}.

\section{Elementos}
Cuando hablamos de conjuntos es común hablar de sus elementos. Muchas
definiciones que se hacen sobre conjuntos y aplicaciones entre
ellos se hacen en base a los elementos de los conjuntos involucrados.
Dos ejemplos de este tipo de definiciones serían las definiciones de
aplicación inyectiva y de producto cartesiano.

\begin{definition*}
Una aplicación $f : \arr{A}{B}$ es inyectiva si dados
$a, a' \in A$ tenemos que $f(a) = f(a') \implies a = a'$.
\end{definition*}

\begin{definition*}
Dados dos conjuntos $A$ y $B$ definimos su producto cartesiano
como:
$$A\times B = \{ (a, b) : a \in A \quad b \in B \}$$
\end{definition*}

Si intentamos trasladar las construcciones que hacemos sobre los
conjuntos y sus aplicaciones a categorías arbitrarias con sus objetos
y sus flechas tenemos que saber cómo trasladar la noción de elemento.

Si consideramos el conjunto con un solo elemento, al que llamaremos
$*$, y las aplicaciones que salen de él nos damos cuentas de que podemos
identificar las aplicaciones entre el conjunto $* = \{ 1 \}$ \textcolor{red}{es curioso, parece que aquí tenemos una visión dual del asunto. A mi me gusta escribir $\uno=\{*\}$ mientras que tu prefieres $*=\{1\}$ fíjate que he usado una macro para uno para que puedas usar el símbolo que quieras, aunque en conjuntos uno=1 es bastante bueno. Yo prefiero mi notación a la tuya pero ????? supongo que tienes tus motivos} y un conjunto
$A$ (estas aplicaciones son de la forma $1 \mapsto a \in A$)
arbitrario con los elementos (en el sentido de teoría de conjuntos)
de $A$. Dicho de otra forma $\Hom(*, A)$ y $A$ son ``\emph{lo mismo}"  como conjuntos.
pero $\Hom(*, A)$ está formado por flechas y eso es algo de lo que
sí podemos hablar dentro de una categoría. Sin embargo, para realizar
este procedimiento de identificar los elementos de un conjunto con
un conjunto de flechas de la categoría hemos tenido que acudir a un
objeto especial de la categoría de conjuntos: $*$. Este procedimiento
es algo que quizá no podamos realizar en otras categorías pero nos
motiva a hacer una definición más general de
\textit{elemento categórico}:

\begin{definition}
En una categoría $\C$ llamaremos elemento de un objeto $A$
a cualquier flecha $x : \arr{T}{A}$ (sea cual sea el objeto
$T$). De forma paralela a como se hace con conjuntos, utilizaremos
la notación $x \in^T A$.
\end{definition}

Con esta noción de elemento notamos que $f : \arr{A}{B}$ lleva
elementos de $A$ a elementos de $B$ mediante la composición: si
$x \in^T A$ entonces $f \circ x \in^T B$. Esto motiva que en lo que
sigue omitamos a veces el signo de composición ($f x$ en lugar
de $f \circ x$) si buscamos que se interpreten algunas flechas
como elementos categóricos y la acción de otras flechas sobre estos.

Veamos como esta noción nos permite llevar a teoría de categorías
algunos conceptos originados sobre conjuntos.

\section{Monomorfismos}
\begin{definition}
Consideremos una categoría $\C$ y una flecha $f : \arr{A}{B}$ de esta.
Diremos que $f$ es un monomorfismo si dados dos elementos de $A$
$x, y \in^T A$ tenemos que $fx = fy \implies x = y$.
\end{definition}

Nótese cómo esta definición es casi idéntica a la definición de
inyectividad sobre aplicaciones entre conjuntos. De hecho es sencillo
demostrar que la noción de monomorfismo sobre $\Set$ se corresponde
efectivamente con las aplicaciones inyectivas.


\paragraph{Ejemplos}
Esta noción se puede aplicar sobre otras categorías. En general
cuando consideramos categorías en la que los objetos son estructuras
matemáticas y las flechas son sus morfismos, los monomorfismos son
aquellas flechas tales que las aplicaciones entre conjuntos subyacentes
son inyectivas. Ejemplos de esto son las categorías
$\Grp$, $\Ring$, ...

\section{Isomorfismos}
\begin{definition}
Consideremos una categoría $\C$ y una flecha $f : \arr{A}{B}$ de
esta. Diremos que $f$ es un isomorfismo si $f$ es una biyección
entre los elementos de $A$ y los elementos de $B$ definido sobre
$T$ para cualquier objeto $T$ de $\C$.
\end{definition}

Esta definición es similar a la definición habitual
de biyección sobre conjuntos pero generalizando los elementos de
los conjuntos sobre elementos de los objeto sobre el resto los elementos
de la categoría.

Nótese que de este concepto podemos dar una definición equivalente
que no requiere del uso de elementos categóricos.

\begin{proposition}
Dada la categoría $\C$ y y una flecha $f: \arr{A}{B}$ tenemos que
$f$ es un isomorfismo sí y solo sí existe una flecha $g: \arr{B}{A}$
tal que $f \circ g = 1_B$ y $g \circ f = 1_A$.
\end{proposition}


Esta caracterización de isomorfismo nos permite ver rápidamente
que los isomorfismos de la categoría $\Set$ se corresponden con las
biyecciones y que los isomorfismos sobre $\Grp$, $\Ring$, $\Top$, ...
se corresponden con los isomorfismos de grupos,
isomorfismos de anillos y homeomorfismos de espacios topológicos.


\section{Productos}
Otra de los ejemplos que dimos anteriormente de construcción que
se realiza sobre conjuntos era el producto cartesiano. Esta definición
podemos llevarla a categorías arbitrarias de la siguiente forma: \textcolor{red}{la redacción de este párrafo es algo farragosa. Yo la simplificaría: Podemos también llevar a categorías cualesquiera la definición de producto cartesiano que hicimos en conjuntos?¿??}

\begin{definition}
Sea $\C$ una categoría y $A$ y $B$ dos objetos de esta. Diremos que
existe el producto de $A$ y $B$ si
existe una terna $(A\times B, \pi_1, \pi_2)$
donde $A\times B$ es un objeto de $\C$ y
$\pi_1 : \arr{A\times B}{A}, \pi_2 : \arr{A\times B}{B}$ son dos flechas tales
que para cualquiera elementos $x : \arr{T}{A}$ de $A$ e
$y: \arr{T}{B}$ de $B$, existe un único elemento
$(x,y): \arr{T}{A\times B}$ de $A\times B$ tal que $\pi_1 (x,y)=x$ y $\pi_2(x,y)=y$.

Expresaremos esto diciendo que  el siguiente diagrama es conmutativo:
\begin{center}
\begin{tikzcd}
&\arrow{ld}[swap]{x} T \arrow[d, dotted, "{\exists! (x, y)}"] \arrow[rd, "y"] & \\
A & \arrow[l, "\pi_1"] A\times B  \arrow{r}[swap]{\pi_2} & B
\end{tikzcd}
\end{center}

\end{definition}

Debemos resaltar dos aspectos importantes de esta definición:
\begin{itemize}
\item El producto de dos objetos en una categoría dada no tiene por qué
      existir.
\item De existir un producto, este solo queda determinado salvo isomorfismo. De hecho si $(A\times B, \pi_1, \pi_2)$ es un producto de $A$ y $B$
y $\phi : \arr{P}{A\times B}$ es un isomorfismo entonces
$(P, \pi_1 \circ \phi , \pi_2 \circ \phi)$ es otro producto de $A$
y $B$.
\end{itemize}


\subsection{Ejemplos}
\paragraph{\Set}
Dados dos conjuntos $A$ y $B$ siempre existe el producto categórico
y además coincide (salvo el isomorfismo
que comentamos antes) con el producto cartesiano de ambos conjuntos
y las proyecciones canónicas. Lo demostramos a continuación:

Sea $a : \arr{T}{A}, b : \arr{T}{B}$ un par de funciones. Podemos
definir la función $(a, b) : \arr{T}{A\times B}$ tal que
$f(x) = (a(x), b(x))$. Esta función cumple en efecto que
$\pi_1 \circ f = a$ y $\pi_2 \circ f = b$. Pero además es la única
que cumple esto pues
$f(x)=(\pi_1\circ f(x), \pi_2\circ f(x))=(f_1(x), f_2(x))$. \textcolor{red}{aquí has hecho una mezcla de antiguo y nuevo que tienes que arreglar, $a$ es $x$, $b$ es $y$, $f$ es $(x,y)$.... además decir que una función $h$ está definida tal que $h(x)=...$ no me hace mucha gracia yo preferiría decir que $h$ está definida como o por $h(x)=....$ Otra cosa que no me hace mucha gracia es hablar de funciones en conjuntos, yo prefiero usar siempre aplicaciones cuando nos referimos a flechas de $\Set$ esto es posible que sea una paranoia pero..... }

\paragraph{Otras estructuras matemáticas}
A continuación mostramos ejemplos de productos categóricos \textcolor{red}{yo eliminaría lo de categóricos simplemente productos....}
en categorías conocidas:

\begin{itemize}
\item En $\Grp$ el producto categórico se corresponde con el
producto directo de grupos.
\item En $\Top$ se corresponde con el producto de espacios topológicos.
\item En $\Ring$ se corresponde con el producto de anillos.
\end{itemize}

\paragraph{Categoría de cuerpos}
En la categoría de cuerpos no existen los productos.
\textit{Demostrar}

\textcolor{red}{no se si puedes probar esto aquí. Ni siquiera se si podrías justificar ( a estar alturas) que el conjunto subyacente al producto de dos cuerpos (si existiese) tendría que ser el producto de los conjuntos subyacentes y en este no hay estructura de cuerpo de manera que las proyecciones sean morfismos?¿?¿?  Es decir no se si a estas alturas puedes justificar que los futores subyacentes a conjuntos conservan productos}


\section{Objetos iniciales}
Al principio de la sección le dimos un papel especial al conjunto
de un solo elemento $*$. Nos gustaría caracterizar a ese objeto de
la categoría $\Set$ con alguna propiedad estrictamente categórica.
Esto es posible:

\begin{definition}
Sea $\C$ una categoría y $I$ un objeto. Diremos que $I$ es un objeto
inicial si dado cualquier objeto $T$ de la categoría solo existe un
elemento de $I$ definido sobre $T$. 
\end{definition}

\textcolor{red}{De nuevo no tengo claro que $I$ sea una buena notación para objeto inicial, yo utilizaría una macro y al final decidiría que poner dentro, mi elección sería algún tipo de uno .... e incluso una $*$ me parece mejor que $I$ }

Esto es cierto sobre $* = \{ 1 \}$: la única flecha que llega 
a $*$ desde cualquier conjunto es la que vale constantemente 1, pero
una vez más esta definición se puede aplicar a otras categorías.

\subsection{Ejemplos}
\paragraph{En $\Grp$}
El objeto inicial en $\Grp$ es el grupo trivial. Solo existe una función
que vaya del grupo trivial a cualquier otro grupo y es la aplicación
nula. Resulta, además, que el objeto final de la categoría es también
el grupo trivial. Si en una categoría tenemos que un objeto
$O$ es a la vez inicial y final diremos que $O$ es un objeto nulo. Esto
es importante porque nos permite definir morfismos nulos. \textcolor{red}{esto no lo puedes hacer aquí....quizás después cuando digas que es objeto final...}
% TODO: definir

\paragraph{En $\Ring$}
En la categoría de los anillos con unidad el objeto inicial es
$\Z$ y el único morfismo que existe de él en cualquier otro anillo
$R$ es el que lleva $1_\Z$ a $1_R$. No existe objeto final en la
categoría de anillos (se razona considerando la característica
del supuesto anillo final). \textcolor{red}{de nuevo final .....}

\section{Dualidad}
El concepto de dualidad que se encuentra frecuentemente en las
matemáticas puede ser analizado de forma muy general desde el punto
de vista categórico. Para introducir esta noción presentamos a
continuación la definición de categoría opuesta.

\subsection{La categoría opuesta}
Dada una categoría $\C$ podemos construir una categoría
$\C^{op}$ en la que definimos $\Obj{\C^{op}} = \Obj{\C}$,
$\Hom_{\C^{op}}(A, B) = \Hom_{\C}(B, A)$ y una operación de
composición dada por $f \circ_{op} g = g \circ f$. Veamos
la construcción realizada describe en efecto una categoría:
\begin{itemize}
\item La composición es asociativa: sean
      $f \in \Hom_{\C^{op}}(A, B)$, $g \in \Hom_{\C^{op}}(B, C)$
      y $h \in \Hom_{\C^{op}}(C, D)$. Tenemos que
\begin{multline*}
(h \circ_{op} g) \circ_{op} f = (g \circ h) \circ_{op} f \\
                              = f \circ (g \circ h)
                              = (f \circ g) \circ h
                              = (g \circ_{op} f)  \circ h
                              = h \circ_{op} (g \circ_{op} f)
\end{multline*}
\item Existen las identidades. Las identidades siguen siendo
      las mismas flechas que son identidad en $\C$. La demostración
      es inmediata.
\end{itemize}

La categoría opuesta nos otorga una herramienta para reaprovechar
definiciones realizadas anteriormente. Veremos ejemplos de propiedades
duales en las siguientes secciones.

\textcolor{red}{aqui tienes que hablar de propiedades duales. Luego yo diría que llamaremos epimorfismos la propiedad dual de monomorfimo, coproducto a la dual de producto, final a la dual de inicial etc.  y luego diría que vas especificar estas propiedades más detalladamente }



\subsection{Epimorfismos}
En una sección anterior especificamos cuando una flecha
$f : \arr{A}{B}$ de una categoría $\C$ era un monomorfismo. La categoría
opuesta nos permite dualizar esta definición de la siguiente manera:

\begin{definition}
Dada una categoría $\C$ y una flecha $f: \arr{A}{B}$, diremos que
$f$ es un epimorfismo si y solo si $f$ es un monomorfismo en
$\C^{op}$.
\end{definition}

En este sentido decimos que las propiedades ``$f$ es un monomorfismo''
y ``$f$ es un epimorfismo'' son propiedades duales. Esta propiedad
se puede enunciar de forma sencilla también sin recurrir a
$\C^{op}$.

\begin{proposition}
$f$ es un epimorfismo si y solo si dado cualquier objeto $X$ y
cualquier par de flechas $g_1, g_2 : \arr{B}{X}$
tenemos que $g_1 \circ f = g_1 \circ f \implies g_1 = g_2$.
Decimos también en este caso que $f$ se puede cancelar por la
derecha.
\end{proposition}

\subsubsection{Ejemplos}
\paragraph{En \Set}
En la categoría de los conjuntos los epimorfismos coinciden con las
aplicaciones sobreyectivas.

\paragraph{Isomorfismos}
Todo isomorfismo es un epimorfismo y un monomorfismo. El recíproco
es cierto en algunas categorías (como en $\Set$ o $\Grp$) pero no
lo es en general.

\paragraph{En \Ring}
Consideremos la categoría de anillos con unidad (en la
que los objetos son anillos con unidad y
las flechas son homomorfismos de anillos). En esta categoría
que $f : \arr{R}{S}$ es equivalente también a que $f$ sea una aplicación
inyectiva.

Sin embargo consideremos la inclusión
$i : \arr{\Z}{\Q}$, un anillo $R$
y un par de aplicaciones de anillos $g_1, g_2 : \arr{\Q}{R}$. Supongamos
que $g_1 \circ i = g_2 \circ i$. Ahora  $\forall x \in \Q$
tenemos que $\exists a, b \in \Z : x = \frac{a}{b}$ y entonces:
\begin{multline*}
g_1(x) = g_1(\frac{a}{b}) = g_1(a)g_1(b)^{-1} \\
       = (g_1 \circ i)(a)(g_1\circ i)(b)^{-1}
       = (g_2\circ i)(a)(g_2\circ i)(b)^{-1}
       = g_2(a)g_2(b)^{-1} = g_2(x)
\end{multline*}

Con lo que $g_1=g_2$. Esto prueba que $i : \arr{\Z}{\Q}$ es epimorfismo
y sin embargo no es sobreyectiva como aplicación. Un isomorfismo en
la categoría de anillos se corresponde con un isomorfismo de anillos
(que en particular es una biyección de conjuntos) y por tanto
en la categoría de anillos se pueden encontrar flechas que son
monomorfismos y epimorfismos pero no son isomorfismos.

\subsection{Objetos finales}
Existe también una propiedad dual a la de ser objeto inicial. \textcolor{red}{toda propiedad tiene su dual, no queda bien decir que existe la propiedad dual de ser inicial, tendrías que decir que la propiedad dual de ser un objeto inicial es ser un objeto final o algo así}

\begin{definition}
Dado un objeto $T$ de una categoría $\C$ decimos que
$T$ es un objeto final si $T$ es un objeto inicial en la
categoría $\C^{op}$.
\end{definition}

Podemos probar una caracterización que no hace referencia a
$\C^{op}$: \textcolor{red}{yo no pondría esto como proposición, diría algo así: Si leemos lo que significa ser objeto final en la propia categoría $\C$ obtendríamos...}

\begin{proposition}
El objeto $T$ es final si y solo si para cualquier otro objeto
$X$ existe una sola flecha tiene a $X$ como dominio y a $T$ como
codominio.
\end{proposition}

\subsubsection{Ejemplos}
\paragraph{En \Set}
En $\Set$ el conjunto vacío $\emptyset$ es un objeto final.

\paragraph{En \Grp}

El objeto final en $\Grp$ es el grupo de un solo elemento. Notemos
que coincide con el objeto inicial. Cuando en una categoría $\C$ coinciden
el objeto inicial y el final $T$ llamamos a $T$ objeto nulo. El objeto
nulo nos permite definir una aplicación nula (\textit{explicar esto mejor}). \textcolor{red}{no tengo muy claro que nulo sea el nombre apropiado para esto. Yo siempre he usado cero}

Otro ejemplo de categoría en la que ocurre esto es en la de espacios vectoriales
sobre un cuerpo $K$.

\paragraph{En \Ring}
\textit{Probar que no existen objetos finales en la categoría de anillos} \textcolor{red}{que me dices del anillo con sólo un elemeno el $0$ que es al mismo tiempo cero y uno?¿?¿ lo que pasa en anillos es que no hay cero, el inicial y el terminal no coinciden}

\subsection{Coproducto}
Podemos repetir la definición de producto categórico sobre la categoría
opuesta y \textit{dando la vuelta} a las flechas para volver a la
categoría inicial $\C$ llegaríamos a la siguiente definición de lo que
llamaremos \textbf{coproducto}: \textcolor{red}{de nuevo yo diría aquí algo así: Coproducto es la propiedad dual de producto, si leemos esta propiedad en la propia categoría $\C$ en lugar de en $\C^{op}$ tendríamaos}

\begin{definition}
Sea $\C$ una categoría y $A$ y $B$ dos objetos de esta. Diremos que
existe el coproducto de $A$ y $B$
si existe una terna $(A+B, i_1, i_2)$
donde $P$ es un objeto de $\C$ y
$i_1 : \arr{P}{A}, i_2 : \arr{P}{B}$ son dos flechas tales
que para cualquier objeto $Y$ y sendas flechas $f_1 : \arr{A}{Y}$,
$f_2 : \arr{B}{Y}$ existe un morfismo
$f : \arr{P}{Y}$ tal que el siguiente diagrama es conmutativo:
\begin{center}
\begin{tikzcd}
&  Y  \arrow[rd, "f_2"]& \\
A \arrow[ur, "f_1"] \arrow{r}[swap]{i_1} &
 \arrow[u, dotted, "\exists! f"] P
 & B \arrow[l, "i_2"]
\end{tikzcd}
\end{center}
\end{definition}

\textit{Retocar esta sección para que sea más consistente con las otras}
\textit{Hablar de que es único salvo isomorfismo}. \textcolor{red}{Quizás deberías decir que la propiedad dual de isomorfismo es de nuevo isomorfismo, esto hace que el carácter \emph{único salvo isomorfismo} que tienen las construcciones definidas por  una \emph{propiedad universal} también lo tengan las construcciones definidas por sus propiedades duales.....}


\subsection{Funtores $\Hom(-, A)$}
\textcolor{red}{Creo que le das demasiadas vueltas a esto. Diría algo así; Dado un objeto $A$ de una categoría $\C$ definimos el funtor $Hom_\C(-,A)$ como $$Hom_\C(-,A)=Hom_{\C^{op}}(A,-):\arr{\C^{op}}{\Set}$$ Si queremos observar el efecto de este funtor sobre $\C$ tenemos que para cada flecha $\arr{C'}{C}$....}



Sea $\C$ una categoría y $A$ un objeto de esta.
Sea $f \in \Hom_{\C^{op}}(C, C')$. Definimos:

$$\Hom_{\C}(f, A) :
  \arr{
    \Hom_{\C}(C, A)
  }{
    \Hom_{\C}(C', A)
  }$$
$$\Hom_{\C}(f, A)(g) = g \circ f$$

(recordando que $g : \arr{C}{A}$ y $f : \arr{C'}{C}$ como
flechas de $\C$ y por tanto se pueden componer)

Se cumple además que:\textcolor{red}{esto es evidente}

\begin{itemize}
\item $\Hom_{\C}(1_C, A)(g) = g \circ 1_C = g$ para toda flecha
  $g\in \Hom_{\C}(C, A)$ y por tanto
  $\Hom_{\C}(1_C, A) = 1_{\Hom_{\C}(C, A)}$ para todo objeto $C$
  de $\C^{op}$.

\item Sean $f \in \Hom_{\C^{op}}(C, D)$ y $g : \Hom_{\C^{op}}(D, E)$
  entonces
  \begin{multline*}
    $$\Hom_{\C}(g \circ_{op} f, A)(h) = h \circ (g \circ_{op} f) \\
      h \circ (f \circ g) = (h \circ f) \circ g = \Hom_{\C}(g, A)(h \circ f) \\
      = \Hom_{\C}(g, A)(\Hom_{\C}(f, A)(h))$$
  \end{multline*}
  para toda $h \in \Hom_{\C}(C, A)$ y por tanto
  $\Hom_{\C}(g \circ_{op} f, A) = \Hom_{\C}(g, A) \circ \Hom_{\C}(f, A)$.
\end{itemize}

Esto nos muestra que $\Hom_\C(-, A)$ es un funtor
$\Hom_\C(-, A) : \arr{\C^{op}}{\Set}$, lo que no resulta
nada sorprendente teniendo en cuenta que
$\Hom_\C(-, A) = \Hom_{\C^{op}}(A, -)$ (que ya vimos que
era un funtor) tanto sobre los objetos
como sobre las flechas. Más aun, se puede demostrar
que $\Hom_\C(-, -) : \arr{\C^{op} \times \C}{\Set}$ es un bifuntor.

Cuando queremos hacer énfasis a que un funtor está actuando sobre
la categoría opuesta a una particular, decimos que es un funtor
contravariante. Por ejemplo, diríamos que $\Hom(-, A)$ es un funtor
contravariante \textcolor{red}{desde la categoría $\C$}.

\section{En programación}
A lo largo de esta sección aplicamos las construcciones que hemos visto
a la categoría \texttt{Hask}.

\subsection{Objetos iniciales y finales}
\texttt{Hask} tiene tanto objetos iniciales como objetos finales. El
objeto final es el tipo con un solo valor, al que se le suele llamar
\texttt{()} (pronunciado \texttt{Unit}). \texttt{()} es un tipo
cuyo único valor es \texttt{()} y para cada tipo \texttt{a} podemos
definir la función:
\begin{verbatim}
constUnit :: a -> ()
constUnit x = ()
\end{verbatim}
Por otro lado tenemos que el objeto inicial de \texttt{Hask} es el tipo
llamado \texttt{Void}, que no tiene ningún valor. En la biblioteca
estándar se encuentra una función llamada \texttt{absurd}
que tiene
como tipo \texttt{Void -> a} y es la única función con este tipo.
En cualquier caso, la función no puede ser llamada puesto que no
hay ningún valor con el que llamarla. La situación tanto como para
el objeto inicial como para el final es muy similar a la que se daba
en \Set.

\subsection{Productos}
\texttt{Hask} tiene productos. Dado dos tipos \texttt{A} y
\texttt{B} podemos construir el tipo \texttt{(A, B)} que tiene
como valores pares de valores de \texttt{A} y de \texttt{B}.
Las proyecciones se implementan en la biblioteca estándar de
Haskell de la siguiente forma:

\begin{verbatim}
fst :: (a, b) -> a
fst (x, y) = x

snd :: (a, b) -> b
fst (x, y) = y
\end{verbatim}

Si tenemos un par de funciones \cod{f1 :: X -> A} y
\cod{f2 :: X -> B} podemos definir la función \texttt{f}:
\begin{verbatim}
f :: X -> (A, B)
f x = (f1 x, f2 x)
\end{verbatim}

\subsection{Coproductos}
En la sección sobre funtores introdujimos el tipo \cod{Either}:
\begin{verbatim}
data Either a b = Left a | Right b
\end{verbatim}
Resulta que \cod{Either} es el coproducto en \cod{Hask}. Si
\cod{A} y \cod{B} son dos tipos de haskell entonces su coproducto
es \cod{Either A B} y las inyecciones canónicas son por un lado
\cod{Left :: A -> Either A B} y por otro lado
\cod{Right :: B -> Either A B}. Si tenemos un par de funciones
\cod{f1 :: A -> Y} y \cod{f2 :: B -> Y} tenemos que la única
función \cod{f :: Either A B -> Y} que se lleva bien con las inyecciones
es:
\begin{verbatim}
f :: Either A B -> Y
f (Left a) = f1 a
f (Right b) = f2 b
\end{verbatim}

Veamos que se lleva bien con las inyecciones:
\begin{verbatim}
(f . Left) a = f (Left a) = f1 a
;; por tanto f . Left = f1

(f . Right) b = f (Right b) = f2 b
;; por tanto f . Right = f2
\end{verbatim}


\chapter{Transformaciones Naturales y el lema de Yoneda}
\section{Transformaciones Naturales}
En el primer capítulo introdujimos los funtores como los morfismos
entre las categorías. Ahora ha llegado el momento de dar un paso más
e introducir las transformaciones naturales como morfismos entre
funtores. Esta es una de las nociones que motivó la creación de la
teoría de categorías: se encuentran ejemplos de esta en múltiples
ramas de las matemáticas.

Procedemos con la definición:

\begin{definition}
  Dados dos funtores $F: \arr{\C}{\D}, G : \arr{\C}{\D}$ decimos
  que $\lambda : \arr{F}{G}$ es una transformación natural si $\lambda$
  asigna a cada objeto $C$ de $\C$ una flecha
  $\lambda_C : \arr{FC}{GC}$ de manera que para
  cualquier flecha $g : \arr{C}{C'}$ el siguiente diagrama
  es conmutativo:
\begin{center}
    \begin{tikzcd}
      F C \arrow{r}{\lambda_C} \arrow{d}[swap]{F g} & GC \arrow{d}{G g} \\
      F C' \arrow{r}{\lambda_{C'}} & G C'
    \end{tikzcd}
  \end{center}
\end{definition}

\textcolor{red}{normalmente se utiliza $\Rightarrow$ para las transformaciones naturales, en lugar de $\rightarrow$}

Los siguientes resultados nos permitirán considerar categorías en las
que los objetos son funtores entre dos categorías $\C$ y $\D$ y las
flechas son las transformaciones naturales entre los funtores:

\begin{proposition}
  \begin{enumerate}
  \item Dado cualquier funtor $F : \arr{\C}{\D}$ podemos definir
    $({1_F})_C  = 1_{(FC)} : \arr{FC}{FC}$. $1_F$ es una transformación
    natural entre $1_F : \arr{F}{F}$. \textcolor{red}{esto no queda  muy bien redactado $1_F$ entre .... }
  \item Podemos componer transformaciones naturales de la siguiente
    forma: dados funtores $F, G, H : \arr{\C}{\D}$ y transformaciones
    naturales $\lambda : \arr{F}{G}$, $\sigma : \arr{G}{H}$ podemos
    definir $\sigma \circ \lambda$ dada \textcolor{red}{podemos definir...dada no cuadra}
    por sus componentes
    $(\sigma\circ\lambda)_C = \sigma_C \circ \lambda_C$.
    $\sigma\circ\lambda$ es una transformación natural
    $\sigma\circ\lambda : \arr{F}{H}$.
  \item La composición de transformaciones naturales
    es asociativa en el siguiente sentido: dados
    $F \xrightarrow{\lambda} G \xrightarrow{\sigma} H \xrightarrow{\tau}I$
    tenemos que $(\tau \circ \sigma) \circ \lambda = \tau \circ (\sigma \lambda)$
  \item Dado cualquier par de funtores $F,G : \arr{\C}{\D}$ y
    transformaciones naturales $\tau : \arr{F}{G}$,
    $\sigma : \arr{G}{F}$ tenemos que $1_F \circ \sigma = \sigma$
    y $\tau \circ 1_F = \tau$.
  \end{enumerate}

  En definitiva los funtores $\arr{\C}{\D}$ y las transformaciones
  naturales entre estos funtores forman una categoría. A esa categoría
  la llamaremos $[\C, \D]$. \textcolor{red}{Supongo que tienes algún motivo para usar esta notación, yo creo que es mucho mejor usar $\D^\C$}
\end{proposition}

\subsection{Ejemplos}
\paragraph{El doble dual}
Consideremos la categoría de espacios vectoriales sobre un cuerpo
$K$ a la que llamaremos $\VectK$. Podemos considerar los endofuntores
identidad $1_{\VectK} : \arr{\VectK}{\VectK}$ y  doble dual:
$(-)^{**} : \arr{\VectK}{\VectK}$. El último de estos funtores actúa
así sobre los morfismos:

$$(-)^{**} : \arr{\Hom(V, W)}{\Hom(V^{**}, W^{**})}$$
$$f^{**}(g)(\phi) = g(\phi \circ f)$$

Para cada espacio vectorial sobre $K$ $V$ \textcolor{red}{uff} podemos además
definir un monomorfismo de espacios vectoriales
 (isomorfismo
en el caso finito-dimensional) en su doble dual:

\begin{equation*}
i_V : \arr{V}{V^{**}}
\end{equation*}
\begin{equation*}
i_V(v)(\phi) = \phi(v)
\end{equation*}

Pues $i$ es una transformación natural entre $1_K$ y
$(-)^{**}$. Más aun, si nos restringimos al caso finito dimensional,
debido a que todas las componentes de la transformación natural
son isomorfismos, decimos que ambos funtores son naturalmente isomorfos. \textcolor{red}{este paraqraph no está muy bien redactado, además naturalmente isomorfos e isomorfos en la categoría de funtores ?¿?¿? a estas alturas no se?¿?? y también la restricción a finito dimensionales uff?¿?¿ estás cambiando los dominios de definición....?¿?¿}

\paragraph{Abelianización de grupos}
Dado un grupo $G$ definimos su abelianización $G^{ab}$ como
$G^{ab} = \frac{G}{[G, G]}$. Llamemos $\pi_G : \arr{G}{G^{ab}}$
a la proyección sobre el cociente. Para cualquier morfismo
de grupos $f : \arr{G}{H}$ tenemos que la aplicación
$\pi_H \circ f$ contiene en su nucleo a $[G, G]$ (ya que
$H^{ab}$ es abeliano y un morfismo de un grupo a un grupo abeliano
lleva los conmutadores al 0) y por tanto factoriza a través
de $G^{ab}$ permitiéndonos definir una aplicación
$f^{ab} : \arr{G^{ab}}{H^{ab}}$ y comprobar
que $(-)^{ab}$ es un funtor. La proyección de $\pi$ es una transformación
natural $\pi : \arr{ 1_{\Grp} }{ (-)^{ab} }$. \textcolor{red}{también tienes que redactar esto un poco mejor}

\section{Lema de Yoneda}
El lema de Yoneda es uno de los primeros resultados
inesperados que surgen a raíz de la teoría de categorías. Una
de las consecuencias más importantes de este resultado es
el \textit{embebimiento} (cambiar) de Yoneda, que nos permite
ver una categoría $\C$ cualquiera dentro de la categoría de
funtores $[\C, \Set]$. Necesitaremos algunos resultados
previos para enunciar el lema:

\begin{proposition*}
  Dada una categoría $\C$ y un objeto suyo
  $A$,
  $$\Nat(\Hom(A, -), -) : \arr{\C \times [\C, \Set]}{\Set}$$
  es un bifuntor donde $\Nat(F, G)$ es el conjunto de transformaciones
  naturales entre los funtores $F$ y $G$.
\end{proposition*}
\begin{proof}
  Consideramos el funtor identidad
  $1_{[\C, \Set]} :\arr{[\C, \Set]}{[\C, \Set]}$ y el funtor
  $\Hom(*, -) : \arr{C}{[\C, \Set]^{op}}$ (demostrar)
  el producto de estos nos define un funtor:
  $$\Hom(*, -)\times 1_{[\C, \Set]} :
  \arr{\C\times [\C, \Set]}{ [\C, \Set]^{op} \times [\C, \Set] }$$
  pero sabemos que
  $$\Nat(-, -) = \Hom_{[\C, \Set]}(-, -) :
    \arr{[\C, \Set]^{op}\times[\C, \Set]}{\Set}$$
    es un funtor. La composición de estos dos funtores
    que hemos dicho es el funtor que buscamos.
\end{proof}


Necesitaremos probar que podemos definir el siguiente funtor:

\begin{proposition*}
  Sea $\C$ una categoría y $Ev(-, -) : \arr{\C\times[\C, \Set]}{\Set}$
  dado por $Ev(C, F) = FC$. $E$ es un funtor y lo llamamos
  \textbf{funtor de evaluación}.
\end{proposition*}
\begin{proof}
  Sean $\sigma : \arr{F}{G}$ y $\tau : \arr{G}{H}$ transformaciones naturales entre funtores
  $F, G, H : \arr{\C}{\Set}$. Sean además $f : \arr{C}{D}$,
  $g : \arr{D}{E}$ flechas de $\C$.
  Podemos definir la acción de $Ev$ sobre las flechas de
  $\arr{\C\times[\C, \Set]}{\Set}$ como
  $Ev(f, \sigma) = \sigma_D \circ Ff : \arr{FC}{GD}$. Veamos que $E$ se comporta bien
  respecto a las composiciones:
  \begin{multline*}
    Ev(g \circ f, \tau\circ\sigma) =
    (\tau \circ \sigma)_E\circ F(g\circ f) =
    \tau_E\circ\sigma_E\circ Fg \circ Ff = \\
  \tau_E\circ Gg \circ\sigma_D \circ Ff =
  \tau_E\circ Gg \circ Ev(f, \sigma) =\\
  Ev(g, \tau) \circ Ev(f, \sigma)
  \end{multline*}

  El hecho de que $Ev(1_C, 1_F) = 1_{FC}$ se prueba de forma sencilla
  y por tanto $Ev : \arr{\C\times[\C, \Set]}{\Set}$ es un funtor

  \textit{Explicar mejor la demostración}
\end{proof}

\begin{theorem}[Lema de Yoneda]
  Sea $\C$ una categoría y $A$ un objeto de esta. Los funtores
  $$(A, F) \mapsto \Nat(\Hom(A, -), F)$$
  $$(A, F) \mapsto FA$$
  son naturalmente isomorfos.
\end{theorem}
\begin{proof}
  Sea $a \in F A$. Definimos la transformación natural entre
  $\lambda_a : \arr{\Hom(A, -)}{F}$ componente a componente de
  la siguiente forma: $(\lambda_a)_C(f) = f(a)$.
  \textit{Continuar mañana}
\end{proof}


\chapter{Adjunciones y Mónadas}
\section{Adjunciones}
De camino a nuestro objetivo de definir el concepto de mónada
pasaremos por la definición de adjunción. Como veremos en las próximas
secciones las mónadas y las adjunciones están estrechamente relacionadas.
Esto no debe hacernos pensar que las mónadas son el único motivo
por el que las adjunciones son importante en matemáticas. Las adjunciones
han sido identificadas por algunos autores como uno de los conceptos
clave de la teoría de categorías. McLane resume la importancia de
esta noción diciendo que ``hay funtores adjuntos por todas partes''
\cite{ariseeverywhere}.

Empezaremos por precisar qué es una adjunción.

\subsection{Definición}
\begin{definition}
  Una adjunción entre categorías $\C$ y $\D$ es un par de funtores
  $F : \arr{\C}{\D}$ y $G: \arr{\D}{\C}$ tales que los funtores
  $\Hom_\D(F-, -) : \arr{\C^{op}\times\D}{\Set}$ y
  $\Hom_\C(-, G-) : \arr{\C^{op}\times\D}{\Set}$ son naturalmente isomorfos.

  Diremos que $F$ es el
  adjunto por la izquierda de $G$ y que $G$ es el adjunto
  por la derecha de $F$. Notaremos esta relación entre ambos funtores
  como $F \adjoint G$.
\end{definition}

Nótese que para cualquier
objeto $C$ de $\C$ y cualquier objeto
$D$ de $\D$ un par de funtores adjuntos $F \adjoint G$ nos permite
establecer biyecciones entre los conjuntos:

$$\Hom_\D(FC, D) \cong \Hom_\C(C, GD)$$

Sea $\phi_{C, D} : \arr{\Hom_\D(FC, D)}{\Hom_\C(C, GD)}$ una tal
familia de biyecciones. Comprobar que $\phi$ es natural se reduce
a, por un lado, comprobar que el siguiente diagrama en $\D$:

\begin{center}
  \begin{tikzcd}
    FC \arrow{r}{f} & D \arrow{r}{g} & D'
  \end{tikzcd}
\end{center}

Se traslada al siguiente diagrama conmutativo en $\C$:

\begin{center}
  \begin{tikzcd}
    C \arrow[bend right=50, swap]{rr}{\phi_{C, D'}(g \circ f)}
      \arrow{r}{\phi_{C,D}(f)} & GD \arrow{r}{G g} & GD'
  \end{tikzcd}
\end{center}

Y por otro lado, que el siguiente diagrama en $\C$:

\begin{center}
  \begin{tikzcd}
    C' \arrow{r}{h} & C \arrow{r}{f} & GD
  \end{tikzcd}
\end{center}

se traslada al siguiente diagrama conmutativo en $\D$

\begin{center}
  \begin{tikzcd}
    FC' \arrow[bend right=50, swap]{rr}{\phi^{-1}_{C', D}(f \circ h)}
        \arrow{r}{Fh} & FC \arrow{r}{\phi^{-1}_{C, D}(f)} & D
  \end{tikzcd}
\end{center}

\subsection{Unidad y counidad}
Dadas categorías $\C$ y $\D$ y funtores adjuntos $F \adjoint G$ es
posible construir dos transformaciones naturales
$\eta : \nat{1_\C}{GF}$ y $\epsilon: \nat{FG}{1_\D}$ a las que llamaremos
respectivamente la \textbf{unidad} y la \textbf{counidad} de la
adjunción.

Sea $\phi : \nat{\Hom(F-, -)}{\Hom(-, G-)}$ el isomorfismo natural que
nos viene dado por la adjunción. Damos la transformación $\eta$
por sus componentes
$\eta_C = \phi_{C, FC}(1_{FC}) : \arr{C}{GFC}$.

Veamos que en efecto es natural. Sea $f : \arr{C}{C'}$
una flecha de $\C$. El diagrama cuya conmutatividad
tenemos que verificar es el siguiente:

\begin{center}
  \begin{tikzcd}
    C \arrow{d}{f}
      \arrow{r}{\eta_C} & GFC \arrow{d}{GFf} \\

    C' \arrow{r}{\eta_{C'}} & GFC'
  \end{tikzcd}
\end{center}

Es decir tenemos que comprobar que
$$\phi_{C', FC'}(1_{FC'}) \circ f = \eta_{C'} \circ f = GFf \circ \eta_C
  = GFf \circ \phi_{C, FC}(1_{FC})$$

Pero teniendo cuenta que $\phi$ es natural tenemos que el diagrama

\begin{center}
  \begin{tikzcd}
    FC \arrow{r}{1_{FC}} & FC \arrow{r}{Ff} & FC'
  \end{tikzcd}
\end{center}

Lo podemos transponer en el siguiente diagrama conmutativo:

\begin{center}
  \begin{tikzcd}
    C
    \arrow[bend right=50, swap]{rr}{\phi_{C, FC'}(f \circ 1_C) = \phi_{C, FC'}(Ff) }
      \arrow{r}{\eta_C}
            & GFC \arrow{r}{GFf} & GFC'
  \end{tikzcd}
\end{center}

Luego $GFf \circ \eta_C = \phi_{C, FC'}(Ff)$ pero a su vez:


\begin{center}
  \begin{tikzcd}
    C \arrow{r}{f} & C'
       \arrow{r}{\eta_{C'}} & GFC'
  \end{tikzcd}
\end{center}

Lo podemos transponer en:

\begin{center}
  \begin{tikzcd}
    FC \arrow[bend right=50, swap]{rr}{\phi_{C, FC'}^{-1}(\eta_{C'} \circ f)}
       \arrow{r}{F f} & FC' \arrow{r}{1_{FC'}} & FC'
  \end{tikzcd}
\end{center}

Este último diagrama muestra que
$Ff = \phi^{-1}_{C, FC'}(\eta_{C'} \circ f)$.


Y por tanto
$$GFf \circ \eta_C = \phi_{C, FC'}(Ff) = \phi_{C, FC'}
  (\phi^{-1}_{C,FC'}(\eta_{C'} \circ f)) = \eta_{C'} \circ f$$

Y la naturalidad de $\eta : \nat{1_\C}{GF}$ queda probada.

\subsection{Ejemplos}
\paragraph{Producto}
Sea $A$ un conjunto. Veamos que los funtores
$-\times A : \arr{\Set}{\Set}$ y
$\Hom(A, -) : \arr{\Set}{\Set}$ conforman una adjunción.
Tenemos que encontrar un isomorfismo natural
$$\phi : \nat{\Hom(-\times A, -)}{\Hom(-, \Hom(A, -))}$$.
Para ello definimos
$$\phi_{B, Z} : \arr{\Hom(B\times A, Z)}{\Hom(B, \Hom(A, Z))}$$
$$\phi_{B, Z}(f)(b)(a) = f(b, a)$$

Veamos que $\phi_{B, Z}$ es una biyección.
En primer lugar vemos que es sobreyectiva.
Sea $f' \in \Hom(B, \Hom(A, Z))$. Definimos la aplicación
$f : \arr{B\times A, Z}$ dada por $f(b, a) = f'(b)(a)$. Aplicando
la definición de $\phi_{B, Z}$
tenemos $\phi_{B,Z}(f)(b)(a) = f(b, a) = f'(b)(a)$ y por tanto
$\phi_{B,Z}(f) = f'$. Veamos que $\phi$ es inyectiva. Supongamos
que tenemos $f_1, f_2 : \arr{B\times A, Z}$ tales que
$\phi_{B, Z}(f_1) = \phi_{B, Z}(f_2)$ pero entonces tenemos que
$f_1(b, a) = \phi_{B, Z}(f_1)(b)(a) = \phi_{B,Z}(f_2)(b)(a) = f_2(b, a)$
y por tanto $f_1 = f_2$.

Comprobamos ahora las condiciones de naturalidad. En primer lugar dado
el siguiente diagrama en $\C$
\begin{center}
  \begin{tikzcd}
    B \arrow{r}{f} & B' \arrow{r}{g} & \Hom(A, Z)
  \end{tikzcd}
\end{center}
Hemos de ver que el siguiente diagrama es conmutativo:
\begin{center}
  \begin{tikzcd}
    {B\times A}
      \arrow[bend right=50, swap]{rr}{\phi^{-1}_{B, Z}(g \circ f)}
      \arrow{r}{f\times A} & B'\times A \arrow{r}{\phi^{-1}_{B', Z}(g)} & Z
  \end{tikzcd}
\end{center}
Y esto equivale a probar
$$\phi_{B,  Z}(g \circ f) = \phi_{B', Z}(g) \circ (f\times A)$$
$$f\times A: \arr{B\times A}{B'\times A}$$
donde $f \times A$ está definida por
$(f\times A)(b, a) = (f(b), a)$. Aplicando sobre un valor cualquiera
$(b, a) \in B\times A$ tenemos que:
$$\phi_{B, Z}^{-1}(g \circ f)(b, a) = (g\circ f)(b)(a) = g(f(b))(a)$$
y
$$(\phi_{B', Z}^{-1}(g) \circ (f\times A))(b, a) =
   \phi_{B', Z}^{-1}(g)(f(b), a) = g(f(b))(a)$$
En segundo lugar dado el siguiente diagrama:
\begin{center}
  \begin{tikzcd}
    B\times A \arrow{r}{f} & Z \arrow{r}{g} & Z'
  \end{tikzcd}
\end{center}
Hemos de ver que el siguiene diagrama es conmutativo:
\begin{center}
  \begin{tikzcd}
    B \arrow[bend right=50, swap]{rr}{\phi_{B, Z'}(g \circ f)}
      \arrow{r}{\phi_{B, Z}(f)} & \Hom(A, Z) \arrow{r}{\Hom(A, g)} & \Hom(A, Z')
  \end{tikzcd}
\end{center}
Es decir, $\phi_{B, Z'}(g \circ f) = \Hom(A, g) \circ \phi_{B, Z}(f)$. Dados
$b \in B, a \in A$ tenemos que:
$$\phi_{B, Z'}(g \circ f)(b)(a) = (g\circ f)(b, a) = g(f(b, a))$$
y
\begin{multline*}
(\Hom(A, g)\circ\phi_{B, Z}(f))(b)(a) = (\Hom(A, g)(\phi_{B, Z}(f)(b))(a)\\
  = (g \circ \phi_{B,Z}(f)(b))(a) = g(f(b, a))
\end{multline*}
Y por tanto $\phi$ es transformación natural.

Esta adjunción nos dice que dar una aplicación que va de $B\times A$ a $Z$
es esencialmente lo mismo que dar una aplicación que vaya de $B$ al conjunto
de aplicaciones entre $A$ y $Z$.
\paragraph{Grupos libres}
El funtor de conjunto subyacente $U : \arr{\Grp}{\Set}$ y el funtor
$F : \arr{\Set}{\Grp}$ que asigna a cada conjunto su grupo libre asociado
son un par de funtores adjuntos $F \adjoint U$.

La unidad $\eta : \nat{1_\Set}{UF}$ es la inclusión
$\eta_X : X \subseteq UFX$ de un conjunto en el conjunto subyacente
al grupo libre generado por él. La counidad
$\mu : \nat{FU}{1_\Grp}$ nos da para cada grupo $G$
el homomorfismo de grupos $\mu_G : \arr{FUG}{G}$
tal que a cada palabra
$g_1g_2\ldots g_n \in FUG$ le asigna el valor $\prod_{i=1}^n g_i \in G$.


\section{Mónadas}
\subsection{Definición}
\begin{definition}
  Una mónada sobre una categoría $\C$
  es una terna $(T, \eta, \mu)$ donde $T : \arr{C}{C}$ es un
  endofuntor y $\eta : \nat{1_\C}{T}$ $\mu: \nat{T^2}{T}$ dos
  transformaciones naturales tales que los siguientes diagramas
  conmutan:


  \begin{center}
    \begin{tikzcd}
      T^3 \arrow{d}{\mu T} \arrow{r}{T\mu} & T^2 \arrow{d}{\mu} \\
      T^2 \arrow{r}{\mu} & T

    \end{tikzcd}

    \begin{tikzcd}
      T \ar[equal]{dr} \arrow{r}{T\eta} & T^2 \arrow{d}{\mu} & T \arrow{l}[swap]{\eta T} \ar[equal]{ld} \\
      & T &
    \end{tikzcd}
  \end{center}


\end{definition}


\chapter{Conclusion}

\printbibliography

\appendix
\chapter{Appendix Title}
\end{document}
